\documentclass[10pt,a4paper]{report}
\usepackage[utf8]{inputenc}
\usepackage{amsmath}
\usepackage{amsfonts}
\usepackage[french]{babel}
\usepackage{amssymb}
\usepackage{enumerate}

\usepackage[cm]{fullpage}
\usepackage{eurosym}
\usepackage{listings}
\usepackage{xcolor}
\usepackage{verbatim}
\usepackage{framed}
\usepackage{ulem}
\usepackage{pigpen}
\usepackage{pifont}

\usepackage{pbox}

\usepackage{graphicx}
\newcommand{\ens}[1]{\lbrace #1 \rbrace}
\newcommand{\abs}[1]{\vert #1 \vert}

\newcommand{\union}{\cup}
\newcommand{\intersection}{\cap}
\newcommand{\mand}{\wedge}
\newcommand{\mor}{\vee}
 
\newcommand{\bigoh}{\mathcal{O}}

\newcommand{\dessin}[1]{\begin{center}\includegraphics[scale=0.6]{images/#1.png}\end{center}}
\newcommand{\dessinS}[2]{\begin{center}\includegraphics[scale=#2]{images/#1.png}\end{center}}

\newtheorem{theorem}{Théorème}[section]
\newtheorem{lemma}[theorem]{Lemma}
\newtheorem{proposition}[theorem]{Proposition}
\newtheorem{corollary}[theorem]{Corollary}

\newcommand{\qed}{\nobreak \ifvmode \relax \else
      \ifdim\lastskip<1.5em \hskip-\lastskip
      \hskip1.5em plus0em minus0.5em \fi \nobreak
      \vrule height0.75em width0.5em depth0.25em\fi}


\newenvironment{proof}[1][Preuve]{\begin{trivlist}
\item[\hskip \labelsep {\bfseries #1}]}{\qed\end{trivlist}}
\newenvironment{definition}[1][Définition]{\begin{trivlist}
\item[\hskip \labelsep {\bfseries #1}]}{\end{trivlist}}
\newenvironment{example}[1][Exemple]{\begin{trivlist}
\item[\hskip \labelsep {\bfseries #1}]}{\end{trivlist}}
\newenvironment{remark}[1][Remark]{\begin{trivlist}
\item[\hskip \labelsep {\bfseries #1}]}{\end{trivlist}}


%% Raccourcis, histoire de ne pas devenir fou avec les notations
\newcommand{\ey}[1]{E_y\{#1\}}
\newcommand{\els}[1]{E_{LS}\{#1\}}

\newcommand{\yh}{\hat{y}}				% y hat
\newcommand{\xh}{\hat{x}}				% x hat
\newcommand{\fh}{\hat{f}}				% f hat

\newcommand{\vary}[1]{var_y\{#1\}}
\newcommand{\varyx}[1]{var_{y\vert \underline{x}}\{#1\}}
\newcommand{\varls}[1]{var_{LS}\{#1\}}

\newcommand{\pyo}{P(\overline{y})}		% P(y barre)
\newcommand{\yo}{\overline{y}}			% y overline
\newcommand{\ao}{\overline{a}}			% a overline
\newcommand{\xo}{\overline{x}}			% x overline

\newcommand{\ab}{\textbf{a}} % a bold
\newcommand{\cb}{\textbf{c}} % c bold
\newcommand{\wb}{\textbf{w}} % w bold
\newcommand{\yb}{\textbf{y}} % w bold

\newcommand{\exy}{\text{E}_{\underline{x}, y}}
\newcommand{\eyx}[1]{\text{E}_{y \vert \underline{x}} \ens{#1}}
\newcommand{\ex}{\text{E}_{\underline{x}}}
\newcommand{\xu}{\underline{x}}			% x underline
\newcommand{\uu}{\underline{u}}			% u underline

\newcommand{\yens}{y_{\text{ens}}}
\newcommand{\yhens}{\hat{y}_{\text{ens}}}

\newcommand{\LS}{\text{LS}}
\newcommand{\TS}{\text{TS}}
\newcommand{\VS}{\text{VS}}
\newcommand{\GS}{\text{GS}}
\newcommand{\TSE}{\text{TSE}}


\newcommand{\specialcell}[2][c]{%
  \begin{tabular}[#1]{@{}c@{}}#2\end{tabular}}

\title{Synthèse/notes \\
Economie monétaire et financière}
\author{Jean-Philippe Collette}

\makeindex
\begin{document}	
	\maketitle
	\tableofcontents
	\chapter{Introduction}

\section{Le système financier}
\textbf{Le marché financier} est le lieu où on échange de la monnaie (aussi appelée des liquidités). C'est un actif qui est plus ou moins liquide.

On y trouve

\begin{itemize}
	\item des agents économiques qui ont besoin de financement et qui demande de la monnaie, par exemple des entreprises, l'Etat.
	\item des agents économiques qui ont des capacités de financement et qui offre de la monnaie, par exemple les ménages.
\end{itemize}

On distingue le marché de la monnaie (où l'actif échangé est de la monnaie) et le marché financier (où l'ensemble des actifs financiers sont considérés).

\dessinS{1}{0.5}

Les banques commerciales ont le monopole de la création monétaire, mais elles sont contrôlées par la banque centrale (BC). A l'opposé, les acteurs du système financier non bancaire ne peuvent pas créer de monnaie.

La BC permet de réguler la monnaie produite. S'il y a trop de monnaie par rapport à la demande, c'est l'inflation, il y a une augmentation des prix.

Les prix sont fixés par taux d'intérêt.

Si une banque a besoin de liquidité, elle prête de la monnaie contre des titres, qu'elle peut revendre quand c'est nécessaire.

\textbf{Un marché de gré à gré} (over-the-counter market) est un marché où il y a un accord sur un prix. On a exactement ce qu'on veut sur mesure et à un prix fixé (CDO : collaterised debt obligation). Le problème est que si ce qu'on veut est trop spécifique, on dit qu'il est trop liquide et peu échangeable. A l'opposé, la monnaie est hyper standardisée.

\section{La monnaie et les autres marchés}

Il y a des interdépendances entre les marchés. Ainsi, un déséquilibre sur les marchés financiers peut entrainer un risque de déséquilibre sur les autres marchés, et inversement. Par exemple, si les banques décident de ne plus octroyer de crédit, les entreprises qui ont besoin de liquidités périclitent.

La monnaie est un moyen de transférer de la valeur du présent vers le futur.

En économie, le futur est incertain. On tente alors de l'anticiper ; les agents économiques tentent de prédire les valeurs futures du marchés. Si ces anticipations sont optimistes, un boom économique est possible. Sinon, une crise peut arriver. Dès que les anticipations varient, on peut faire plonger ou grimper les marchés bousiers, ce qui affecte toute l'économie.

	\chapter{La monnaie}

\section{Les conceptions de la monnaie}

	\subsection{Du troc à la monnaie}
	
	Il y a une conception classique et néoclassique de la monnaie. Dans tous les cas, le but de la monnaie est de fluidifier les échanges.
	
	\textbf{Le troc} est l'échange de biens réels contre d'autres biens réels, sans utilisation de monnaie. L'économie du troc est ainsi non monétaire, cependant on rencontre des difficultés :
	
	\begin{enumerate}
		\item il y a la recherche d'un partenaire d'échange potentiel ;
		\item il faut une double coïncidence des besoins : la personne avec laquelle on troque doit être intéressée par ce qu'on propose et inversement ;
		\item il faut une définition des termes d'échange, autrement dit un prix.
	\end{enumerate}
	
	La monnaie permet de résoudre les deux premières difficultés et de simplifier la troisième.
	
	Par exemple, supposons que l'on ait une économie de trois biens A, B et C. Il faut définir des prix pour toutes les permutations (AB, BA, AC, CA, BC et CB), soit 6 prix (permutation, $3!$), 3 (combinaison, $C_3^2 = 3$) sans de la redondance.
	
	Sans monnaie, il faut trois définitions de terme des échanges pour une économie à trois biens. Si on introduit de la monnaie, par exemple C, on n'a que deux prix à définir : AC et BC.
	
	On peut imaginer trois définition de la monnaie :
	
	\begin{enumerate}
		\item la monnaie est un bien échangeable contre tous les autres biens. Cette action d'échangeabilité est la liquidité.
		
		\textbf{Un actif} est \textbf{liquide} si sa valeur nominale est stable et s'il est convertible immédiatement et sans coût en un moyen de payement.
		
		\item La monnaie est une institution sociale qui permet de fluidifier les échanges entre les acteurs de cette société. Un bien acquière le statut de monnaie si tout le monde croit que ce bien est échangeable contre les autres biens, lorsqu'il y a confiance en la monnaie.
		
		NB : l'Etat n'est pas à l'origine de la monnaie, mais un ensemble de personnes. L'Etat ne peut rien faire si la monnaie n'est pas acceptée.
		
		\item[$\rightarrow$] La monnaie est le stock d'actifs qui peut être immédiatement utilisé pour réaliser des transactions.
	\end{enumerate}
	
	% Cours 18/2/2013
		
	\subsection{La conception keynésienne}
	
	Keynes accepte la conception classique et néoclassique (fluidifier les échanges de biens et services) et y ajoute  la notion de réserve de valeur. Il suppose qu'il y a une incertitude dans l'environnement économique et le fait que La monnaie (au même titre que tous les actifs, par exemple un bien immobilier) peut transférer de la valeur au cours du temps, mais elle est érodée par l'inflation. 
	
	Face à l'incertitude, les agents économiques forment généralement une épargne de précaution, une réserve de valeur qui peut aider à financer une dépense non anticipée. La monnaie est alors demandée pour elle-même. La fonction de réserve de valeur de la monnaie a une influence sur le marché des biens et des services, jouer sur la politique monétaire peut entraîner une fluctuation de l'économie en général. Actuellement, la monnaie n'est plus vraiment utilisée comme réserve de valeur, sauf en temps de crise.
	
	\begin{itemize}
		\item[$\rightarrow$] Classique et néo-classique : monnaie de marché pour échanger des biens et services.
		\item[$\rightarrow$] Dichotomie classique entre variables réelles et variables monétaires : la monnaie n'a pas d'influence sur l'économie réelle.
		\item[$\rightarrow$] Keynes ajoute une dimension supplémentaire : la monnaie est demandée pour elle-même.
		\item[$\rightarrow$] Influence de la monnaie sur les variables réelles.
	\end{itemize}

\section{Les fonctions de la monnaie}

	\subsection{L'unité de compte}
	
	Toutes les dettes sont exprimées en unité de compte (en Belgique et en zone euro : euro). C'est en 1999 que l'euro est une unité de compte mais les monnaies nationales sont encore utilisées. C'est en 2002 que l'euro devient un intermédiaire des échanges. Cela permet de simplifier la 3ème difficulté du troc.
	
	$\rightarrow$ la monnaie établit les termes dans lesquels les prix et les dettes inscrites dans les livres de compte sont exprimées
	
	\subsection{Réserve de valeur}
	
	$\rightarrow$ La monnaie comme réserve de valeur permet le transfert de pouvoir d'achat dans le futur. 
	
	Ce transfert est imparfait car la valeur de la monnaie au cours du temps n'est pas stable : elle dépend de la confiance en la monnaie et de l'inflation. Ainsi, si l'inflation augmente, la confiance en la monnaie diminue, ce qui fait augmenter l'inflation, etc.
	
	\subsection{L'intermédiaire des échanges}
	
	La monnaie est utilisée pour faciliter les transactions de biens et de services. Cela permet de régler les 1ère et 2ème difficultés du troc.
	
\section{Les différentes formes de monnaie}

	\subsection{La monnaie marchandise}
	
	Collectivement, dans les marchés, un bien était élevé à l'état de monnaie (bétail, céréale, les métaux précieux) jusqu'à la seconde guerre mondiale.
	
	\subsection{Monnaie métallique}
		
	Elle a été
		
	\begin{itemize}
		\item pesée,
		\item comptée, et
		\item frappée, d'où l'apparition des pièces de monnaie avec une valeur faciale.
	\end{itemize}
		
	La monnaie frappée a été inventée vers -500/-600 avec JC, par Crésus (royaume de Lydie). Cela permet de différencier la valeur faciale et la valeur intrinsèque (par exemple une pièce de 5g d'or qui n'en contient que 2g). Cela permet d'augmenter la production de monnaie par l'Etat (échange de 5g d'or contre des pièces n'en contenant que 2g). La différence entre la valeur faciale et la valeur intrinsèque s'appelle le seigneuriage, c'était avant tout un moyen de prélever un impôt.
		
	Au fil du temps, cette différence augmente, il y a de moins en moins de métaux précieux dans les pièces mais la confiance reste (cela marche tant qu'il y en a) : c'est la dématérialisation de la monnaie. Vu que c'est totalement dématérialisé, les coûts de production sont faibles et on serait tenté d'inonder le marché de monnaie, au risque de rompre la confiance étable.
		
	Le bimétallisme est le fait de ne considérer que l'or et l'argent, où généralement l'argent permet d'échanger des biens et des services et l'or est utilisé comme réserve. C'est dû à la tendance de vouloir faire partir la monnaie qui vaut le moins (loi de Gresham) : la mauvaise monnaie chasse la bonne.
		
	\subsection{La monnaie de papier ou monnaie fiduciaire}
		
		Ce sont les billets de banque, où la valeur faciale diffère aussi de la valeur intrinsèque. Un billet représente une quantité d'or stocké banque, ils seront au final utilisés pour tous les transferts. La monnaie est \textbf{dématérialisée}.
		
		Jusqu'à la seconde guerre mondiale, il y avait une relation entre le papier et l'or : un billet de banque était un titre certifiant le dépôt d'une certaine quantité d'or dans une banque. Ces titres se sont échangés petit à petit jusqu'à ce qu'il n'y ait plus que ça (ce qui montre la confiance de tous les agents économiques), sans qu'on demande leur conversion en or.
		
		Panique bancaire (bank run) : s'il y a émission de plus de billets de banque qu'il n'y a d'or et que beaucoup trop de personnes viennent faire un échange de leurs billets contre de l'or.
		
	\subsection{La monnaie à cours forcé, à cours légal (fiat money)}
		
	Les autorités gouvernementales décident que la monnaie a cours forcé, c'est-à-dire que les agents économiques sont obligés d'accepter les billets comme moyen de paiement des biens et services.
		
	\subsection{La monnaie scripturale}
		
	C'est la monnaie inscrite au bilan des banques. Elle est plus sûre et pratique que la monnaie fiduciaire, il n'y a pas de transformation à effectuer.
		
	\subsection{La monnaie électronique}
		
	Moyen de stocker de la monnaie au format électronique (cartes prépayées, Proton).
		
\section{Définition technique de la monnaie}

	Comment mesure-t-on la quantité de monnaie en circulation.
	
	Il y a 3 agrégats monétaires en zone euro : M1, M2 et M3.
	
	\begin{enumerate}
		\item[M1]narrow money, la monnaie la plus liquide, qui regroupe
		
		\begin{itemize}
			\item Numéraire : billets et pièces
			\item Dépôts à vue dans les institutions financières : comptes courants
		\end{itemize}
	
		\item[M2] regroupe M1 et
		
		\begin{itemize}
			\item les dépôts à terme ($<$ 2 ans)
			\item les comptes d'épargne
		\end{itemize}
		
		\item[M3] regroupe M2 et des instruments négociables émis par les institutions financières (certificat de dépôt, dépôts à terme, obligations à échéance $<$ 2 ans).
	\end{enumerate}
	
	Plus $i$ est petit et plus M$i$ est liquide.
	
	Pour faire des transactions, il faut M1 (donc convertir M2 et M3). La banque centrale surveille surtout M3 et les taux de conversion en M1, car les prix augmenteront s'il y a une conversion soudaine de M3 en M1 (il y a plus de monnaie en circulation).
	
	Comment mesurer la quantité de monnaie dans l'économie : la monnaie est un stock d'actifs, c'est-à-dire que pour mesurer la quantité de monnaie il suffit de mesurer le stock d'actifs en circulation, donc mesurer la masse monétaire.

	\chapter{La demande de monnaie}

\section{La théorie quantitative de la monnaie (TQM)}

	Dans la théorie classique et néo-classique, le principe de base est que la monnaie est neutre à long terme. Il y a également une dichotomie entre les variables réelles et les variables nominales (déterminées par l'offre et la demande et par les prix relatifs). La TQM de Fisher va expliquer la détermination des variables nominales.
	
	\subsection{L'équation des échanges de Fisher (1911)}
	
	$$MV = PT$$
	
	\begin{itemize}
		\item $T$ : nombre de transactions sur les marchés  ; au cours d'une année, c'est le nombre de fois que l'on échange de la monnaie contre des biens et des services.
		\item $M$ : quantité de monnaie en circulation dans l'économie.
		\item $V$ : vitesse de circulation de monnaie ; nombre de transactions effectuées par unité monétaire, soit le nombre de fois qu'un billet change de main au cours d'une période. Lors d'inflation, la monnaie circule plus vite car on cherche à s'en débarrasser.
		\item $P$ : prix moyen des transaction ; nombre moyen d'unités monétaires (UM) échangées par transaction.
		
		$\rightarrow$ $PT$ est la valeur en euros de ce qui a été échangé sur les marchés au cours d'une année, c'est la valeur nominale des transactions.
	\end{itemize}
	
	Exemple : supposons que 100 cannettes de bière soient vendues à 1 euro la pièce au cours d'une année.
	
	$\rightarrow T = 100\text{ / an, } P = 1 \text{ euro par cannette}$
	$\rightarrow P \times T = 100 = \text{ nombre total d'euros échangés}$
	
	Supposons que la quantité de monnaie en circulation est de $20$ euros. La vitesse de circulation $V = \frac{P \times T}{M} = \frac{100}{20} = 5$. Chaque euro change donc 5 fois de main sur un an.
	
	\paragraph{Augmentation de la masse monétaire}
	
	Si la masse monétaire passe de 20 à 40 euros, il y a 3 possibilités :
	
	\begin{enumerate}
		\item réduction de la vitesse, ou
		\item les prix vont augmenter (à 2 euro), donc la vitesse ne va pas bouger, ou
		\item la vitesse et les prix ne changent pas mais la production augmente (elle passe à 200/an).
	\end{enumerate}
	
	Si on suppose que les firmes sont au maximum des capacités de production, la 3ème hypothèse tombe. Avec la définition classique et néoclassique, on suppose que la vitesse est stable, donc le prix va augmenter.
	
	\paragraph{Intégration de la production}
		
	On va remplacer le nombre de transaction par la production. Comme il est impossible de calculer toutes les transactions, $T$ est approximé par $Y$ le PIB réel. L'équation devient
	
	$$M \times V = \underbrace{P \times \underbrace{Y}_{\text{PIB réel}}}_{\text{PIB nominal}}$$
	
	\begin{itemize}
		\item $PY$ : PIB nominal distribué sous forme de revenu, en salaire et dividende de capital.
		\item $V$ : vitesse de circulation de la monnaie par euro de revenu.
	\end{itemize}
	
	Cette équation des échanges, sans hypothèses sur les variables, est une identité : on ne sait pas ce qu'il va se passer si une des variables change. Par exemple, dans la théorie quantitative, si $M$ augmente, $P$ augmente.
		
	Nb : $Y \neq T$ car la production n'est pas proportionnelle aux transactions (il faut considérer les ventes de seconde main, ce qui n'est pas connu), mais $Y$ et $T$ varient proportionnellement.
	
	% Cours 25/2/2013
	
	\subsubsection{La TQM comme fonction de demande de monnaie}
	
	Par Marshall et Pigou, qui font clairement une différence entre l'offre ($M$) et la demande ($M^d$) de monnaie :
	
	\begin{itemize}
		\item la demande de monnaie, que les agents économiques expriment pour des achats de biens et services.
		
		$$M^d =k.PY$$
		
		$M^d$ est la demande d'encaisse monétaire et $k$ est la part de la valeur nominale du PIB que les agents économiques souhaitent détenir sous forme de monnaie.
		
		\item l'offre de monnaie $M$ : exogène, fixée par les autorités monétaires.
	\end{itemize}
	
	\paragraph{Equilibre du marché de la monnaie} La condition d'équilibre est que offre ($M$) = demande ($M^d$). On a donc 
	
	$$M^d = M = k.PY$$
	
	Cette théorie de demande de monnaie permet d'expliquer le niveau des prix observé dans l'économie.
	
	Ici, $Y$ est la production de biens et services ; les facteurs de production (quantité et prix) et la technologie. $Y$ est déterminé en dehors du modèle.
	
	$M$ : l'offre de monnaie est exogène. Si les autorités monétaires augmentent $M$, l'équilibre est rompu car $M > M^d$. Supposons que $k$ (variable comportementale) est stable, ce sont alors les prix $P$ qui vont rétablir l'équilibre en augmentant.
	
	En conclusion, la monnaie est neutre à long terme, elle ne permet pas de modifier la production de biens et de services. Le seul effet à long terme est la variation des prix.
	
	\paragraph{Variation des paramètres}
	
	\begin{itemize}
		\item si $PY$ augmente :
		
		\begin{itemize}
			\item[$\Leftrightarrow$] on doit réaliser plus de transactions
			\item[$\Leftrightarrow$] on doit demander plus d'encaisses monétaires
			\item[$\Leftrightarrow$] $M^d$ augmente, en supposant que $k$ est constant et $M$ exogène.
		\end{itemize}
		
		\item si $M$ augmente, $P$ augmente
		\item si $M$ augmente, $M > M^d$, par conséquence la demande de biens et de services augmente, mais la production reste inchangée : la demande de biens est plus grande que l'offre de biens, donc le prix augmente.
	\end{itemize}
	
	\paragraph{Liens entre les deux versions de la TQM}
	
	Pour Fisher, $MV = PY$, alors que pour Marshal et Pigou, $M = kPY$. Si on soustrait, on a $V = \frac{1}{k}$.
	
	Si $k$ est élevé, une grande part du revenu est détenue en monnaie, alors la vitesse de circulation va diminuer. Si $k$ est faible, la  vitesse de circulation devra être élevée pour permettre l'ensemble des transactions.
	
	\subsection{La dichotomie classique et la neutralité à long terme de la monnaie}
	
	\dessin{2}
	
	\begin{itemize}
		\item $L$ : quantité de travail
		\item $\frac{W}{P}$ : le salaire réel
		\item $O_L$ : offre de travail (émise par les salariés)
		\item $D_L$ : demande de travail (émise par les entreprises)
	\end{itemize}
	
	
	\paragraph{Sur le marché du travail} L'offre et la demande de travail déterminent le salaire réel d'équilibre $\frac{W_0}{P_0}$ et la quantité de travail d'équilibre $L_0$.
	
	\paragraph{Sur le marché de la production} Le facteur de production $L_0$ est fixé. La technologie est exogène, donc $Y_0$ est déterminé à l'intersection de $L_0$ et $Y$. Le niveau de production est déterminé par le travail.
	
	\paragraph{Sur le marché des biens et des services} L'offre est déterminée par les facteurs de production et la technologie uniquement. Elle ne dépend pas du prix, on a donc une droite verticale en $Y_0$, $O_A$.
	
	La demande agrégée de biens et services ($D_A$) dépend du niveau général des prix, dès lors $\frac{dD_A}{dP} < 0$.
	
	On suppose que les autorités ont déterminé la masse monétaire $M_0$. Supposons qu'elles décident d'augmenter la masse monétaire à $M_1$ :
	
	\begin{itemize}
		\item les encaisses monétaires augmentent.
		\item la demande agrégées de biens et de services augmente. L'offre agrégée ne dépend pas de l'offre de travail et de la technologie, donc reste constante.
		\item le prix augmente
	\end{itemize}
	
	
	\paragraph{Courbe du salaire nominal}
	
	Plus les prix sont élevés, plus le salaire réel (le pouvoir d'achat) est faible. Si $P_0$ passe à $P_1$, $\frac{W_0}{P_0}$ descend à $\frac{W_0}{P_1}$, ce qui a pour conséquence que
	
	\begin{itemize}
		\item moins de travailleurs sont prêts à travailler pour ce prix
		\item excès de la demande sur l'offre
		\item $O_L < D_L$
		\item le salaire réel doit augmenter
		\item le prix augmente
		\item les entreprises doivent proposer un salaire nominal plus élevé, donc $W_0$ monte à $W_1$
	\end{itemize}
	
	\paragraph{Conclusion}
	
	La hausse de la quantité de monnaie n'a aucun effet sur la production de biens et services ($Y_0$), sur le salaire réel ($\frac{W_0}{P_0}$ = $\frac{W_1}{P_1}$) ou sur le niveau d'emploi ($L_1$). On a donc bien la neutralité de la monnaie à long terme. Le seul effet de l'augmentation de $M$ se traduit sur les prix.
	
	Nb : on sous-entend que $V$ est stable.
	
	\bigbreak
	\begin{center}
	-- Anciennes notes --
	\end{center}
	
	Si $\frac{W}{P}$ augmente, $O_L$ augmente et $D_L$ diminue, donc l'offre de travail est une fonction croissante du salaire réel, alors que la demande de travail est une fonction décroissante.
	
	L'équilibre du travail (offre $O_L$ = demande $D_L$) définit le salaire réel d'équilibre $\frac{W_0}{P_0}$ et la quantité d'équilibre $L_0$.
	
	La frontière technologique est représentée par une fonction de production $Y = A F(K, L)$, avec $Y$ la quantité de biens et services, $F$ la fonction, $K$ le capital, $L$ le travail et $A$ la productivité totale des facteurs (niveau technologique de notre économie). $Y_0$ est le niveau optimal de production nationale à la fois pour les entreprises et pour les consommateurs.                                                                                                                                                                            
	
	$OA$ : offre agrégée, verticale car elle ne dépend pas des prix mais bien de facteurs de production.
	
	$DA_0(M_0)$ est la demande agrégée :
	
	\begin{itemize}
		\item dépend de la masse monétaire en circulation $M_0$
		\item dépend aussi des prix
		\item[$\rightarrow$] la demande augmente quand les prix diminuent
	\end{itemize}
	
	Equilibre du marché des biens : $OA = DA$. Le niveau des prix des biens et services est déterminé par l'équilibre du marché des biens et service.
	
	Supposons que les autorités monétaires augmentent la masse monétaire de $M_0$ à $M_1$ (en vert), donc les agents économiques vont vouloir détenir plus de monnaie et échangeront cette monnaie contre des biens et services. Comme l'offre de biens et services est déterminée par des facteurs réels (donc restera intacte), cette hausse de la demande agrégée ne conduit qu'à la hausse des prix. Si les prix augmentent, le salaire réel va diminuer, l'offre de travail diminue et la demande du travail, on a un déséquilibre sur le marché du travail, un écart entre $L_E$ et $L_S$, qui exerce une pression sur le salaire nominal. Ce déséquilibre favorise l'offre, les entreprises vont augmenter les salaires minimales pour retrouver l'équilibre ($W_1$).
	
	En conclusion, la hausse de quantité de monnaie n'a eu aucune conséquence sur la production de biens et service, sur le salaire réel et sur les niveaux d'emploi : c'est la neutralité de la monnaie à long terme.
	
	\bigbreak
	\begin{center}
	-- Fin anciennes notes --
	\end{center}
	
	\bigbreak
		
	\paragraph{Application dynamique}
	Prenons l'hypothèse que $V$ est stable. Si les autorités monétaires augmentent la masse monétaire entre $t$ et $t + 1$.
	
	En $t$, $M_t V_t = P_t Y_t$, en $t + 1$, $M_{t + 1} V_{t + 1} = P_{t + 1} Y_{t + 1}$. On a $1 + \text{taux} = \frac{M_{t + 1} V_{t + 1}}{M_t V_t}$. Si on prend le logarithme :
	
	$$(\ln M_{t + 1} + \ln V_{t + 1}) - (\ln M_t + \ln V_t) = (\ln P_{t + 1} + \ln Y_{t + 1}) - (\ln P_t + \ln Y_t)$$
	$$\Leftrightarrow \underbrace{\ln M_{t + 1} - \ln M_t}_{\text{taux de variation de la masse monétaire}} + \underbrace{\ln V_{t + 1} - \ln V_t}_{0\text{ par hypothèse}} = \underbrace{\ln P_{t + 1} -\ln P_t}_{\text{taux d'inflation}} + \underbrace{\ln Y_{t + 1} - \ln Y_t}_{\text{taux de croissance du PIB réel}}$$


% 27/02/2013
\section{La théorie keynésienne de la demande de monnaie}

	Keynes remet en cause la stabilité de la demande de monnaie:  pour lui, elle est instable.
	
	\subsection{Le marché du travail et la neutralité de la monnaie}
	
	\dessinS{3}{.2}
	
	%\dessinS{4}{.2}
	
	
	Soient
	
	\begin{itemize}
		\item $L_E$ la demande de travail des entreprises
		\item $L_O$ l'offre de travail au prix $P$
		\item $L_O - L_E$ est le nombre de chômeurs (involontaires, car ils veulent bien travailler mais les entreprises ne veulent pas les embaucher au salaire $\frac{W_0}{P_0}$)
	\end{itemize}
	
	Le point de départ est l'équilibre en $(L_0, \frac{W_0}{P_0})$ et $(Y*, P_0)$. On suppose que la demande agrégée sur le marché des biens diminue (pendant la crise de 1929 par exemple) ; on passe de $D_{A_0}$ à $D_{A_1}$.  Vu que $D_{A_0} < D_{A_1}$, $D_{A_1} < O_A$, les prix doivent diminuer (de $P_0$ à $P_1$).
	
	Keynes conteste la flexibilité des salaires nominaux ; pour lui, vu qu'ils sont négociés à long terme, ils sont rigides, donc si $P$ augmente, $\frac{W_0}{P_0}$ augmente, donc $\frac{W_0}{P_0} <\frac{W_0}{P_1}$, le marché du travail est alors en déséquilibre. Les entreprises vont donc moins embaucher et la production va diminuer, de $Y*$ à $Y_C$, car elles produisent $L_E$ au lieu de $L_O$, ce qui va modifier l'offre agrégée.
	
	On a deux situations :
	
	\begin{itemize}
		\item $Y < Y*$ et un salaire nominal rigide : l'économie est en dessous de son niveau optimal et l'offre agrégée $OA$ dépend du prix. Si $P$ augmente, le salaire réel diminue (à salaire nominal constant), les entreprises ont un incitant à embaucher puisque le salaire réel $\frac{W}{P}$ est plus faible. Elles peuvent embaucher car il y a des chômeurs. La production $Y$ va donc augmenter. On a bien un effet des prix sur l'offre.
		
		\item lorsque $Y \geq Y*$ et un salaire nominal pas nécessairement rigide, si le prix augmente alors $\frac{W}{P}$ diminue. Les entreprises ont un incitant à embaucher mais elles ne peuvent plus embaucher car il n'y a pas de chômeurs. $Y$ ne change plus et donc l'offre agrégée $OA$ n'est plus dépendante du prix $P$, on retrouve donc la verticalité de $OA$, qui ne dépendra plus du prix. La masse monétaire ne doit plus être augmentée.
	\end{itemize}
	
	
%	 A court terme, les salaires vont augmenter, les entreprises vont donc moins embaucher, donc le chômage va augmenter car il y a un déséquilibre; $L_C < L_D$, le nombre de chômeurs est de $L_S - L_E$ (chômage involontaire ; les salariés aimeraient travailler au salaire en vigueur $\frac{W_0}{P_1}$, ce que les entreprises ne veulent pas). Alors qu'on aurait une renégociation du salaire nominal, il est ici rigide à la baisse à court terme. Cela a pour conséquence :
%	
%	\begin{itemize}
%		\item le chômage va durer
%		\item la production va diminuer de $Y_0$ à $Y_E$ car les entreprises produisent avec $L_E$ au lieu de $L_O$. L'offre agrégée est donc modifiée
%	\end{itemize}
%	
%	On a deux situations :
%	
%	\begin{itemize}
%		\item $Y < Y_0$ : l'économie est en dessous de son niveau optimal. L'offre agrégée $OA$ dépend du prix. Si $P$ augmente, le salaire réel diminue (à salaire nominal constant), les entreprises ont un incitant à embaucher puisque le salaire réel $\frac{W}{P}$ est plus faible. Elles peuvent embaucher car il y a des chômeurs. La production $Y$ va donc augmenter. On a bien un effet des prix sur l'offre.
%		\item lorsque $Y \geq Y_0$, si le prix augmente alors $\frac{W}{P}$ augmente. Les entreprises ont un incitant à embaucher mais elles ne peuvent plus embaucher. $Y$ ne change plus et donc l'offre agrégée $OA$ n'est plus dépendante du prix $P$, on retrouve donc la verticalité de $OA$, qui ne dépendra plus du prix. La masse monétaire ne doit plus être augmentée.
%	\end{itemize}
	
	
	On a deux réponses politiques possibles :
	
	\begin{itemize}
		\item Classique : diminuer les salaires nominaux
		\item Keynésiennes : augmenter la masse monétaire
	\end{itemize}
	La politique économique sera d'utiliser la politique monétaire, c'est-à-dire augmenter les prix en augmentant la quantité de monnaie en circulation, ce qui fera diminuer les salaires réels.
	
	Conséquences d'une augmentation de masse monétaire :
	
	\begin{itemize}
		\item (directe) augmentation des prix
		\item (indirecte) baisse du salaire réel.
	\end{itemize}
	
	Si le salaire réel diminue, les entreprises embauchent les chômeurs qui sont sur le marché du travail, donc la production augmente ainsi que la demande agrégée (car les agents économiques détiennent plus d'encaisses monétaires pour acheter des biens et services) :
	
	$$M \uparrow \Rightarrow \frac{W}{P} \downarrow \Rightarrow Y \uparrow \text{ et } DA \uparrow$$
	
	L'accroissement de la masse monétaire a permit de rétablir l'équilibre de plein emploi par l'augmentation des prix. Lorsque $Y < Y_0$, la variation de la quantité de monnaie a des effets réels. Dans ce cas, la monnaie n'est plus neutre.
	
	Remarques :
	
	\begin{itemize}
		\item On a supposé que l'économie était fermée. Si elle est ouverte, il y aura une baisse de la compétitivité du pays. Elle peut être récupérée si le taux de change est variable. En zone euro, le taux est fixe.
		\item Cela ne peut pas marcher en Belgique ou au Luxembourg car le salaire est indexé sur les prix (si les prix augmentent, le salaire aussi).
	\end{itemize}
	
	
	
	\subsection{La fonction de demande de monnaie : la préférence pour la liquidité}
	
	Innovation majeure de Keynes : introduction du taux d'intérêt dans la fonction de demande de monnaie. Il voit trois motifs pour que les agents économiques choisissent de détenir de la monnaie :
	
	\begin{enumerate}
		\item \underline{motif de transaction} (idem que les classiques) : on veut de la monnaie pour effectuer des transactions.
		
		$$\frac{M^D}{P} = L(Y)$$
		
		\begin{itemize}
			\item $M^D$ est la demande de monnaie
			\item $P$ le prix
			\item $\frac{M^D}{P}$ est le pouvoir d'achat
			\item $L$ est la fonction de préférence pour la liquidité
			\item $Y$ le revenu courant.
		\end{itemize}
			
		Si $Y$ augmente, la demande de monnaie augmente.
		
		\item \underline{motif de précaution} : les agents économiques demandent de la monnaie pour des transactions mais aussi comme épargne, pour faire face à des imprévus car l'avenir est incertain.
		
		La demande d'encaisse dépend du revenu, car il est plus facile d'économiser quand on a un revenu important.
		
		\item \underline{motif de spéculation} : la demande de monnaie dépend du taux d'intérêt.
	\end{enumerate}
	
	2 actifs monétaires sont possibles : la monnaie (pas de rémunération) et les obligations (rapportent un taux d'intérêt mais moins liquide que la monnaie).
	
	Le taux d'intérêt est le prix de la renonciation à la liquidité. Plus il est élevé, plus les agents économiques ont des obligations. Plus il est faible et plus ils ont des liquidités.
	
	Keynes pensait qu'il existait un taux d'intérêt de long terme normal. Deux situations peuvent survenir :
	
	\begin{enumerate}
		\item $r > r_{\text{normal}}$ : les agents économiques s'attendent à ce que le taux d'intérêt baisse dans le futur et à ce que le prix des obligations augmentent. Cette demande d'obligations va donc augmenter et la demande de monnaie va baisser.
		
		\item $r < r_{\text{normal}}$, les agents s'attendront à une hausse du taux d'intérêt et donc à une baisse du prix des obligations. La demande des obligations va donc baisser.
	\end{enumerate}
		
	
	Quantitative easing : aux USA, les taux sont trop bas. Les banques vont alors acheter des obligations pour relancer le marché obligataire. 
	
	La demande de monnaie est inversement proportionnelle au taux d'intérêt. On a
	
	$$\frac{M^D}{P} = L(r, Y)$$
	
	\dessinS{5}{.4}
	
	Si on prend l'inverse de la fonction et qu'on la multiplie par $Y$, on a
	
	$$\frac{PY}{M^D} = \frac{Y}{L(r, Y)}$$
	
	A l'équilibre du marché monétaire, $M = M^D$ et le membre de droite devient $\frac{PY}{M}$, qui est $V$ dans l'équation de Fisher : $MV = PY$. On a donc
	
	$$V = \frac{Y}{L(r, Y)}$$
	
	On a donc que la vitesse de circulation de la monnaie dans l'économie varie en fonction de $r$ ; si $r$ est instable, $L(r, Y)$ sera aussi instable.
	
	2 réponses à la critique selon laquelle la monnaie n'est plus vraiment une réserve de valeur :
	
	\begin{itemize}
		\item Baumol-Tobin : on ne va pas au guichet bancaire chaque fois qu'on doit faire une transaction.
		\item Tobin (1958) : introduit le motif de spéculation de la monnaie en disant que la monnaie est un actif peu risqué et donc permet d'optimiser le risque agrégé du porte-feuille d'actifs.
	\end{itemize}
	
	
\section{La version moderne de la TQM : Milton Friedman (1912-2006)}

	\subsection{La fonction de demande de monnaie de Friedman}
	
	Pourquoi les agents économiques veulent-ils avoir de la monnaie ? Friedman pose que c'est une demande d'un actif parmi d'autres actifs financiers et des actifs non financiers (bien durables, par ex un bien immobilier, un terrain).
	
%	- Contraintes de ressources : 
%	- Prix des actifs
	Il définit la contrainte de richesse globale : 
	
	\begin{itemize}
		\item la somme actualisée des revenus futurs de l'individu est utilisée pour calculer la richesse globale d'un individu au cours de sa vie.
		\item le rendement de la monnaie est comparé par rapport au rendement des autres actifs.
		\item Les goûts et les préférences de l'individu
	\end{itemize}
	
	L'allocation de la richesse globale entre ces différents actifs se fait en fonction des taux de rendement relatifs de ces différents actifs :
	
	$$\frac{M^D}{P} = f(Y^P, r_b - r_m, r_e - r_m,\pi_e - r_m, u)$$
	
	\begin{itemize}
		\item $\frac{M^D}{P}$ est la demande d'encaisse réelle
		\item  $Y^P$ le revenu permanent (valeur actualisée de tous les revenus futurs)
		\item $r_m$ le rendement attendu de la monnaie (rémunération des comptes courants + les services rendus par la monnaie, c'est-à-dire la possibilité de faire des achats)
		\item $r_b$ le rendement attendu/espéré des obligations,
		\item $r_e$ le rendement attendu des actions
		\item $\pi_e$  le taux d'inflation anticipé
		\item $u$ : autres facteurs
	\end{itemize}
	
	On a que
	
	\begin{itemize}
		\item Si $Y^P \uparrow \Rightarrow$ demande de monnaie $\uparrow$, $r_b - r_m$ ou $r_e - r_m \uparrow \Rightarrow f \downarrow$
		\item Si $\pi_e \uparrow$, les prix des biens durables augmentent, le pouvoir d'achat de la monnaie diminue et donc la demande de monnaie diminue.
		
		\item La demande de monnaie sera d'autant plus grande que le niveau de richesse $Y^P$ sera élevé, que le rendement des autres actifs sera faible et que $\pi_e$ sera faible. 
		
		\item[$\rightarrow$] les individus ajusteront l'allocation de leur richesse globale jusqu'à ce que les taux de rendement marginaux des différents actifs s'égalisent.
	\end{itemize}
		
	
	On obtient donc l'équilibre quand les taux de rendement marginaux des différents actifs sont égaux. De plus, Friedman considère que la monnaie est un substitut à des actifs financiers réels, mais aucun actif ou ensemble d'actifs ne peut être un substitut parfait à la monnaie.
	
	Si la banque centrale injecte des liquidités ($M \uparrow$) en achetant des obligations vendues par les banques, le rendement de l'actif monnaie diminue et les taux de rendement ne seront plus égaux. $r_m \neq r_b \neq r_e \neq \pi_e$, $r_m < r_b, re, \pi_e$. Les agents économiques vont alors acheter des obligations, des biens et des services et d'autres actifs financiers. Si la demande des actifs augmente, le prix de ces actifs augmente.
	

	\subsection{Comparaison Keynes/Friedman}
	
\begin{center}
 	\begin{tabular}{|c|c|}
	\hline 
	Keynes & Friedman \\ 
	\hline 
	Demande de monnaie $\frac{M^D}{P}$ instable & Demande de monnaie $\frac{M^D}{P}$ stable \\ 
	\hline 
	\specialcell{Contrôler l'offre de monnaie \\ risque d'échouer à cause \\ de cette instabilité} & \specialcell{Contrôler l'offre de monnaie \\ pour contrôler les prix} \\  
	\specialcell{$\Rightarrow$ contrôle du taux d'intérêt \\ pour contrôler l'inflation} &  \\ 
	\hline 
	\end{tabular}
 \end{center} 
 
 	Pour Keynes, la politique des taux est efficace vu que c'est $r$ qui fait varier la demande de monnaie, et donc la masse monétaire. On va donc fixer $r*$.
 	
 	Pour Friedman, la politique monétaire est monétariste, on se fixe des objectifs d'offre de monnaie, la croissance de $M$ est donc fixée à l'avance. On va donc fixer $M*$. Cette politique était correcte à son époque car la vitesse $V$ était stable.
	% 11/03/2013
\part{Les crises financières}

\section{Les facteurs de la crise financière}	
	
	\subsection{L'effet du marché des actifs sur les bilans}
	
		\subsubsection{Chute du marché boursier}
	
		Une forte baisse des cours boursiers peut gravement détériorer le bilan des entreprises qui empruntent. En effet, une baisse des cours diminue la capitalisation boursière (nombre d'actions émises par une entreprise multiplié par le cours de bourse). Plus cette capitalisation diminue, plus les actifs nets (notamment le fond propre (capital dont on déduit les frais d'établissement)) diminuent et donc plus l'entreprise aura des difficultés à obtenir des crédits.
	
		Par la suite, cette détérioration peut accroître les problèmes d'anti-sélection et de risque moral sur les marchés financiers et provoquer une crise financière. 
	
		Les banques ont normalement un rôle d'intermédiation financière : elles accumulent des liquidités et les prêtent aux entreprises et ménages pour faire tourner l'économie. Afin de réduire les risques, des banques peuvent diminuer leur octroi de crédit et en refuser, sauf à des entreprises très solides. En diminuant ces offres de crédit, elles ne jouent plus leur rôle d'acteur dans l'économie.
	
		Chute de la bourse $\rightarrow$ banques prêtent moins $\rightarrow$ les entreprises investissent moins.

		Risque moral : risque que le bilan d'une entreprise se détériore après qu'elle ait obtenu un crédit.
	
		Accords de Baltroy : demande aux banques de quantifier les risques. Plus il y en a, plus du capital doit être mobilisé. Ainsi, plus une banque accorde des crédits à des entreprises de mauvaise qualité, plus les risques sont élevés et plus les augmentations de capital seront nécessaire.
		
		\subsubsection{Baisse non anticipée du niveau des prix}
		
		Dans les économies où l'inflation est restée modérée, les contrats de dette sont souvent à maturité éloignée et à taux d'intérêt fixe. Une baisse non anticipée du niveau général des prix augmente la valeur en termes réels des dettes de l'entreprise, c'est-à-dire qu'elle accroît la charge de la dette. 
		
		La charge de la dette est plus importante si le taux d'intérêt est fixe. Par exemple, une entreprise peut vendre un bien puis en diminuer le prix : avec un taux fixe, la charge est plus importante après la baisse. Avec un taux variable, on n'a qu'un pourcentage des revenus ou autre.
		
		La dette est plus importante, donc le bilan des entreprises se détériore, ce qui les privent de crédit, donc de capacités d'investir.
		
		\subsubsection{Baisse non anticipée de la valeur de la monnaie nationale}
		
		Si les contrats de dette sont libellés en monnaie étrangère, toute baisse non anticipée de la valeur de la monnaie nationale augmente la charge des entreprises et conduit à une baisse de l'activité économique. Si les contrats de dette sont libellés en monnaie étrangère, toute baisse non anticipée de la valeur de la monnaie nationale augmente la charge des entreprises et conduit à une baisse de l'activité économique.
		
		Pour une entreprise qui a une dette en devise étrangère, si la devise nationale baisse, la dette va augmenter.
		
		\subsubsection{Dépréciation des actifs}
		
		La dépréciation de la valeur des actifs financiers enregistrés au bilan des institutions financières peut provoquer une dégradation des bilans et une contraction du crédit.
		
	\subsection{La détérioration des bilans des institutions 
financières}

	Si les institutions financières souffrent d'une détérioration de leurs bilans, elles subissent en conséquence une contraction de leur capital: les ressources bancaires seront moindres et le crédit bancaire diminuera.
	
	La contraction du crédit conduit alors à une baisse de la dépense d'investissement qui ralentit l'activité économique.
	
	\subsection{Les défaillances bancaires}

	\begin{itemize}
		\item Effet de contagion 
		\item Panique bancaire 
		\item La faillite d'un grand nombre de banques sur une brève période de temps signale que la production d'information sur les marchés financiers est dégradée: par conséquent, l'intermédiation financière par le secteur bancaire risque de s'interrompre 
	\end{itemize}
	
	La chute du crédit diminue l'offre de fonds disponibles pour les 
emprunteurs. Il en résulte une baisse des prêts pour financer les investissements productifs et une contraction encore plus grave de l'activité 

	\subsection{La montée de l'incertitude}
	
	Une forte hausse de l'incertitude sur un marché financier accentue la difficulté pour les prêteurs de distinguer les bons et les mauvais risques de crédit. Par exemple, les éclatements de bulles (immobilières) font chuter les prix et augmenter l'incertitude.
	
	Conséquence : baisse du crédit, de l'investissement, détérioration de l'activité économique.
	
	\subsection{La hausse des taux d'intérêt}
	
Augmentation des charges de dettes pour les entreprises. Il y a dès lors une diminution de l'autofinancement et un risque de diminution des investissements et une contraction de l'activité économique.


	Une augmentation imprévue du taux d'intérêt peut faire éclater une bulle spéculative.

	\subsection{Le déséquilibre budgétaire de l'état}
	
	\begin{itemize}
		\item Le déséquilibre budgétaire peut faire craindre un défaut de l'état sur le paiement de sa dette 
		\item L'état peut rencontrer des difficultés à placer les titres de la dette publique auprès des investisseurs. 
		\item Les bilans des institutions financières peuvent se détériorer. 
		\item Diminution de l'octroi de crédit, contraction de l'activité économique
		\item Crise de change
	\end{itemize}


\section{La dynamique des crises financières}

	\dessinS{schema_crise}{.3}
	
	\subsection{Phase 1 : déclenchement d'une crise financière}
	
	Une mauvaise maîtrise du processus de libéralisation/innovation financière 
\begin{itemize}
	\item Emballement du crédit : les banques ont du s'endetter sur le marché à travers la titrisation, et donc l'augmentation du risque.
	\item Levier financier 
	\item Mauvaise gestion des risques 
	\item Contraction du crédit 
\end{itemize}

	On libéralise un marché pour faciliter la circulation des capitaux. Maintenant, on tente de régulier le marché afin de diminuer les risques.

	\subsubsection{La bulle et la chute du prix des actifs}
	
	Bulle spéculative : prix déconnecté de la réalité et des déterminants économiques.
	
	\begin{itemize}
		\item Prix déconnecté de leurs déterminants économiques fondamentaux 
		\item Exubérance irrationnelle 
		\item Emballement du crédit 
		\item Quand la bulle éclate, 
		\begin{itemize}
			\item les emprunteurs vont décroître leur capacité d'emprunter, contraction du crédit et de la dépense 
			\item Détérioration du bilan des institutions financières, diminution du levier
		\end{itemize}
	\end{itemize}
	
	\subsubsection{Les pics de taux d'intérêt}
	
	la hausse des taux d'intérêt fait baisser les flux de revenus des 
ménages et des entreprises, réduit le nombre de bons risques en 
recherche d'emprunts, ce qui augmente l'anti-sélection et le risque 
moral et fait baisser l'activité économique

	\subsubsection{Montée de l'incertitude}
	Souvent après le début d'une récession, soit après un krach boursier. Un trait commun des crises financières est la défaillance des institutions financières
	

	\subsection{Phase 2 : crise bancaire}
	
	Avec la détérioration de l'activité économique, l'augmentation des 
défaillances d'entreprises et de ménages et l'incertitude sur la solidité des banques, les déposants commencent à retirer leurs fonds des banques ( crise ou panique bancaire).

	La diminution du nombre de banques en activité fait perdre leurs ressources informationnelles, l'anti-sélection et le risque moral s'aggravent sur les marchés du crédit : spirale descendante


	Moyens de surmonter la crise financière :
	\begin{itemize}
		\item Régulation des marchés
		\item Recapitalisation des banques pour réduire l'incertitude sur les marchés
	\end{itemize}

	\subsection{Phase 3 : déflation par la dette}
	
	Quand la déflation par la dette s'établit, les problèmes d'anti-sélection et de risque moral s'aggravent davantage, ce qui provoque une dépression de longue durée des prêts, de la dépense d'investissement et de l'activité économique globale. 


\section{La crise financière de 2007-2010}

	\subsection{La détérioration des prêts au logement}
	
	L'élément déclencheur est la baisse du taux d'intérêt directeur, qui a fait décoller les marchés, car on pouvait s'endetter très fortement.
	
	GSE : agences bénéficiant de la garantie de l'état. Elles ont racheté les crédits hypothécaire accordées par les banques et les ont titrisées. Ces titres ont été mis sur le marché avec la garantie de l'état américain. Le but était de réduire les risques au niveau des banques et transférer les risques au niveau des acteurs de la bourse. Elles ont créé un marché attractif.
	
	RMBS : filiales des GSE, institutions qui ont comme compte propre des actifs titrisés.
	
	\subsection{L'assouplissement des critères dans l'attribution des prêts}
	
	A partir de 2000, les prêteurs ont assoupli les critères liés à l'attribution des crédit immobiliers, ce qui a entraîné les prix à la hausse. Ainsi, l'exigence de capital passait de 4 à 2\% lorsque les titres hypothécaires sont titrisés, car le risque sur le fond propre est moindre.

	Cette double augmentation, crédits attribués et prix des maisons, était très intéressante à la fois pour les vendeurs de prêts (ou brokers) et les prêteurs. 

	Afin de soutenir le marché, les intermédiaires ont rendu les critères 
d'attribution des crédits de moins en moins contraignants. On en est arrivé à des prêts immobiliers à taux ajustable.

	Depuis 1990, le gouvernement souhaitait promouvoir l'accès à la propriété en incitant les prêteurs à attribuer davantage de crédits aux familles à revenus modestes.

	Aux Etats-Unis, toute banque était obligée de donner des crédits hypothécaires, même à des personnes insolvables, afin de faciliter l'accès à la propriété. La titrisation permettait d'exporter ces risques.
	
	\subsection{La titrisation}
	
	Les banques qui accordent des prêts les financent traditionnellement avec des dépôts de clients. 

	La titrisation consiste à regrouper des portefeuilles de prêts et à vendre les promesses de cash-flows associées (intérêt et principal) sous forme d'actifs financiers. Elle permet aussi d'accroître les volumes de prêts consentis plus rapidement que ne le permettrait l'évolution des dépôts. 
	
	\dessinS{14}{.5}
	
	Les banques se débarrassent donc des prêts au SPV (Special Purpose Vehicule), un véhicule spécial de titrisation. Ce dernier va alors émettre sur le marché différents types de titre à destination des investisseurs (des obligations par exemple). Ces différents types d'obligation sont les tranches senior, mezzanine et equity, au sein d'un ABS (Asset-Backed Security).
	
	Le cash-flow des actifs est donc réparti selon un principe de cascade à travers les différentes tranches. On rembourse d'abord les actifs et les intérêts de la tranche senior, puis de la tranche mezzanine et enfin la tranche equity. 
	
	La tranche equity est plus risquée mais rapporte un meilleur rendement. La tranche mezzanine peut être de nouveau titrisée, et donc de nouveau avoir une tranche senior, mezzanine et equity.
	
	\dessinS{15}{.5}
	
	Equity est la première tranche à supporter les pertes sur le portefeuille. On remonte ensuite dans la cascade pour rembourser les pertes. On remonte ensuite dans la tranche equity  de l'ABS (ou la tranche mezzanine) et ainsi de suite.
		
	
	AAA signifie que le risque d'insolvabilité est très faible. On peut remarquer que la tranche senior supporte des pertes à partir de 10\% de pertes sur le portefeuille.
	
	\dessinS{16}{.35}
	
	Les banques achètent ces titres entre elles, ainsi que des compagnies d'assurance, des fonds d'investissement, etc.
	
	Ainsi, les SPE, SIV et VIE sont des véhicules de titrisation. Les titres titrisés AAA ont besoin de moins de fonds propres.
		
	\dessinS{17}{.55}
	
	Le collateral est le fait de mettre des titres en garantie lors de l'octroi de prêts. CDS signifie Credit Depot Swap.
	
	\subsection{L'éclatement de la bulle immobilière}
	
	Il y a eu une augmentation des défaillances sur les prêts hypothécaires, ce qui a entraîné des saisies immobilières et une dépréciation des actifs financiers liés aux crédits immobiliers.
	
	Le marché interbancaire des emprunts de liquidités contre titres (repurchase agreements) et celui des billets de trésorerie adossés à des actifs (Asset- backed commercial papers) se sont grippés étant donné l'incertitude sur la valeur des titres utilisés comme collatéral. 

	Il y a eu de fortes tensions sur les taux d'intérêt et au final les banques et autres institutions financières ont brutalement cessé de se prêter les unes aux autres.
	
	\subsection{Les défaillances et la panique financière de 2008}
	
	Il y a eu une détérioration des bilans bancaires. Pour se refinancer, les SIV émettent du papier commercial ABCP à court terme. Cependant, les SIV et les conduits ont été incapables de continuer à placer leur commercial paper.

	Du coup, l'impossibilité de se refinancer a tari le paiement des coupons. Le marché des ABCP et des repruchase agreements sont les premières sources de financement du système bancaire : assèchement dès le premier semestre 2007.
	
	
	Si on récapitule les voies par lesquelles les bilans des banques et institutions financières ont pu être exposés à la dépréciation des actifs titrisés, on trouve 5 principaux canaux :

	\begin{enumerate}
		\item La banque a produit et vendu des titres adossés à des créances et elle a gardé pour compte propre des tranches superseniors notées AAA afin de structurer le total de ses actifs pondérés des risques (RWA) et de minimiser sa charge en capital réglementaire.
		\item La banque a conservé un lien avec les investisseurs auxquels elle a vendu des titres structurés à travers un véhicule de titrisation. Le véhicule de titrisation ne parvient pas à isoler complètement l'originator de la faillite des investisseurs. 
		\item La banque a acheté des produits structurés à d’autres institutions 
financières à des fins de négoce (trading) ou d’investissement soit pour compte de la clientèle soit pour compte propre. La banque ne sait plus à quel prix elle doit valoriser les titres dépréciés dans ses comptes; ses contreparties peuvent refuser ces titres en collatéral des emprunts de la banque (rep's, ABCP) 
		\item La banque a émis un CDS sur un titre complexe ou elle est partenaire d'un pool de CDS. Si le CDS est activé par un événement de crédit, la banque doit fournir du collatéral supplémentaire ou de meilleure qualité et/ou payer en espèces le bénéficiaire de la protection. 
		\item La banque a pris des positions courtes sur un produit structuré, c'est-à-dire qu'elle l'a vendu à découvert: elle emprunte le titre qu'elle doit livrer à terme en espérant le racheter moins cher.
	\end{enumerate}

	\chapter{La banque et la gestion des institutions financières}

	\section{Le bilan bancaire}
	
\begin{center}
	\begin{tabular}{|c|c|}
	\hline 
	Actif & Passif \\ 
	\hline 
	Caisse, banques centrales & Caisse, banques centrales \\ 
	\hline 
	Prêts aux établissements de crédit & Emprunts auprès des établissements de crédit \\ 
	\hline 
	Crédits à la clientèle & Ressources émanant de la clientèle \\ 
	\hline 
	Opérations sur titres & Opérations sur titres \\ 
	\hline 
	Valeurs immobilisées & Provisions, capitaux propres \\ 
	\hline 
	Divers & Divers \\ 
	\hline 
	\end{tabular} 
\end{center}
	
	Passif :
	
	\begin{itemize}
		\item Opérations sur titres : titres émis sur le marché des capitaux à court terme
		\item Provisions en cas de dépréciation d'actifs. 
		
			Capitaux propres : capital de la banque + ses bénéfices non distribués + toutes les dettes subordonnées (c'est-à-dire que la dette ne sera remboursée qu'après les autres dettes)
	\end{itemize}
	
	Actif :
	
	\begin{itemize}
		\item Caisse, banques centrales : liquidité en dépôt auprès de la banque centrale
		\item Prêts aux établissements de crédit : argent de la banque en dépôt auprès d'autres banques/établissement de crédits
		\item Opérations sur titres : titres détenus par la banque soit pour son propre compte soit pour compte de la clientèle
		\item Valeurs immobilisées : immobilisations corporelles et incorporelles de la banque
		Prêts subordonnées : participation dans des entreprises liées
	\end{itemize}
	
	\section{L'exploitation bancaire}
	
	Les mécanismes de base d'une fonctionnement d'une banque :
	
	\begin{itemize}
		\item Transformation d'actifs : transformation des dépôts de sa clientèle pour accorder des prêts ; elle a un rôle d'intermédiation
		\item Transformation d'échéances : par exemple emprunter à court terme sur les marchés pour prêter à long terme
		\item La banque vend des services qu'elle facture à ses clients : par exemple traitement des virements, relevés bancaires, conseils en placement
	\end{itemize}
	
	Un dépôt dans une banque entraine une réserve supplémentaire.
	
\begin{center}
	\begin{tabular}{|c|c|c|c|}
	\hline 
	\multicolumn{2}{|c|}{Actif} & \multicolumn{2}{|c|}{Passif} \\ 
	\hline 
	Réserves & +100 & Dépôts & +100 \\ 
	\hline 
	\end{tabular} 
\end{center}

	Comment les banques réagissent à une hausse du montant des dépôts.
	
	\begin{center}
	\begin{tabular}{|c|c|c|c|}
	\hline 
	\multicolumn{2}{|c|}{Actif} & \multicolumn{2}{|c|}{Passif} \\ 
	\hline 
	Réserves obligatoires & +10 & Dépôts & +100 \\ 
	\hline 
	Réserves & +90 & & \\ 
	\hline 
	\end{tabular} 
\end{center}
	
	\section{Principes de gestion de bilan}
	
	\subsection{Gestion de liquidité}
	
	Une banque doit s'assurer qu'elle dispose d'assez de réserve/de liquidité pour rembourser les déposant qui retirent de l'argent de leur compte. Par exemple, avec un taux de 10\%, les dépôts des clients doivent être assurés à hauteur de 10\% par les réserves.
	
	Par exemple, supposons un taux de réserves obligatoires de 10\%. Avant un retrait :
	
\begin{center}
	\begin{tabular}{|c|c|c|c|}
	\hline 
	\multicolumn{2}{|c|}{Actif} & \multicolumn{2}{|c|}{Passif} \\ 
	\hline 
	Réserves & 20 & Dépôts & 100 \\ 
	\hline 
	Prêts & 80 & Fonds propres & 10 \\ 
	\hline 
	Titres & 10 &  &  \\ 
	\hline 
	\end{tabular} 
\end{center}
	
	Après un retrait de dépôts de 10.
	
\begin{center}
		\begin{tabular}{|c|c|c|c|}
	\hline 
	\multicolumn{2}{|c|}{Actif} & \multicolumn{2}{|c|}{Passif} \\ 
	\hline 
	Réserves & 10 & Dépôts & 90 \\ 
	\hline 
	Prêts & 80 & Fonds propres & 10 \\ 
	\hline 
	Titres & 10 &  &  \\ 
	\hline 
	\end{tabular}
	\end{center}	
	
	Afin de satisfaire le taux de réserves obligatoires, une banque a 4 options :
	
	\begin{itemize}
		\item emprunter auprès d'autres banques ou institutions financières
		
\begin{center}
\begin{tabular}{|c|c|c|c|}
\hline 
\multicolumn{2}{|c|}{Actif} & \multicolumn{2}{|c|}{Passif} \\
\hline 
Réserves & 9 & Dépôts & 90 \\ 
\hline 
Prêts & 90 & Emprunts/I.F. & 9 \\ 
\hline 
Titres & 10 & Fonds propres & 10 \\ 
\hline 
\end{tabular} 
\end{center}		
				
		\item vendre une partie de ses titres
		
		\begin{center}
\begin{tabular}{|c|c|c|c|}
\hline 
\multicolumn{2}{|c|}{Actif} & \multicolumn{2}{|c|}{Passif} \\
\hline 
Réserves & 9 & Dépôts & 90 \\ 
\hline 
Prêts & 90 &  . &   \\ 
\hline 
Titres & 1 & Fonds propres & 10 \\ 
\hline 
\end{tabular} 
\end{center}

		\item emprunter les liquidités auprès de la banque centrale
		
\begin{center}
\begin{tabular}{|c|c|c|c|}
\hline 
\multicolumn{2}{|c|}{Actif} & \multicolumn{2}{|c|}{Passif} \\
\hline 
Réserves & 9 & Dépôts & 90 \\ 
\hline 
Prêts & 90 & Emprunts/B.C. & 9 \\ 
\hline 
Titres & 10 & Fonds propres & 10 \\ 
\hline 
\end{tabular} 
\end{center}		
		
		\item Réduire le montant de ses prêts : ne pas renouveler des prêts à court terme arrivant à échéance, au risque de perdre des clients, vendre des prêts 
à d'autres banques 

\begin{center}
\begin{tabular}{|c|c|c|c|}
\hline 
\multicolumn{2}{|c|}{Actif} & \multicolumn{2}{|c|}{Passif} \\
\hline 
Réserves & 9 & Dépôts & 90 \\ 
\hline 
Prêts & 81 &  &   \\ 
\hline 
Titres & 10 & Fonds propres & 10 \\ 
\hline 
\end{tabular} 
\end{center}
	\end{itemize}
	
	Conclusion: justification de la détention de 
réserves excédentaires, qui permet: 
	\begin{itemize}
		\item d'éviter d'emprunter auprès d'autres banques 
		\item d'éviter de vendre des titres 
		\item d'éviter d'emprunter auprès de la banque centrale 
		\item d'éviter de résilier ou de vendre des titres 
	\end{itemize}
	
	NB : les banques peuvent aussi détenir plus de titres liquides (réserves secondaires) 
	
	\subsection{Gestion d'actif}
	
	Elle doit garder un niveau de risque faible et avoir des actifs suffisamment diversifiés et rémunérateur.
	
	Il y a 3 objectifs :
	
	\begin{enumerate}
		\item chercher des rendements les plus élevés possibles sur les prêts et titres 
		\item réduire les risques 
		\item préserver une liquidité suffisante
	\end{enumerate}
	
	4 moyens :
	
	\begin{itemize}
		\item Trouver des emprunteurs qui paieront des taux élevés, et peu susceptibles de faire défaut : examen sélectif pour réduire les points de base (centième de pourcent) d'antisélection ; on sélectionne les bons clients de manière à réduire au maximum le risque de défaut.
		\item Acheter des titres à rendement élevé et risque faible
		\item Diversification des risques : en achetant différents types d'actifs ( maturité, émetteur, \dots), éviter de trop se spécialiser sur un secteur (immobilier, énergie, \dots) 
		\item Gérer la liquidité : décider du montant des réserves excédentaires, des titres émis par l'état ( réserves secondaires). Il y a un équilibre à trouver entre avoir des liquidités et un rendement
	\end{itemize}
	
	\subsection{Gestion de passif}
	
	Acquérir des fonds à un faible coût.
	
	Avant 1960, la gestion de passif n'était pas développée : la plus grande partie des ressources étaient constituées de dépôts à vue, non rémunérés. Le marché interbancaire était peu développé 

	A partir des années 60 aux USA, les grandes banques utilisent davantage les marchés financiers, développent de nouveaux instruments (certificats de dépôts négociables). Il y a une nouvelle flexibilité dans la gestion du passif, recherche de fonds au fur et à mesure des besoins liés à la croissance de l'actif, au- delà du montant des dépôts. Les banques gèrent les 2 côtés du bilan en même 
temps, dans des comités de gestion actif-passif 
(ALM) 

Des changements importants dans la composition 
des bilans bancaires depuis 30 ans: quelques 
exemples: 
\begin{itemize}
	\item Certificats de dépôts négociables et emprunts 
interbancaires: de 2\% (1960) à 47\% ( 2008): USA 
	\item Prêts: 46\% des actifs bancaires à 61\%: USA 
	\item Part des obligations émises par les banques : 6\% à 
18\%: France 
\end{itemize}

	\subsection{Adéquation du capital}
	
	Gérer le montant des fonds propres à détenir en adéquation avec les accords de Basel, qui impose un montant minimum de fonds propres.
	
	 La faillite est l'impossibilité de remplir les obligations de remboursement envers les déposants et autres créanciers. Une banque détient du capital pour réduire sa probabilité de devenir 
insolvable.
	
	Raisons d'avoir des capitaux propres :
	
	\begin{itemize}
	
		\item éviter la faillite. Exemple de deux banques (B pas assez capitalisé), qui se rendent compte que 5 ne valent rien.
		
		\dessinS{6}{.5}

		\item l'effet du capital sur le rendement des actionnaires. On peut définir
		
		\begin{itemize}
			\item le coefficient de rendement = return on assets (RAO) = profit net après impôts/actifs 
			\item le coefficient de rentabilité = return on equity = profit net après impôts/fonds 
propres 
			\item EM = multiplicateur de fonds propres= Actifs/fonds propres 
		\end{itemize}
		
		$$ROE = ROA \times EM $$
		
		Pour un ROA donné, moins la banque est capitalisée ( plus EM petit) et plus la rentabilité du capital est élevée (ROE élevé). Par exemple, le rendement pour les actionnaires de la banque B est meilleure que pour la banque A car B est sous-capitalisée.
	\end{itemize}
	
	Il y a un arbitrage des actionnaires entre la sécurité et rentabilité . Les avantages et inconvénients du capital bancaire :
	\begin{itemize}
		\item[+] il protège de la probabilité de faillite 
		\item[-] il diminue la rentabilité ( à ROA donné) 
	\end{itemize}
		
	 Il y a également des exigences en capital réglementaire.
	
	
	\subsection{La gestion du risque de crédit}
	
	\begin{enumerate}	
		\item Sélection et surveillance 
		\item Relation de clientèle à long terme 
		\item Engagements de financement 
		\item Collatéral et dépôt de garantie 
		\item Rationnement du crédit
	\end{enumerate}
	

\section{Gestion du risque de taux d'intérêt}

\dessin{7}
Si les taux augmentent en moyenne de 5 points ( de 10 à 15\%): 
\begin{itemize}
	\item Les revenus d'actifs augmentent de 20 x 5\% = 1M\euro
	\item Les charges d'intérêts sur dettes augmentent de 50 x 5\% = 2,5 M\euro
	\item $\rightarrow$ le profit de la banque diminue de 1,5 M\euro 
\end{itemize}


Si les taux d'intérêts diminuent en moyenne de 5 points, le profit de la banque augmente de 1,5 M.

Si une banque possède plus de dettes que d'actifs sensibles aux taux, une hausse du taux d'intérêt réduit son profit, une baisse des taux l'augmente.


Méthode des impasses comptables  : le montant des dettes sensibles aux taux d'intérêt est soustrait du montant des actifs sensibles aux taux = actifs sensibles nets. Ainsi,

\begin{itemize}
	\item Impasse ou gap = -30M\euro
	\item Impasse (-30) x variation du taux ( 5\%) = variation du profit (-1,5) 
\end{itemize}

Analyse de duration : la duration moyenne des actifs de la banque est de 3 ans (la durée de vie moyenne des revenus est de 3 ans) tandis que la duration moyenne des dettes est de 2 ans.
	
Quelle stratégie pour gérer le risque de taux ? 
\begin{itemize}
	\item Si vous anticiper une baisse des taux d’intérêt : ne rien faire pour bénéficier de la baisse attendue 
	\item Diminuer la duration des actifs, augmenter celle des dettes…mais cela peut être difficiel à court terme (reflet de la spécialisation de la banque) 
	\item Utiliser des produits dérivés pour réduire l'exposition au taux sans modifier la structure du bilan
\end{itemize}

\section{Activités hors-bilan}

Les éléments hors-bilan sont composés d'un ensemble de comptes retraçant des engagements qui ne donnent pas lieu à des flux de trésorerie immédiats. Par exemple, un engagement de financement à l'égard de la clientèle, de garantie ou de titre.
	% Conférence 25/03/2013

\section{Accord de Basel}

\subsection{Basel I}
Basel I : nécessité d'un minimum de fonds propres, afin de pouvoir supporter un retrait massif d'argent par des clients qui soldent leur compte. Le ratio (cook) est d'environ 8\%. C'est un accord très général et standardisé, il touche autant des petites banques que des géants.

Le problème de Basel I est le critère de solvabilité (trop compliqué pour des petites banques, trop simples pour des plus grosses). De plus, il manque la gestion du risque opérationel (risque d'une panne du service informatique par exemple).

\subsection{Basel II}

Dans Basel II, Basel I est réutilisé pour les banques dans des pays moins développés car il est suffisant.

Pondération sur trois niveaux :

\begin{itemize}
	\item basic : calcul semblable à Basel I, mais plus précis.
	
	Grosse erreur de l'époque : utilisation de la notation d'un pays (donné par une agence de notation) pour mesurer son risque.
	
	\item foundation : idem mais gestion d'un risque personnel.
	\item advanced : les banques gèrent comme elles veulent les risques, mais ne peuvent pas faire n'importe quoi (le modèle doit être approuvé par un régulateur).
\end{itemize}

Disclosure : information qui doit être légalement publiée. Cela oblige la banque à prévoir des marchés qui dévient ou des situations critiques.

Basel II n'est qu'un agreement, pas un accord : des pays l'ont signé mais cela n'a pas été appliqué partout (par exemple aux Etats-Unis).

Problème de timing, l'accord est arrivé en Europe au début de la crise financière, elle aurait été minimisée.

Autre problème : le critère de solvabilité était désuet : des banques avec un bon seuil de solvabilité ont fait faillite par manque de liquidités.

\subsection{Basel III}

Transition entre Basel II et III initiée par la crise, beaucoup plus rapide qu'entre Basel I et II.


	% 15/4/2013
\chapter{Offre de monnaie}

\section{La finance intermédiée}

	\subsection{Le rôle des intermédiaires financiers}
	
	Intermédiaire financier : institution servant d'interface entre des emprunteurs et des épargnants. Il y a des intermédiaires financiers bancaires et non bancaires (ex : compagnie d'assurance).
	
	Le besoin d'intermédiaires financiers est né de l'imperfection des marchés financiers :
	
	\begin{itemize}
		\item L'information est asymétrique et incomplète pour les acteurs, les intermédiaires prennent le risque qui devrait être pris par la personne qui prête de l'argent.
	
		\item Il peut y avoir une inadéquation entre les besoins du prêteur et ceux du demandeur. Les intermédiaires s'arrangent pour que les besoins soient satisfais des deux côtés
\end{itemize}
	Le risque est à la charge des banques et intermédiaires financiers. Vu leur taille, ils ont la possibilité de diversifier les risques.
	
	
	\subsection{Le rôle de la banque}
	
	Un intermédiaire financier bancaire peut créer de la monnaie (avec un agrément auprès des institutions de régulation bancaires). Un intermédiaire non bancaire ne peut pas recevoir des dépôts bancaires pour pouvoir créer de la monnaie.
	
	La banque va créer de la monnaie en octroyant des crédits aux emprunteurs. Elle va également sélectionner des projets d'investissements. Pour faire cette sélection, elle va regarder
	
	\begin{itemize}
		\item les facteurs externes de rentabilité (conjonctions macroéconomiques)
		\item facteurs liés  au projet et à l'entreprise. Pour cela, la banque dispose d'informations privatives sur l'emprunteur, elles ne sont pas publiques.
	\end{itemize}
	
	L'actif bancaire (prêt octroyé aux ménages et aux entreprises) est très difficilement cessible sur les marchés financiers, il est illiquide car il porte sur le long terme et par le secret bancaire on ne connaît pas le risque associé. La titrisation permet de rendre cet actif plus liquide car en agrégeant beaucoup d'actifs on répartit et diversifie le risque. On peut avoir des problèmes si le modèle statistique utilisé est erroné et qu'au final le risque soit plus grand que ce qui est attendu.
	
	[note manuscrites]
	
\section{La création monétaire}

Il y a trois sources à la création de monnaie, appelées les contreparties de la masse monétaire :

\begin{itemize}
	\item contrepartie extérieure de la masse monétaire. Elle est issue des opérations de change de la banque centrale et du solde de la balance courante.
	
	\item contrepartie "créances nettes" de l'état : les banques achètent de la dette publique. La banque centrale européenne ne peut acheter de la dette des banques de l'union européenne.
	\item contrepartie "créance nette" sur l'économie (l'essentiel de la création monétaire) : ensemble des crédits octroyés aux ménages et aux entreprises.
\end{itemize}

On a l'offre de monnaie $M$ = billet et pièces $C$ + objets à vue bancaire $D$.

	\subsection{Un système bancaire avec 100\% de réserves}
	
	Supposons une économie sans manque, où $D = 0$, donc $M = C$, la masse monétaire n'est constituée que des billets et des pièces. Par exemple, prenons $M = C = 1000$ euros.
	
	Introduisons les banques mais celles-ci n'acceptent que les dépôts et n'octroient aucun crédit. Si les 1000 euros sont déposés à la banque, $M = D = 1000$ euros.
	
	Bilan de la banque :
	
	\begin{tabular}{c|c}
	Actifs & Passif \\ 
	\hline 
	Réserves, 1000 & Dépôts, 1000
	\end{tabular} 
	
	Avec 100\% de réserve, il n'y a pas de création monétaire.
	
	\subsection{Un système bancaire avec des réserves partielles}
	
	\subsubsection{Avec une seule banque}
	
	Les banques peuvent octroyer des crédits. Supposons que la banque octroie un crédit de 800 euros.
	
	\begin{tabular}{c|c}
	Actif & Passif \\ 
	\hline 
	Réserve, 200 & Dépots, 1000 \\ 
	Crédit, 800 & 
	\end{tabular} 
	
	Masse monétaire : $M = C + D = 800+  1000 = 1800$. On a donc une création monétaire de $1800 - 1000 = 800$.
	
	\begin{itemize}
		\item supposons que les 800 de liquidité sont déposés à la banque. Cette dernière n'octroie pas de crédit.
		
		\begin{tabular}{c|c}
		Actif & Dépot \\ 
		\hline 
		Réserves, 1000 & Dépots, 1800 \\ 
		Crédit, 800 &  
		\end{tabular} 
		
		La masse monétaire est de 1800 sous forme de dépôt ; on n'a pas de création monétaire, seule la composition de la masse monétaire a changé.
		
		\item Si un individu retire 200 euros, le dépôt ne s'élève plus qu'à 1600 et les réserves sont à 200. (??)
		
		\item supposons que l'emprunteur rembourse son prêt.
		
		\begin{tabular}{c|c}
		Actif & Passif \\ 
		\hline 
		Réserves, 1000 & Dépôt, 1000 \\ 
		Crédit, 0 & 
		\end{tabular}
		
		La masse monétaire est de 1000 euros ; chaque fois qu'un crédit est remboursé on a une destruction monétaire (ici de 800 euros).
	\end{itemize}
	
	La création monétaire augmente si le nombre de crédits ouverts est plus grand que le nombre de crédits remboursés.
	
	
	\subsubsection{Avec plusieurs banques}
	
	Supposons deux banques, A et B. La masse monétaire en circulation est égale à 1000 euros, déposés à la banque A. Celle-ci a un taux de réserve sur dépôt de 20\%.
	
	\begin{tabular}{c|c}
	Actif (A) & Passif (A)\\ 
	\hline 
	Réserve, 200 & Dépôts, 1000 (D1) \\ 
	Crédits, 800 &  
	\end{tabular} 
	
	Création monétaire : 1800 - 1000 = 800 euros. La masse monétaire est de 800 + 1000 = 1800.
	
	L'emprunteur dépose les 800 euros à la banque B, qui a un taux de réserve de 20\%.
	
	\begin{tabular}{c|c}
	Actif (B) & Passif (B) \\ 
	\hline 
	Réserve, 160 & Dépôts, 800 (D2) \\ 
	Crédit, 640 &  
	\end{tabular} 
	
	$M = 1800 + 640 = 2440$. La création monétaire est de 680 euros.
	
	Relations possibles entre A et B. Supposons qu'un client de la banque A retire 100 euros.
	
	\begin{tabular}{c|c}
	Actif (A) & Passif (A) \\ 
	\hline 
	Réserves, 100 & Dépôts, 900 \\ 
	Crédit, 800 &  
	\end{tabular} 
	
	Le taux de réserve est inférieur à 20\%, il y a un manque de liquidité. La banque A s'adresse à la banque B sur le marché interbancaire et emprunte 100 euros. Des opérations interbancaires (OI) s'insèrent dans le bilan (au taux EONIA) :
	
	\begin{tabular}{c|c}
	Actif (A) & Passif (A) \\ 
	\hline 
	Réserves, 200 & Dépôts, 900 \\ 
	Crédit, 800 &   OI, 100
	\end{tabular}
	
	\begin{tabular}{c|c}
	Actif (B) & Passif (B) \\ 
	\hline 
	Réserves, 60 & Dépôts, 800 \\ 
	Crédit, 640 &  \\
	OI, 100 &
	\end{tabular}
	
	Si la banque B refuse ou ne peut pas prêter le crédit, la banque A va s'adresser à la banque centrale. Elle ne s'y adresse pas directement, elle va d'abord voir chez les autres banques, car les taux d'intérêts sont plus élevés. Cet emprunt à la banque centrale appelé une opération de refinancement.
	
	\begin{tabular}{c|c}
	Actif (A) & Passif (A) \\ 
	\hline 
	Réserves, 200 & Dépôts, 900 \\ 
	Crédit, 800 &   Refinancement, 100
	\end{tabular}
	
		
	\begin{tabular}{c|c}
	Actif (B) & Passif (B) \\ 
	\hline 
	Réserves, 160 & Dépôts, 800 \\ 
	Crédit, 640 &
	\end{tabular}
	
	
	\begin{tabular}{c|c}
	Actif (C) & Passif (C) \\ 
	\hline 
	X, 500 & Billets, 240 \\ 
	Refinancement, 100 & Réserves, 360 \\
	\end{tabular}
	
	Les réserves de la BC sont l'addition des réserves de A et B, les dépôts des banques à la BC.
	
	Supposons que la BC émette davantage de billets en circulation.
	
	\begin{tabular}{c|c}
	Actif (C) & Passif (C) \\ 
	\hline 
	X, + 100 & Billets, +100 
	\end{tabular}
	
	Une émission de billets entraîne une dévaluation de la monnaie et une réévaluation des monnaies étrangère : le stock n'a pas changé mais bien sa valeur.
	
	Si l'économie connaît un déficit de la balance courante (c'est-à-dire importation $>$ exportation), il y a un achat de devises auprès de la BC.
	
	\begin{tabular}{c|c}
	Actif (C) & Passif (C) \\ 
	\hline 
	X, -100 & Billets, -100
	\end{tabular}
	
	Tant que les importations sont $>$ exportations, il y a une diminution des réserves de devises.
	\chapter{La politique monétaire}

Autrefois, les Etats battaient la monnaie et les banques centrales dépendaient directement du pouvoir politique. Les banques centrales utilisaient des instruments autoritaires pour contrôler l'offre de monnaie : il y avait un encadrement autoritaire du crédit.

Aujourd'hui, les banques centrales sont indépendantes du pouvoir politique. La politique monétaire prend la forme d'une régulation monétaire où les instruments utilisés par les banques centrales exercent une contrainte de liquidité sur le principal levier de la création monétaire : le crédit bancaire.

\section{Les canaux de transmission de la politique monétaire}

	\subsection{Les canaux de transmission réels}

Une modification de la politique monétaire de la banque centrale a des effets sur le court terme mais pas sur le long terme, même s'il y a des effets sur les prix.  Les variations du taux d'intérêt modifie toute une série de comportements économiques :

\begin{itemize}
	\item Consommation et épargne
	\item L'investissement des entreprises et l'investissement des ménages (biens immobiliers)
	\item La compétitivité des entreprises par rapport à celle des entreprises étrangères, car si le taux augmente, la marge des entreprises va diminuer
	\item La dette publique
	\item Le secteur bancaire qui voit ses profits baisser, car il ne peut pas changer facilement son taux directeur ; si le taux d'intérêt augmente la marge va aussi diminuer
\end{itemize}

	\subsection{Les canaux de transmission monétaires}
	
	A moyen et long terme, l'offre de monnaie a un effet sur le taux d'inflation : c'est la théorie quantitative de la monnaie.	A court terme, on peut considérer qu'il existe une certaine rigidité des prix et notamment une rigidité des salaires nominaux. La politique monétaire, en manipulant le taux d'intérêt, peut affecter la production et l'emploi si les prix ne s'ajustent pas instantanément (théorie keynésienne). L'influence de la politique monétaire sur les salaires vient de leur rigidité. Si l'impact était immédiat, il n'y aurait pas d'effets.	La demande de monnaie dépend aussi du niveau des taux d'intérêt.
	
\section{Neutralité de la monnaie}

C'est un débat essentiel en économie monétaire. D'un point de vue pragmatique, toutes les banques centrales considèrent que la politique monétaire a des effets réels à court terme. Si on manipule trop la monnaie, il y aura un effet sur les prix.

\section{Objectifs de la politique monétaire}

\begin{itemize}
	\item La stabilité des prix
	\item La croissance économique et le plein emploi
	\item L'équilibre extérieur (équilibre des comptes de la balance des paiements). Si un pays importe plus qu'il n'exporte, la balance courante en déficit, il a une dette envers les autres pays. Pour contrebalancer, il va demander un crédit. Si les autres pays refusent de faire crédit, il y a une crise de la balance courante. C'est ce qu'il se passe actuellement entre les USA et la Chine, qui achète les bons du trésor américains tandis que les américains achètent des produits chinois. Lorsqu'un pays est en cessation de payement (c'est-à-dire qu'il ne peut plus rien importer), il s'adresse au FMI.
	\item La stabilité du système financier : il y a injection de liquidités pour que les investisseurs ne revendent pas leur parts.
\end{itemize}

Ces objectifs sont souhaitables mais difficilement compatibles et atteignables ensemble et en même temps. Il faut dès lors faire un choix :

\begin{itemize}
	\item Dans les années 60, l'objectif était plutôt la croissance économique. La politique monétaire était donc plus laxiste par rapport à l'objectif de stabilité des prix. Peu à peu, les taux d'inflation ont augmenté pour passer à deux chiffres à la fin des années 70 lorsque l'inflation devient grande (5-7\%), déclenchement du mécanisme des [...].
	\item Au début des années 80, c'est l'objectif de stabilité des prix qui est devenu prioritaire pour un grand nombre de banques centrales.
\end{itemize}

\section{Objectifs de la BCE}

Son objectif (stipulé dans le traité de Maastricht) est la stabilité des prix, qui est défini comme un taux d'inflation annuel inférieur à 2\% et proche de la valeur de 2\%. Cette valeur est choisie pour qu'on n'atteigne jamais la déflation, ce qui pourrait être le cas avec 1\%.

\section{Les outils de la politique monétaire}

La BC n'a pas d'outils pour atteindre les objectifs possibles directement. Par conséquent, la BC doit choisir des buts atteignables qui ont un impact sur les objectifs finaux de la politique monétaire. 
Ces buts atteignables sont des "cibles intermédiaires"

Il n'y a pas de cibles intermédiaires à priori. Il faut seulement que ces cibles soient mesurables, sous le contrôle de la BC et ayant un impact sur les objectifs de la politique monétaire. Les deux cibles intermédiaires sont généralement :

\begin{itemize}
	\item La croissance de la masse monétaire
	\item Le niveau des taux d'intérêt
\end{itemize}

Ces deux cibles sont mesurables, liées aux objectifs finaux et sous le contrôle partiel de la BC. Ce contrôle n'est que partiel car
\begin{itemize}
	\item la BC ne contrôle pas directement la masse monétaire mais la base monétaire. On sait tout de même qu'il existe un lien entre la base monétaire et la masse monétaire.
	\item la BC ne contrôle que les taux d'intérêt à court terme. Or ce sont les taux longs qui importent pour les décisions d'investissement. On sait tout de même qu'il y a en général un lien entre les taux courts et les taux longs
\end{itemize}

Grâce aux deux cibles intermédiaires définies, la BC peut avoir une influence sur les objectifs possibles et notamment sur celui qui est l'objectif prioritaire aujourd'hui : la stabilité des prix.

Maintenant, il reste à la BC à intervenir sur le marché monétaire afin d'atteindre les cibles intermédiaires. Pour cela, elle dispose d'un arsenal opérationnel appelé "instruments de la politique monétaire".



Les instruments de la politique monétaires :

\begin{itemize}
	\item Opérations d'open market (taux de refinancement par appel d'offres dans la zone euro, le taux du marché des fonds fédéraux (Fed Funds) aux Etats-Unis). C'est une intervention de la BC sur le marché interbancaire, où elle achète des titres aux banques commerciales (injection de liquidité) ou en achète (retrait de liquidité).
	\item Les taux des facilités permanentes (débiteur ou créditeur) dans la zone euro, taux d'escompte (débiteur) aux Etats-Unis. C'est le taux auquel la BC prête aux banques commerciales.
	\item Le taux de change. Il n'est jamais utilisé et est décidé par les politiciens.
	\item Le taux des réserves obligatoires
	\item La croissance de la base monétaire. Depuis les années 80, le ciblage se fait sur l'inflation et non la masse monétaire car la demande de monnaie est devenue instable. Le contrôle strict de la quantité de monnaie était possible avant les années 70 car la demande était stable, mais ne l'est plus depuis la dérégularisation.
\end{itemize}

\dessinS{schema_taux}{.3}

La BC détermine ainsi le taux d'intérêt directeur, la BCE le taux de refinancement (main refinancing rate) et le Fed l'Intended Federal Funds rate. Le taux de base est défini par la banque centrale, tandis que les taux de maturité sont définis par chaque banque. Mais le taux directeur influence l'EONIA qui lui-même influence ces taux de maturités.

Plus le taux est élevé et plus on considère qu'il y a des risques. Une courbe problématique entraîne un ralentissement de l'activité économique.

Structure par risque des taux d'intérêt : pour une même maturité, les taux d'intérêt peuvent varier selon la qualité de l'émetteur (donc le risque de non remboursement). Cette probabilité de non remboursement est évaluée par les agences de notation.
	
La prime de risque : différence entre taux d'intérêt d'un pays et celui de référence


\section{Politique d'intervention}

Il y a le choix entre deux politiques :

\begin{itemize}
	\item Politique discrétionnaire : politique monétaire ajustable et réajustable au gré des circonstances économiques et financières.
	\item Poursuite à long terme d'une règle fixe et transparente pour les agents économiques quelles que soient les circonstances économiques
\end{itemize}

La différence entre les deux pour un agent économique : avec une politique à base d'une règle, il y a de la transparence et une certitude sur l'évolution des taux. De plus, cela apporte de la crédibilité.

	\subsection{Politique discrétionnaire}

	Exemple : Si le prix du baril de pétrole augmente sensiblement mais pour une raison particulière (guerre dans un pays producteur, \dots), la probabilité que la hausse des prix soit temporaire est assez élevée. A ce moment-là, on adapte la politique monétaire au contexte et on s'abstient de réagir sur le marché monétaire.

	La Réserve Fédérale a plutôt la réputation de mener une politique discrétionnaire.

	\subsection{Poursuite d'une règle}
	
	L'autre politique d’intervention de la BC est la poursuite d'une règle annoncée 
à l'avance et à laquelle se tient la BC.
Par exemple, deux types de règle :
\begin{itemize}
	\item Ciblage monétaire : la BC annonce aux agents économiques qu'elle interviendra sur le marché monétaire de telle sorte que la croissance monétaire ne dépasse pas x\% par an. C'est le type de politique monétaire préconisée par Friedman et les monétaristes.
	\item  Ciblage d'inflation : la BC annonce aux agents économiques qu'elle interviendra sur le marché monétaire de telle sorte que le taux d’inflation ne dépasse pas x\% par an. C'est la politique monétaire de la BCE.
\end{itemize}


Intérêts de poursuivre une règle : 

\begin{itemize}
	\item Politique monétaire peu interventionniste. Les monétaristes et surtout les nouveaux classiques considèrent la politique discrétionnaire comme source d'instabilité macroéconomique.
	\item Crédibilité. L'inflation est un phénomène où les anticipations des agents économiques jouent un rôle très important. Par conséquent, on considère que si la BC se tient à une politique fixe, alors les anticipations des agents économiques se calqueront sur cette politique. La poursuite d'une règle permet donc à la BC d'acquérir une crédibilité auprès des agents économiques (ménages et entreprises).

	Or, dans une économie où la monnaie est dématérialisée, la confiance dans la monnaie repose sur la stabilité des prix et donc sur la crédibilité de la politique monétaire.
\end{itemize}

	\subsection{Dans les faits}
	
	Le choix annoncé d'une politique d'intervention de la BC peut toutefois être légèrement différent en pratique. Normalement, si la BC fixe une règle, cela veut dire qu'elle va déterminer le taux d'intérêt en fonction des conditions économiques et de l'objectif poursuivi (stabilité des prix). La question est de savoir comment utiliser le taux d'intérêt pour atteindre 
l'objectif du taux d'inflation souhaité, d'où la règle de Taylor :

	$$i = i* + a(\pi - \pi*) - b(u - u*)$$

	\begin{itemize}
		\item $i$ est le niveau du taux d'intérêt nominal à court terme qu'il faut fixer;
		\item $i*$ est le taux d'intérêt nominal visé et compatible avec le taux d'inflation souhaité $\pi*$;
		\item $\pi$ est le taux d'inflation courant;
		\item $u$ est le taux de chômage courant;
		\item $u*$ est le taux de chômage structurel ;
		\item $a$ et $b$ sont des coefficients estimés selon les cas ; $a$ favorise l'inflation (BCE) et $b$ le chômage (Etats-Unis).
	\end{itemize}

	% 29/04/2013
\chapter{Les produits financiers dérivés}

Un produit dérivé est un actif financier dont la valeur dépend de la valeur d'actifs sous-jacents

\section{Les forwards de taux d'intérêt}

Un forward est un contrat irrévocable entre 2 parties stipulant qu'une transaction financière déterminée sera réalisée à une date future fixée. C'est un engagement ferme entre deux parties d'acheter ou de vendre un actif sous-jacent à un prix convenu et à une date fixée (donc avec une certaine échéance). Un forward de taux d'intérêt implique la vente future d'un instrument de dette.

Par exemple, un forward de taux d'intérêt entre BNP Paribas et l'entreprise d'assurances AXA peut prendre la forme d’un engagement de BNP Paribas à vendre à Axa, dans un an, des obligations du Trésor américain de première catégorie d'échéance 2030 et de coupon 6\% pour un montant de 5 millions de \$, à un prix défini tel que ces titres rapportent le même rendement (yield) qu’aujourd'hui, à savoir 6\%. 

L'acheteur a ce qu'on appelle une position longue, tandis que le vendeur a une position courte.

Dans l'exemple, Axa compte sur une diminution des taux sur le temps pour avoir une augmentation des prix des actions. A l'inverse, BNP spécule sur une augmentation des taux, donc une diminution du prix des actions qu'elle a dans son porte-feuille.

Un forward de taux d'intérêt suppose la définition de plusieurs paramètres: 
\begin{itemize}
	\item Les caractéristiques de l'instrument de dette qui fera l'objet de la vente 
	\item Le montant de la dette 
	\item Le prix (le taux d'intérêt) de l'instrument de dette 
	\item La date à laquelle s'effectuera la livraison
\end{itemize}

L'exemple soulève plusieurs questions :

\begin{itemize}
	\item Pourquoi BNP Paribas accepte-t-elle de conclure ce contrat avec Axa ? 
	\item[$\rightarrow$] il y a un risque de perte car les actions sont liées au taux d'intérêt, s'il augmente le prix des actions diminuera.
	\item Comment peut-elle se couvrir contre ce risque ?
	\item[$\rightarrow$] elle peut se couvrir avec un forward avec taux d'intérêt, afin d'avoir une position courte. Grâce au forward de taux, on est sûr du prix qu'il y aura dans un an vu qu'il est fixé.
	\item Pourquoi Axa accepte-t-elle de conclure le contrat forward ?
	\item[$\rightarrow$] Le risque d'Axa est que le taux du marché diminue et que le prix de ses actions augmente.
\end{itemize}


	\subsection{Avantages et inconvénients}
	
	\begin{itemize}
		\item[+] souplesse car marché de gré à gré
		\item[-] déficit de liquidité
		\item[-] risque de défaut ou de contrepartie 
	\end{itemize}
	
\section{Les futures de taux d'intérêt}

Un contrat financier future de taux d'intérêt long terme est semblable à un forward de taux d'intérêt, dans la mesure où il spécifie qu'un titre de dette fera l'objet d'une transaction future entre 2 parties, à une date déterminée. La différence est qu'un future permet de s'affranchir des risques de liquidité et de défaut liés aux contrats forward.

La différence est qu'on se trouve sur un marché boursier, et donc les risques de liquidité et de défaut sont moindres.

Exemple : pour un future sur obligation dont le notionel (montant des obligations sous-jacentes) est de 100 000, il faut payer 149 890 avec une cote 149,89 (avec un point de base ($\Leftrightarrow$ un pourcent) égal à 1000\$).

	\subsection{Couverture}
	
	On a
	
	$$NC = \frac{VA}{VC}$$
	
	\begin{itemize}
		\item NC, nombre de contrats
		\item VA, valeur de l'actif à couvrir
		\item VC, valeur unitaire du contrat
	\end{itemize}

	On peut distinguer
	\begin{itemize}
		\item microcouverture : par un future on a ...
		\item macrocouverture : on ne couvre pas un seul actif mais sur plusieurs types d'actif. On choisit l'actif sous-jacent qui permet au mieux de couvrir l'ensemble du porte-feuille
	\end{itemize}

	\subsection{Organisation des transactions sur le marché des futures}
	
	\begin{itemize}
		\item Standardisation des montants des contrats et des échéances 
		\item Marché liquide 
		\item Plusieurs actifs sous-jacents peuvent indifféremment être livrés 
		\item Possibilité de traiter les futures en continu 
		\item Chambre de compensation donc pas de risque de défaut 
		\item Dépôt de marge initial ( deposit ou margin requirement) sur un compte de marge ; compte de garantie
		\item Evaluation des futures au jour le jour (aussi appelé le principe marked to market). Les gains ou les pertes alimentent le compte détenu à la chambre de compensation
		\item Si le solde d'un compte de marge passe en dessous d'un minimum préétabli, la chambre de compensation demande de procéder à un nouveau versement sur le compte (assurance contre le risque de défaut de l'agent)
		
		\item avec un futur, possibilité de conclure un contrat arrivé à échéance sans procéder à la livraison physique de l'actif sous-jacent ( position fermée)
	\end{itemize}
	
	\subsection{Couverture du risque de change}
	
	(Illustration) Le 1er mars, une entreprise américaine s'attend à recevoir 50 millions de yens. Le risque de change est une baisse du dollar par rapport au yen. Elle peut alors contracter un futur sur le yen, avec une échéance à la fin du trimestre (fin juin). La taille du contrat est de 12,5 millions de yens, elle peut vendre 4 contrats (car $4 \times 12,5$ millions = 50 millions). Elle rachètera alors ses futures à un taux plus bas.

	Exemple chiffré : le 1er mars, le taux de change pour un futur est 1 yen = 0.78 cents. Le 30 juin, sur le marché des changes comptant, 1 yen = 0.72 cents et le prix du futur est de 0.725 cents. Si une entreprise prend la position de vendeuse sur 4 contrats à terme, son gain sur le marché des futures est de 0.78 - 0.725 = 0.055 cents par yen. Le 30 juin, elle vendra ses yens à 0.72 cents. Grâce à l'opération de couverture, c'est comme si elle vendait ses yens le 30 juin à 0.72 + 0.055.
	
	
\section{Les options}

Les options sont des contrats qui donnent le droit à leur acheteur d'acheter 
ou de vendre l'actif sous-jacent à un prix spécifique ( le prix d'exercice ou 
strike) pendant une période donnée ou à une date donnée. 

Le vendeur de l'option est obligé de vendre ou d'acheter l'actif sous-jacent 
si l'acheteur de l'option décide d'exercer son droit d'acheter ou de vendre.

L'acheteur de l'option doit payer un montant défini (la prime) au vendeur. Il existe deux types d'options: américaines et européennes (ne peut être exercé qu'à l'échéance).

Un call est un contrat qui donne à son détenteur le droit d'acheter le sous-jacent au prix d'exercice fixé. 

Un put est inversement un contrat qui donne le droit à son détenteur de 
vendre l'actif sous-jacent au prix d'exercice .

	\subsection{Profils de gains et de pertes des futures et des options}
	
	En février, un agent A achète contre 2000\$ un call sur le contrat future des obligations du Trésor américain, échéance fin juin, montant facial 100000\$ et prix d'exercice 115. Option européenne. Le montant notionnel est de 100 000\$.

Supposons: à l'échéance, le future sous-jacent cote 110, donc les obligations sous-jacentes valent 110. L'agent A n'exerce pas son option et perd la prime de 2000\$.

Dans une telle situation, le prix de l'actif sous-jacent est inférieur au prix d'exercice : option « out of the money » 

Supposons, à la date d'échéance, le prix du future est de 115. L'agent A est indifférent entre exercer ou non l'option. Perte du montant de la prime. C'est une option "at the money" car le prix d'exercice est égal à la quotation de l'actif sous-jacent.

Supposons, à la date d'échéance, le future cote 120. L'agent A exerce l'option. Gain: ((120-115) x 1000\$)- 2000\$ = 3000\$ . Option "in the money". 

Supposons, à la date d'échéance, le future cote 125. Gain : 8000\$

Supposons qu'au lieu d'acheter l'option sur le future, l'agent A ait investi directement sur le future. 

\begin{itemize}
	\item Si le prix des obligations décline jusqu'à 110, le prix du futur baisse lui aussi à 110. Perte : 5 x 1000\$ = 5000\$ 
	\item Si le prix des obligations vaut 115, perte ou gain nul 
	\item Si le prix des obligations vaut 120, gain 5000\$ 
	\item Si le prix des obligations vaut 125, gain : 10000\$
\end{itemize}


Principales différences entres les futures et options :

\begin{itemize}
	\item Les fonctions de gains des futures sont linéaires, contrairement à celles des 
options qui sont non linéaires (voir graphique) 
	\item Achat d'un future : appel de marge /Achat d'une option : prime 
	\item Future : marked to market / Option européenne : non
\end{itemize}

Illustration: se couvrir grâce aux options : les options apparaissent comme des instruments extrêmement précieux pour la macrocouverture du risque de taux encouru par les banques et les institutions financières.

\begin{center}
\begin{tabular}{|c|c|c|}
\hline 
  & CALL & PUT \\ 
\hline 
Prix du sous-jacent & + & - \\ 
\hline 
Prix d'exercice & - & + \\ 
\hline 
Temps jusqu'à échéance & + & + \\ 
\hline 
Volatilité & + & + \\ 
\hline 
Taux d'intérêt & + & - \\ 
\hline 
Dividendes attendus & - & + \\ 
\hline 
\end{tabular} 
\end{center}


\section{Les swaps de taux d'intérêt}

Le swap le plus courant : swap de taux « vanille » . On échange les cash-flows liés aux taux d'intérêt, donc seule la charge d'intérêt est échangée, mais on n'échange pas le principal (contrairement à un swap de devises).

Dans un tel contrat, une entreprise s'engage à payer des cash-flows égaux aux intérêts à taux fixe sur un principal donné, pendant un certain nombre d'années. En retour, elle reçoit des intérêts à un taux variable sur le même principal, pendant la même durée. 

Le taux variable le plus courant dans les contrats de swap est le taux LIBOR. 

Exemple : considérons un swap à 3 ans entre deux entreprises, Microsoft et Mobistar, initié le 5/03/2013. Microsoft s’engage à payer un taux d'intérêt de 5\% à Mobistar sur un principal de 100 millions de dollars US. En retour, Mobistar s'engage à payer des intérêts à Microsoft au taux Libor 6 mois. Les flux seront échangés tous les 6 mois et le taux fixe de 5\% est en composition semestrielle. Microsoft est le payeur de taux fixe et Mobistar le payeur de taux variable. Grâce au swap, Microsoft transforme sa dette à taux variable en une dette à taux fixe de 5,1 \%.

\dessinS{8}{.30}
% 6/5/2013

	\subsection{L'utilisation des swaps pour transformer un engagement}
	
	Pour Microsoft, le swap pourrait être utilisé pour transformer une dette à 
taux variable en une dette à taux fixe.  Supposons que Microsoft ait contracté une dette de 100 millions au taux Libor + 10 points de base. Cash-flows équivalents à un emprunt à taux fixe à 5,1\% 

	3 ensembles de cash-flows: 
	\begin{itemize}
		\item Paiement de Libor + 10 à ses prêteurs 
		\item Réception du Libor sur le swap 
		\item Paiement de 5\% selon les termes du swap
	\end{itemize}

	Pour Mobistar, ce swap pourrait être employé pour transformer une dette à taux fixe en une dette à taux variable. Supposons que Mobistar ait contracté un emprunt au taux fixe de 5,2\% pour 100 millions.  Après la mise en place du swap, 3 types de cash-flows seront reçus ou payés par Mobistar: 
	\begin{itemize}
		\item Paiement de 5,2\% à ses prêteurs 
		\item Réception de 5\% sur le swap 
		\item Paiement du Libor sur le swap 
	\end{itemize}

	Ces cash-flows sont équivalents à un emprunt à taux variable Libor + 20.
	
	\dessinS{9}{.35}

	\subsection{L'utilisation d'un swap pour transformer un actif}

	
	Un swap peut aussi servir pour Microsoft, à transformer un actif rapportant un taux fixe en un actif rapportant un taux variable. Supposons que Microsoft détienne un portefeuille d'obligations de 100 millions qui paient 4,7\% de taux de coupon (semestriel) dans les 3 prochaines années.
	
	3 séries de cash-flows :
	
	\begin{itemize}
		\item L'encaissement de 4,7\% sur le portefeuille d'obligations 
		\item Réception du Libor sur le swap 
		\item Paiement de 5\% sur le swap 
	\end{itemize}

	\dessinS{10}{.35}
	
	En fin de compte, le portefeuille d'obligations à taux fixe a été transformé en un portefeuille obligataire à taux variable rapportant Libor – 30 (points de base, soit 0.3\%). 

	Symétriquement, si Mobistar possède un portefeuille d'obligations à taux variable rapportant Libor -20, le swap permet de transformer ce portefeuille en portefeuille de titres à taux fixe rapportant 4,8\%.
	
	\subsection{L'avantage comparatif}
	
	\dessinS{11}{.45}
	
	Bcorp souhaite emprunter à taux fixe alors que Acorp préférerait emprunter à taux variable, fondé sur LIBOR 6 mois. Bcorp a donc un avantage comparatif sur le taux variable et Acorp un avantage sur le marché à taux fixe. Il va y avoir contraction d'un swap.
	
	La société B emprunte avec un plus grand taux que A, mais la différence de taux variable (0.7) est moindre que la différence de taux fixe (1.2), elle est moins pénalisée sur le taux variable. Un swap est possible entre ces taux : le gain du swap sera $1.2 - 0.7 = 0.5$ à répartir entre les deux parties (A aura $0.25$).
	
	Séries de cashflow pour A : 1) emprunte 10 millions au taux fixe avec intérêt à 4\%, 2) dans le cadre du swap elle reçoit de B 3.95\% (taux fixe) et paie le taux d'intérêt variable Libor, comme si A avait emprunté à Libor + 5 points de base (à la place de Libor + 30, soit 25 points de base en moins.
	
	Séries de cashflow pour B : 1) paie les intérêts au taux Libor + 1 \%, et 2) dans le cadre du swap paie à A un taux fixe de 3.95 et reçoit de A le taux d'intérêt Libor, comme si B a emprunté à Libor + 4,95\%.
	
	\dessinS{12}{.35}

	Grâce au swap, A passe d'un crédit à taux fixe à un crédit à taux variable, tandis que B passe d'un taux variable à un taux fixe. Il y a un gain de 25 points de base des deux côtés : $(5,2\%-4\%) – (Libor + 100) – (Libor + 30) = 0,5\%$
	
	\subsection{Avantages et inconvénients}
	
	Le swap permet aux institutions financières et entreprises de convertir des actifs à taux fixe en actifs à taux variables ou inversement, sans coût et modification du bilan.

	Les swaps peuvent être conclus pour des durées de vie très longues (jusqu'à 20 ans) alors que les options et les futures ont une moyenne de durées de vie beaucoup plus courtes, inférieures à 1 an.
	
	Toutefois, le marché des swaps peut souffrir d'un risque de liquidité 
	
	Les contrats de swaps sont exposés au risque de contrepartie 

	Les grandes banques commerciales et banques d'investissement ont créé des marchés de swaps sur lesquels elles interviennent en tant qu'intermédiaires. 

\section{Les swaps de change}

Un swap de devises implique l'échange d'un principal et d'intérêts dans une devise contre un principal et des intérêts dans une autre. Ces sommes sont échangées au début et à la fin de la durée de vie du swap.

Un swap peut être utilisé pour transformer un emprunt dans une devise en un emprunt dans une autre devise. 

Un swap peut être utilisé pour transformer la nature des actifs. Supposons que AIG puisse investir 10 millions de £ en Angleterre à 5\% par an pour les 5 prochaines années mais pense que le \$ va se renforcer par rapport à la £ pendant ce temps. Le swap a pour effet de transformer l'investissement en £ à 5\% en un investissement en \$ à 6\%.

Un swap de taux comme de devise permettent de transformer un emprunt ou un actif, en échangeant taux variable/taux fixe ou les devises.


Exemple : supposons que Virgin souhaite emprunter 20 millions de AUD et Qantas Airways 15 millions de \$ et que le taux de change soit de 0,75 \$/AUD. Chaque entreprise va emprunter sur le marché sur lequel elle a un avantage comparatif et conclure un swap qui transformera l'emprunt de GE en un emprunt en AUD et celui de Qantas Airways en un emprunt en \$. 

\dessinS{13}{.4}


\end{document}	