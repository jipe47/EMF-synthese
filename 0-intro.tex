\chapter{Introduction}

\section{Le système financier}
\textbf{Le marché financier} est le lieu où on échange de la monnaie (aussi appelée des liquidités). C'est un actif qui est plus ou moins liquide.

On y trouve

\begin{itemize}
	\item des agents économiques qui ont besoin de financement et qui demandent de la monnaie, par exemple des entreprises, l'Etat.
	\item des agents économiques qui ont des capacités de financement et qui offrent de la monnaie, par exemple les ménages.
\end{itemize}

On distingue le marché de la monnaie (où l'actif échangé est de la monnaie) et le marché financier (où l'ensemble des actifs financiers sont considérés).

\dessinS{1}{0.5}

Les banques commerciales ont le monopole de la création monétaire, mais elles sont contrôlées par la banque centrale (BC). A l'opposé, les acteurs du système financier non bancaire ne peuvent pas créer de monnaie.

La BC permet de réguler la monnaie produite. S'il y a trop de monnaie par rapport à la demande, c'est l'inflation, il y a une augmentation des prix.

Les prix sont fixés par taux d'intérêt.

Si une banque a besoin de liquidités, elle prête de la monnaie contre des titres, qu'elle peut revendre quand c'est nécessaire.

\textbf{Un marché de gré à gré} (over-the-counter market) est un marché où il y a un accord sur un prix. On a exactement ce qu'on veut sur mesure et à un prix fixé (CDO : collaterised debt obligation). Le problème est que si ce qu'on veut est trop spécifique, on dit qu'il est trop liquide et peu échangeable. A l'opposé, la monnaie est hyper standardisée.

\section{La monnaie et les autres marchés}

Il y a des interdépendances entre les marchés. Ainsi, un déséquilibre sur les marchés financiers peut entraîner un risque de déséquilibre sur les autres marchés, et inversement. Par exemple, si les banques décident de ne plus octroyer de crédit, les entreprises qui ont besoin de liquidités périclitent.

La monnaie est un moyen de transférer de la valeur du présent vers le futur.

En économie, le futur est incertain. On tente alors de l'anticiper ; les agents économiques tentent de prédire les valeurs futures du marchés. Si ces anticipations sont optimistes, un boom économique est possible. Sinon, une crise peut arriver. Dès que les anticipations varient, on peut faire plonger ou grimper les marchés boursiers, ce qui affecte toute l'économie.
