\chapter{Introduction}

Le marché financier est le lieu où on échange de la monnaie (aussi appelée des liquidités). C'est un actif qui est plus ou moins liquide.

On y trouve

\begin{itemize}
	\item des agents économiques qui ont besoin de financement et qui demande de la monnaie, par exemple des entreprises, l'Etat.
	\item des agents économiques qui ont des capacités de financement et qui offre de la monnaie, par exemple les ménages.
\end{itemize}

On distingue le marché de la monnaie (où l'actif échangé est de la monnaie) et le marché financier (où l'ensemble des actifs financiers sont considérés).

\dessin{1}

Les banques commerciales ont le monopole de la création monétaire, mais elles sont contrôlées par la banque centrale (BC). A l'opposé, les acteurs du système financier non bancaire ne peuvent pas créer de monnaie.

La BC permet de réguler la monnaie produite. S'il y a trop de monnaie par rapport à la demande, c'est l'inflation, il y a une augmentation des prix.

Les prix sont fixés par taux d'intérêt.

Si une banque a besoin de liquidité, elle prête de la monnaie contre des titres, qu'elle peut revendre quand c'est nécessaire.

Un marché de gré à gré (over-the-counter market) est un marché où il y a un accord sur un prix. On a exactement ce qu'on veut sur mesure et à un prix fixé (CDO : collaterised debt obligation). Le problème est que si ce qu'on veut est trop spécifique, on dit qu'il est trop liquide et peu échangeable. A l'opposé, la monnaie est hyper standardisée.