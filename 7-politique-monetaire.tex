\chapter{La politique monétaire}

Autrefois, les Etats battaient monnaie et les banques centrales dépendaient directement du pouvoir politique. Les banques centrales utilisaient des instruments autoritaires pour contrôler l'offre de monnaie : il y avait un encadrement autoritaire du crédit.

Aujourd'hui, les banques centrales sont indépendantes du pouvoir politique. La politique monétaire prend la forme d'une régulation monétaire où les instruments utilisés par les banques centrales exercent une contrainte de liquidité sur le principal levier de la création monétaire : le crédit bancaire.

\section{Les canaux de transmission de la politique monétaire}

	\subsection{Les canaux de transmission réels}

Une modification de la politique monétaire de la banque centrale a des effets sur le court terme mais pas sur le long terme, même s'il y a des effets sur les prix.  Les variations du taux d'intérêt modifie toute une série de comportements économiques :

\begin{itemize}
	\item Consommation et épargne
	\item L'investissement des entreprises et l'investissement des ménages (biens immobiliers)
	\item La compétitivité des entreprises par rapport à celle des entreprises étrangères, car si le taux augmente, la marge des entreprises va diminuer
	\item La dette publique
	\item Le secteur bancaire qui voit ses profits baisser, car il ne peut pas changer facilement son taux directeur ; si le taux d'intérêt augmente la marge va aussi diminuer
\end{itemize}

	\subsection{Les canaux de transmission monétaires}
	
	A moyen et long terme, l'offre de monnaie a un effet sur le taux d'inflation : c'est la théorie quantitative de la monnaie.	A court terme, on peut considérer qu'il existe une certaine rigidité des prix et notamment une rigidité des salaires nominaux. La politique monétaire, en manipulant le taux d'intérêt, peut affecter la production et l'emploi si les prix ne s'ajustent pas instantanément (théorie keynésienne). L'influence de la politique monétaire sur les salaires vient de leur rigidité. Si l'impact était immédiat, il n'y aurait pas d'effets.	La demande de monnaie dépend aussi du niveau des taux d'intérêt.
	
\section{Neutralité de la monnaie}

C'est un débat essentiel en économie monétaire. D'un point de vue pragmatique, toutes les banques centrales considèrent que la politique monétaire a des effets réels à court terme. Si on manipule trop la monnaie, il y aura un effet sur les prix.

\section{Objectifs de la politique monétaire}

\begin{itemize}
	\item La stabilité des prix
	\item La croissance économique et le plein emploi
	\item L'équilibre extérieur (équilibre des comptes de la balance des paiements). Si un pays importe plus qu'il n'exporte, la balance courante en déficit, il a une dette envers les autres pays. Pour contrebalancer, il va demander un crédit. Si les autres pays refusent de faire crédit, il y a une crise de la balance courante. C'est ce qu'il se passe actuellement entre les USA et la Chine, qui achète les bons du trésor américains tandis que les américains achètent des produits chinois. Lorsqu'un pays est en cessation de payement (c'est-à-dire qu'il ne peut plus rien importer), il s'adresse au FMI.
	\item La stabilité du système financier : il y a injection de liquidités pour que les investisseurs ne revendent pas leur parts.
\end{itemize}

Ces objectifs sont souhaitables mais difficilement compatibles et atteignable ensemble et en même temps. Il faut dès lors faire un choix :

\begin{itemize}
	\item Dans les années 60, l'objectif était plutôt la croissance économique. La politique monétaire était donc plus laxiste par rapport à l'objectif de stabilité des prix. Peu à peu, les taux d'inflation ont augmenté pour passer à deux chiffres à la fin des années 70 orsque l'inflation devient grande (5-7\%), déclenchement du mécanisme des [...].
	\item Au début des années 80, c'est l'objectif de stabilité des prix qui est devenu prioritaire pour un grand nombre de banques centrales.
\end{itemize}

\section{Objectifs de la BCE}

Son objectif (stipulé dans le traité de Maastricht) est la stabilité des prix, qui est défini comme un taux d'inflation annuel inférieur à 2\% et proche de la valeur de 2\%. Cette valeur est choisie pour qu'on n'atteigne jamais la déflation, ce qui pourrait être le cas avec 1\%.

\section{Les outils de la politique monétaire}

La BC n'a pas d'outils pour atteindre les objectifs possibles directement. Par conséquent, la BC doit choisir des buts atteignables qui ont un impact sur les objectifs finals de la politique monétaire. 
Ces buts atteignables sont des "cibles intermédiaires"

Il n'y a pas de cibles intermédiaires a priori. Il faut seulement que ces cibles soient mesurables, sous le contrôle de la BC et ayant un impact sur les objectifs de la politique monétaire. Les deux cibles intermédiaires sont généralement :

\begin{itemize}
	\item La croissance de la masse monétaire
	\item Le niveau des taux d'intérêt
\end{itemize}

Ces deux cibles sont mesurables, liées aux objectifs finaux et sous le contrôle partiel de la BC. Ce contrôle n'est que partiel car
\begin{itemize}
	\item la BC ne contrôle pas directement la masse monétaire mais la base monétaire. On sait tout de même qu'il existe un lien entre la base monétaire et la masse monétaire.
	\item la BC ne contrôle que les taux d'intérêt à court terme. Or ce sont les taux longs qui importent pour les décisions d'investissement. On sait tout de même qu'il y a en général un lien entre les taux courts et les taux longs
\end{itemize}

Grâce aux deux cibles intermédiaires définies, la BC peut avoir une influence sur les objectifs possibles et notamment sur celui qui est l'objectif prioritaire aujourd'hui : la stabilité des prix.

Maintenant, il reste à la BC à intervenir sur le marché monétaire afin d'atteindre les cibles intermédiaires. Pour cela, elle dispose d'un arsenal opérationnel appelé "instruments de la politique monétaire".



Les instruments de la politique monétaires :

\begin{itemize}
	\item Opérations d'open market (taux de refinancement par appel d'offres dans la zone euro, le taux du marché des fonds fédéraux (Fed Funds) aux Etats-Unis). C'est une intervention de la BC sur le marché interbancaire, où elle achète des titres aux banques commerciales (injection de liquidité) ou en achète (retrait de liquidité).
	\item Les taux des facilités permanentes (débiteur ou créditeur) dans la zone euro, taux d'escompte (débiteur) aux Etats-Unis. C'est le taux auquel la BC prête aux banques commerciales.
	\item Le taux de change. Il n'est jamais utilisé et est décidé par les politiciens.
	\item Le taux des réserves obligatoires
	\item La croissance de la base monétaire. Depuis les années 80, le ciblage se fait sur l'inflation et non la masse monétaire car la demande de monnaie est devenue instable. Le contrôle strict de la quantité de monnaie était possible avant les années 70 car la demande était stable, mais ne l'est plus depuis la dérégularisation.
\end{itemize}

\dessinS{schema_taux}{.3}

La BC détermine ainsi le taux d'intérêt directeur, la BCE le taux de refinancement (main refinancing rate) et le Fed l'Intended Federal Funds rate. Le taux de base est défini par la banque centrale, tandis que les taux de maturité sont définis par chaque banque. Mais le taux directeur influence l'EONIA qui lui-même influence ces taux de maturités.

Plus le taux est élevé et plus on considère qu'il y a des risques. Une courbe problématique entraîne un ralentissement de l'activité économique.

Structure par risque des taux d'intérêt : pour une même maturité, les taux d'intérêt peuvent varier selon la qualité de l'émetteur (donc le risque de non remboursement). Cette probabilité de non remboursement est évalué par les agences de notation.
	
La prime de risque : différence entre taux d'intérêt d'un pays et celui de référence


\section{Politique d'intervention}

Il y a le choix entre deux politiques :

\begin{itemize}
	\item Politique discrétionnaire : politique monétaire ajustable et réajustable au grédes circonstances économiques et financières.
	\item Poursuite à long terme d'une règle fixe et transparente pour les agents économiques quelles que soient les circonstances économiques
\end{itemize}

La différence entre les deux pour un agent économique : avec une politique à base d'une règle, il y a de la transparence et une certitude sur l'évolution des taux. De plus, cela apporte de la crédibilité.

	\subsection{Politique discrétionnaire}

	Exemple : Si le prix du baril de pétrole augmente sensiblement mais pour une raison particulière (guerre dans un pays producteur …), la probabilité que la hausse des prix soit temporaire est assez élevée. A ce moment-là, on adapte la politique monétaire au contexte et on s'abstient de réagir sur le marché monétaire.

	La Réserve Fédérale a plutôt la réputation de mener une politique discrétionnaire.

	\subsection{Poursuite d'une règle}
	
	L'autre politique d’intervention de la BC est la poursuite d'une règle annoncée 
à l'avance et à laquelle se tient la BC.
Par exemple, deux types de règle :
\begin{itemize}
	\item Ciblage monétaire : la BC annonce aux agents économiques qu'elle interviendra sur le marché monétaire de telle sorte que la croissance monétaire ne dépasse pas x\% par an. C'est le type de politique monétaire préconisée par Friedman et les monétaristes.
	\item  Ciblage d'inflation : la BC annonce aux agents économiques qu'elle interviendra sur le marché monétaire de telle sorte que le taux d’inflation ne dépasse pas x\% par an. C'est la politique monétaire de la BCE.
\end{itemize}


Intérêts de poursuivre une règle : 

\begin{itemize}
	\item Politique monétaire peu interventionniste. Les monétaristes et surtout les nouveaux classiques considèrent la politique discrétionnaire comme source d'instabilité macroéconomique.
	\item Crédibilité. L'inflation est un phénomène où les anticipations des agents économiques joue un rôle très important. Par conséquent, on considère que si la BC se tient à une politique fixe, alors les anticipations des agents économiques se calqueront sur cette politique. La poursuite d'une règle permet donc à la BC d'acquérir une crédibilité auprès des agents économiques.

	Or, dans une économie où la monnaie est dématérialisée, la confiance dans la monnaie repose sur la stabilité des prix et donc sur la crédibilité de la politique monétaire.
\end{itemize}

	\subsection{Dans les faits}
	
	Le choix annoncé d'une politique d'intervention de la BC peut toutefois être légèrement différent en pratique. Normalement, si la BC fixe une règle, cela veut dire qu'elle va déterminer le taux d'intérêt en fonction des conditions économiques et de l'objectif poursuivi (stabilité des prix). La question est de savoir comment utiliser le taux d'intérêt pour atteindre 
l'objectif du taux d'inflation souhaité, d'où la règle de Taylor :

	$$i = i* + a(\pi - \pi*) - b(u - u_n)$$

	\begin{itemize}
		\item $i$ est le niveau du taux d'intérêt nominal à court terme qu'il faut fixer;
		\item $i*$ est le taux d'intérêt nominal visé et compatible avec le taux d'inflation souhaité $\pi*$;
		\item $\pi$ est le taux d'inflation courant;
		\item $u$ est le taux de chômage courant;
		\item $u*$ est le taux de chômage structurel ;
		\item $a$ et $b$ sont des coefficients estimés selon les cas ; $a$ favorise l'inflation (BCE) et $b$ le chômage (Etats-Unis).
	\end{itemize}
