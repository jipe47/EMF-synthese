\part{La politique monétaire}

Une modification de la politique monétaire de la banque centrale a des effets sur le court terme mais pas sur le long terme, même s'il y a des effets sur les prix.

3 : si le taux augmente, la marge des entreprises va diminuer. Le secteur bancaire voit ses profits baisser car il ne peut pas changer facilement son taux directeur ; si le taux d'intérêt augmente la marge va aussi diminuer.

4 : l'influence de la politique monétaire sur les salaires vient de leur rigidité. Si l'impact était immédiat, il n'y aurait pas d'effets.

5 : si on manipule trop la monnaie, il y aura un effet sur les prix.

6 : si un pays importe plus qu'il n'exporte => balance courante en déficit, il a une dette envers les autres pays. Pour contrebalancer, il va demander un crédit. Si les autres pays refusent de faire crédit, il y a une crise de la balance courante. C'est ce qu'il se passe actuellement entre les USA et la Chine, qui achète les bons du trésor américain tandis que les américains achètent des produits chinois. Lorsqu'un pays est en cessation de payement (<=> il ne peut plus rien importer), il s'adresse au FMI.

système financier : il y a injection de liquidités pour que les investisseurs ne revendent pas leur parts.

7 : Lorsque l'inflation devient grande (5-7\%), déclenchement du mécanisme des [...].

8 : 2\% pour être sûr qu'on n'atteindra jamais la déflation, ce qui pourrait être le cas avec 1\%.

\section{Les outils de la politique monétaire}

10 : Le contrôle n'est que partiel, car la BCE n'a qu'à contrôle la base monétaire pour modifier la masse monétaire.

La BC détermine le taux d'intérêt directeur, la BCE le taux de refinancement (main refinancing rate) et le Fed l'Intended Federal Funds rate.

Le taux de base est défini par la banque centrale, tandis que les taux de maturité sont définis par chaque banque. Mais le taux directeur influence l'EONIA qui lui-même influences ces taux de maturité.

[courbe des taux]

Plus le taux est élevé et plus on considère qu'il y a des risques.

Structure par risque des taux d'intérêt :

\begin{itemize}
	\item pour une même maturité, les taux d'intérêt peuvent varier selon la qualité de l'émetteur (donc le risque de non remboursement). Cette probabilité de non remboursement est évalué par les agences de notation.
	
	Prime de risque : différence entre taux d'intérêt d'un pays et celui de référence
\end{itemize}

12 : le taux de change n'est jamais utilisé. Il est décidé par les politiciens.

Croissance de la base monétaire : depuis les années 80, le ciblage se fait sur l'inflation et non la masse monétaire car la demande de monnaie est devenue instable. Le contrôle strict de la quantité de monnaie était possible avant les années 70 car la demande était stable, mais ne l'est plus depuis la dérégularisation.

Taux des facilités permanentes : taux auquel la BC prête aux banques commerciales.

Opérations d'open market : intervention de la BC sur le marché interbancaire, où elle achète des titres aux banques commerciales (injection de liquidité) ou en achète (retrait de liquidité).

\section{Politique d'intervention}

Différence entre les deux politiques pour un agent économique : avec une politique à base d'une règle, il y a de la transparence et une certitude sur l'évolution des taux. De plus, cela apporte de la crédibilité.

18 : a et b sont des coefficients estimés selon les cas (a favorise l'inflation (BCE), b le chômage (Etats-Unis)).