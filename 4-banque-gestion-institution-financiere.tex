\chapter{La banque et la gestion des institutions financières}

	\section{Le bilan bancaire}
	
\begin{center}
	\begin{tabular}{|c|c|}
	\hline 
	Actif & Passif \\ 
	\hline 
	Caisse, banques centrales & Caisse, banques centrales \\ 
	\hline 
	Prêts aux établissements de crédit & Emprunts auprès des établissements de crédit \\ 
	\hline 
	Crédits à la clientèle & Ressources émanant de la clientèle \\ 
	\hline 
	Opérations sur titres & Opérations sur titres \\ 
	\hline 
	Valeurs immobilisées & Provisions, capitaux propres \\ 
	\hline 
	Divers & Divers \\ 
	\hline 
	\end{tabular} 
\end{center}
	
	Passif :
	
	\begin{itemize}
		\item Opérations sur titres : titres émis sur le marché des capitaux à court terme
		\item Provisions en cas de dépréciation d'actifs. 
		
			Capitaux propres : capital de la banque + ses bénéfices non distribués + toutes les dettes subordonnées (c'est-à-dire que la dette ne sera remboursée qu'après les autres dettes)
	\end{itemize}
	
	Actif :
	
	\begin{itemize}
		\item Caisse, banques centrales : liquidité en dépôt auprès de la banque centrale
		\item Prêts aux établissements de crédit : argent de la banque en dépôt auprès d'autres banques/établissement de crédits
		\item Opérations sur titres : titres détenus par la banque soit pour son propre compte soit pour compte de la clientèle
		\item Valeurs immobilisées : immobilisations corporelles et incorporelles de la banque
		\item Prêts subordonnées : participation dans des entreprises liées et remboursés en dernier lieu
	\end{itemize}
	
	\section{L'exploitation bancaire}
	
	Les mécanismes de base du fonctionnement d'une banque :
	
	\begin{itemize}
		\item Transformation d'actifs : transformation des dépôts de sa clientèle pour accorder des prêts ; elle a un rôle d'intermédiation
		\item Transformation d'échéances : par exemple emprunter à court terme sur les marchés pour prêter à long terme
		\item La banque vend des services qu'elle facture à ses clients : par exemple traitement des virements, relevés bancaires, conseils en placement
	\end{itemize}
	
	Un dépôt dans une banque entraine une réserve supplémentaire.
	
\begin{center}
	\begin{tabular}{|c|c|c|c|}
	\hline 
	\multicolumn{2}{|c|}{Actif} & \multicolumn{2}{|c|}{Passif} \\ 
	\hline 
	Réserves & +100 & Dépôts & +100 \\ 
	\hline 
	\end{tabular} 
\end{center}

	Comment les banques réagissent à une hausse du montant des dépôts.
	
	\begin{center}
	\begin{tabular}{|c|c|c|c|}
	\hline 
	\multicolumn{2}{|c|}{Actif} & \multicolumn{2}{|c|}{Passif} \\ 
	\hline 
	Réserves obligatoires & +10 & Dépôts & +100 \\ 
	\hline 
	Réserves & +90 & & \\ 
	\hline 
	\end{tabular} 
\end{center}
	
	\section{Principes de gestion de bilan}
	
	\subsection{Gestion de liquidité}
	
	Une banque doit s'assurer qu'elle dispose d'assez de réserve/de liquidité pour rembourser les déposants qui retirent de l'argent de leur compte. Par exemple, avec un taux de 10\%, les dépôts des clients doivent être assurés à hauteur de 10\% par les réserves.
	
	Par exemple, supposons un taux de réserves obligatoires de 10\%. Avant un retrait :
	
\begin{center}
	\begin{tabular}{|c|c|c|c|}
	\hline 
	\multicolumn{2}{|c|}{Actif} & \multicolumn{2}{|c|}{Passif} \\ 
	\hline 
	Réserves & 20 & Dépôts & 100 \\ 
	\hline 
	Prêts & 80 & Fonds propres & 10 \\ 
	\hline 
	Titres & 10 &  &  \\ 
	\hline 
	\end{tabular} 
\end{center}
	
	Après un retrait de dépôts de 10.
	
\begin{center}
		\begin{tabular}{|c|c|c|c|}
	\hline 
	\multicolumn{2}{|c|}{Actif} & \multicolumn{2}{|c|}{Passif} \\ 
	\hline 
	Réserves & 10 & Dépôts & 90 \\ 
	\hline 
	Prêts & 80 & Fonds propres & 10 \\ 
	\hline 
	Titres & 10 &  &  \\ 
	\hline 
	\end{tabular}
	\end{center}	
	
	Afin de satisfaire le taux de réserves obligatoires, une banque a 4 options :
	
	\begin{itemize}
		\item emprunter auprès d'autres banques ou institutions financières
		
\begin{center}
\begin{tabular}{|c|c|c|c|}
\hline 
\multicolumn{2}{|c|}{Actif} & \multicolumn{2}{|c|}{Passif} \\
\hline 
Réserves & 9 & Dépôts & 90 \\ 
\hline 
Prêts & 90 & Emprunts/I.F. & 9 \\ 
\hline 
Titres & 10 & Fonds propres & 10 \\ 
\hline 
\end{tabular} 
\end{center}		
				
		\item vendre une partie de ses titres
		
		\begin{center}
\begin{tabular}{|c|c|c|c|}
\hline 
\multicolumn{2}{|c|}{Actif} & \multicolumn{2}{|c|}{Passif} \\
\hline 
Réserves & 9 & Dépôts & 90 \\ 
\hline 
Prêts & 90 &  . &   \\ 
\hline 
Titres & 1 & Fonds propres & 10 \\ 
\hline 
\end{tabular} 
\end{center}

		\item emprunter les liquidités auprès de la banque centrale
		
\begin{center}
\begin{tabular}{|c|c|c|c|}
\hline 
\multicolumn{2}{|c|}{Actif} & \multicolumn{2}{|c|}{Passif} \\
\hline 
Réserves & 9 & Dépôts & 90 \\ 
\hline 
Prêts & 90 & Emprunts/B.C. & 9 \\ 
\hline 
Titres & 10 & Fonds propres & 10 \\ 
\hline 
\end{tabular} 
\end{center}		
		
		\item Réduire le montant de ses prêts : ne pas renouveler des prêts à court terme arrivant à échéance, au risque de perdre des clients, vendre des prêts 
à d'autres banques 

\begin{center}
\begin{tabular}{|c|c|c|c|}
\hline 
\multicolumn{2}{|c|}{Actif} & \multicolumn{2}{|c|}{Passif} \\
\hline 
Réserves & 9 & Dépôts & 90 \\ 
\hline 
Prêts & 81 &  &   \\ 
\hline 
Titres & 10 & Fonds propres & 10 \\ 
\hline 
\end{tabular} 
\end{center}
	\end{itemize}
	
	Conclusion: justification de la détention de 
réserves excédentaires, qui permet: 
	\begin{itemize}
		\item d'éviter d'emprunter auprès d'autres banques 
		\item d'éviter de vendre des titres 
		\item d'éviter d'emprunter auprès de la banque centrale 
		\item d'éviter de résilier ou de vendre des titres 
	\end{itemize}
	
	NB : les banques peuvent aussi détenir plus de titres liquides (réserves secondaires) 
	
	\subsection{Gestion d'actifs}
	
	Elle doit garder un niveau de risque faible et avoir des actifs suffisamment diversifiés et rémunérateurs.
	
	Il y a 3 objectifs :
	
	\begin{enumerate}
		\item chercher des rendements les plus élevés possibles sur les prêts et titres 
		\item réduire les risques 
		\item préserver une liquidité suffisante
	\end{enumerate}
	
	4 moyens :
	
	\begin{itemize}
		\item Trouver des emprunteurs qui paieront des taux élevés, et peu susceptibles de faire défaut : examen sélectif pour réduire les points de base (centième de pourcent) d'antisélection ; on sélectionne les bons clients de manière à réduire au maximum le risque de défaut.
		\item Acheter des titres à rendement élevé et risque faible
		\item Diversification des risques : en achetant différents types d'actifs ( maturité, émetteur, \dots), éviter de trop se spécialiser sur un secteur (immobilier, énergie, \dots) 
		\item Gérer la liquidité : décider du montant des réserves excédentaires, des titres émis par l'état ( réserves secondaires). Il y a un équilibre à trouver entre avoir des liquidités et un rendement
	\end{itemize}
	
	\subsection{Gestion de passif}
	
	Acquérir des fonds à un faible coût.
	
	Avant 1960, la gestion de passif n'était pas développée : la plus grande partie des ressources étaient constituées de dépôts à vue, non rémunérés. Le marché interbancaire était peu développé 

	A partir des années 60 aux USA, les grandes banques utilisent davantage les marchés financiers, développent de nouveaux instruments (certificats de dépôts négociables). Il y a une nouvelle flexibilité dans la gestion du passif, recherche de fonds au fur et à mesure des besoins liés à la croissance de l'actif, au- delà du montant des dépôts. Les banques gèrent les 2 côtés du bilan en même 
temps, dans des comités de gestion actif-passif 
(ALM) 

Des changements importants dans la composition 
des bilans bancaires depuis 30 ans: quelques 
exemples: 
\begin{itemize}
	\item Certificats de dépôts négociables et emprunts 
interbancaires: de 2\% (1960) à 47\% ( 2008): USA 
	\item Prêts: 46\% des actifs bancaires à 61\%: USA 
	\item Part des obligations émises par les banques : 6\% à 
18\%: France 
\end{itemize}

	\subsection{Adéquation du capital}
	
	Gérer le montant des fonds propres à détenir en adéquation avec les accords de Basel, qui imposent un montant minimum de fonds propres.
	
	 La faillite est l'impossibilité de remplir les obligations de remboursement envers les déposants et autres créanciers. Une banque détient du capital pour réduire sa probabilité de devenir 
insolvable.
	
	Raisons d'avoir des capitaux propres :
	
	\begin{itemize}
	
		\item éviter la faillite. Exemple de deux banques (B pas assez capitalisé), qui se rendent compte que 5 ne valent rien.
		
		\dessinS{6}{.5}

		\item l'effet du capital sur le rendement des actionnaires. On peut définir
		
		\begin{itemize}
			\item le coefficient de rendement = return on assets (ROA) = profit net après impôts/actifs 
			\item le coefficient de rentabilité = return on equity = profit net après impôts/fonds 
propres 
			\item EM = multiplicateur de fonds propres= Actifs/fonds propres 
		\end{itemize}
		
		$$ROE = ROA \times EM $$
		
		Pour un ROA donné, moins la banque est capitalisée ( plus EM petit) et plus la rentabilité du capital est élevée (ROE élevé). Par exemple, le rendement pour les actionnaires de la banque B est meilleure que pour la banque A car B est sous-capitalisée.
	\end{itemize}
	
	Il y a un arbitrage des actionnaires entre la sécurité et rentabilité . Les avantages et inconvénients du capital bancaire :
	\begin{itemize}
		\item[+] il protège de la probabilité de faillite 
		\item[-] il diminue la rentabilité ( à ROA donné) 
	\end{itemize}
		
	 Il y a également des exigences en capital réglementaire.
	
	
	\subsection{La gestion du risque de crédit}
	
	\begin{enumerate}	
		\item Sélection et surveillance 
		\item Relation de clientèle à long terme 
		\item Engagements de financement 
		\item Collatéral et dépôt de garantie 
		\item Rationnement du crédit
	\end{enumerate}
	

\section{Gestion du risque de taux d'intérêt}

\dessin{7}
Si les taux augmentent en moyenne de 5 points ( de 10 à 15\%): 
\begin{itemize}
	\item Les revenus d'actifs augmentent de 20 x 5\% = 1M\euro
	\item Les charges d'intérêts sur dettes augmentent de 50 x 5\% = 2,5 M\euro
	\item $\rightarrow$ le profit de la banque diminue de 1,5 M\euro 
\end{itemize}


Si les taux d'intérêts diminuent en moyenne de 5 points, le profit de la banque augmente de 1,5 M.

Si une banque possède plus de dettes que d'actifs sensibles aux taux, une hausse du taux d'intérêt réduit son profit, une baisse des taux l'augmente.


Méthode des impasses comptables  : le montant des dettes sensibles aux taux d'intérêt est soustrait du montant des actifs sensibles aux taux = actifs sensibles nets. Ainsi,

\begin{itemize}
	\item Impasse ou gap = -30M\euro
	\item Impasse (-30) x variation du taux ( 5\%) = variation du profit (-1,5) 
\end{itemize}

Analyse de duration : la duration moyenne des actifs de la banque est de 3 ans (la durée de vie moyenne des revenus est de 3 ans) tandis que la duration moyenne des dettes est de 2 ans.
	
Quelle stratégie pour gérer le risque de taux ? 
\begin{itemize}
	\item Si vous anticipez une baisse des taux d’intérêt : ne rien faire pour bénéficier de la baisse attendue 
	\item Diminuer la duration des actifs, augmenter celle des dettes…mais cela peut être difficiel à court terme (reflet de la spécialisation de la banque) 
	\item Utiliser des produits dérivés pour réduire l'exposition aux taux sans modifier la structure du bilan
\end{itemize}

\section{Activités hors-bilan}

Les éléments hors-bilan sont composés d'un ensemble de comptes retraçant des engagements qui ne donnent pas lieu à des flux de trésorerie immédiats. Par exemple, un engagement de financement à l'égard de la clientèle, de garantie ou de titre.