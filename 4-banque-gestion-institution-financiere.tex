\chapter{La banque et la gestion des instituions financières}

	\section{Le bilan bancaire}
	
	[image slide 2/39]
	
	Passif :
	
	\begin{itemize}
		\item ...
		\item Opérations sur titres : titres émis sur le marché des capitaux à court terme
		\item Provisions en cas de dépréciation d'actifs. 
		
			Capitaux propres : capital de la banque + ses bénéfices non distribués + toutes les dettes subordonnées (c'est-à-dire que la dette ne sera remboursée qu'après les autres dettes)
	\end{itemize}
	
	Actif :
	
	\begin{itemize}
		\item Caisse, banques centrales : liquidité en dépôt auprès de la banque centrale
		\item Prêts aux établissements de crédit : argent de la banque en dépôt auprès d'autres banques/établissement de crédits
		\item ...
		\item Opérations sur titres : titres détenus par la banque soit pour son propre compte soit pour compte de la clientèle
		\item Valeurs immobilisées : immobilisations corporelles et incorporelles de la banque
		Prêts subordonnées : participation dans des entreprises liées
	\end{itemize}
	
	\section{L'exploitation bancaire}
	
	Les mécanismes de base d'une fonctionnement d'une banque :
	
	\begin{itemize}
		\item Transformation d'actifs : transformation des dépôts de sa clientèle pour accorder des prêts ; elle a un rôle d'intermédiation
		\item Transformation d'échéances : par exemple emprunter à court terme sur les marchés pour prêter à long terme
		\item La banque vend des services qu'elle facture à ses clients : par exemple traitement des virements, relevés bancaires, conseils en placement
	\end{itemize}
	
	Un dépôt dans une banque entraine une réserve supplémentaire.
	
	\section{Principes de gestion de bilan}
	
	\subsection{Gestion de liquidité}
	
	Une banque doit s'assurer qu'elle dispose d'assez de réserve/de liquidité pour rembourser les déposant qui retirent de l'argent de leur compte.
	
	Taux de 10\% $\Leftrightarrow$ les dépôts des clients doivent être assurés à hauteur de 10\% par les réserves.
	
	...
	
	Une banque a 4 options :
	
	\begin{itemize}
		\item emprunter auprès d'autres banques ou institutions financières
		\item vendre une partie de ses titres
		\item emprunter les liquidités auprès de la banque centrale
		\item réduire le montant des prêts accordés à ses clients : ..., vendre des prêts (titrisation)
	\end{itemize}
	
	Conclusion: justification de la détention de 
réserves excédentaires, qui permet: 
	\begin{itemize}
		\item d'éviter d'emprunter auprès d'autres banques 
		\item d'éviter de vendre des titres 
		\item d'éviter d'emprunter auprès de la banque centrale 
		\item d'éviter de résilier ou de vendre des titres 
	\end{itemize}
	
	NB : les banques peuvent aussi détenir plus de titres liquides (réserves secondaires) 
	
	\subsection{Gestion d'actif}
	
	Elle doit garder un niveau de risque faible et avoir des actifs suffisamment diversifiés et rémunérateur.
	
	Il y a 3 objectifs :
	
	\begin{enumerate}
		\item chercher des rendements les plus élevés possibles sur les prêts et titres 
		\item réduire les risques 
		\item préserver une liquidité suffisante
	\end{enumerate}
	
	4 moyens :
	
	\begin{itemize}
		\item Trouver des emprunteurs qui paieront des taux élevés, et peu susceptibles 
de faire défaut : examen sélectif pour réduire les points de base (centième de pourcent) d'antisélection ; on sélectionne les bons clients de manière à réduire au maximum le risque de défaut.
		\item Acheter des titres à rendement élevé et risque faible
		\item Diversification des risques : en achetant différents types d'actifs ( maturité, 
émetteur, \dots), éviter de trop se spécialiser sur un secteur (immobilier, 
énergie, \dots) 
		\item Gérer la liquidité : décider du montant des réserves excédentaires, des titres 
émis par l’état ( réserves secondaires). Il y a un équilibre à trouver entre avoir des liquidités et un rendement
	\end{itemize}
	
	\subsection{Gestion de passif}
	
	Acquérir des fonds à un faible coût.
	
	Avant 1960, la gestion de passif n’était pas développée: 
	La plus grande partie des ressources étaient constituées de dépôts à vue, 
non rémunérés 
 Le marché interbancaire était peu développé 

A partir des années 60 aux USA, les grandes banques utilisent davantage 
les marchés financiers, développent de nouveaux instruments (certificats de 
dépôts négociables) 
* Nouvelle flexibilité dans la gestion du passif, recherche de fonds au fur et à 
mesure des besoins liés à la croissance de l’actif, au- delà du montant des 
dépôts 

Les banques gèrent les 2 côtés du bilan en même 
temps , dans des comités de gestion actif-passif 
(ALM) 
* Des changements importants dans la composition 
des bilans bancaires depuis 30 ans: quelques 
exemples: 
– Certificats de dépôts négociables et emprunts 
interbancaires: de 2% (1960) à 47% ( 2008): USA 
– Prêts: 46\% des actifs bancaires à 61%: USA 
– Part des obligations émises par les banques : 6\% à 
18\%: France 

	\subsection{Adéquation du capital}
	
	Gérer le montant des fonds propres à détenir en adéquation avec les accords de Bâle, qui impose un montant minimum de fonds propres.
	
	Raisons d'avoir des capitaux propres :
	
	\begin{itemize}
	
		\item éviter la faillite

	
	Comment le capital protège-t-il du risque de faillite? 
* Faillite= impossibilité de remplir les obligations de remboursement envers 
déposants et autres créanciers 
* Une banque détient du capital pour réduire sa probabilité de devenir 
insolvable 

		\item l'effet du capital sur le rendement des actionnaires
		
		Coefficient de rendement= return on assets (RAO) = profit net après 
impôts/actifs 
		Coefficient de rentabilité= return on equity = profit net après impôts/fonds 
propres 

		ROE = ROA x EM 
		
		Avec EM = multiplicateur de fonds propres= Actifs/fonds propres 
		Pour un ROA donné, moins la banque est capitalisée ( plus EM petit), plus 
la rentabilité du capital est élevée ( ROE élevé) 

		Par exemple, le rendement pour les actionnaires de la banque B est meilleure que pour la banque A car B est sous-capitalisée.
		
		
		L’arbitrage des actionnaires entre sécurité et rentabilité 
– Les avantages et inconvénients du capital bancaire: 
* (+) il protège de la probabilité de faillite 
* (-) il diminue la rentabilité ( à ROA donné) 
* Les exigences en capital réglementaire 
	\end{itemize}
	
	
	\subsection{La gestion du risque de crédit}
	
	\begin{enumerate}	
		\item Sélection et surveillance 
		\item Relation de clientèle à long terme 
		\item Engagements de financement 
		\item Collatéral et dépôt de garantie 
		\item Rationnement du crédit
	\end{enumerate}
	

\section{Gestion du risque de taux d'intérêt}

Si les taux augmentent en moyenne de 5 points ( de 10 à 15%): 
* Les revenus d’actifs augmentent de 20 x 5% = 1M\euro
* Les charges d’intérêts sur dettes augmentent de 50 x 5% = 2,5 M\euro
* - le profit de la banque diminue de 1,5 M\euro 

Si les taux d’intérêts diminuent en moyenne de 5 points: 
* Le profit de la banque augmente de 1,5 M\euro
* Si une banque possède plus de dettes que d’actifs sensibles aux taux, une 
hausse du taux d’intérêt réduit son profit, une baisse des taux l’augmente.

...

\subsection{Analyse des impasses et des durations}

Analyse des impasses et des durations
* Méthode des impasses comptables 
– Le montant des dettes sensibles aux taux d’intérêt est soustrait du 
montant des actifs sensibles aux taux = actifs sensibles nets 
– Impasse ou gap = -30M\euro
– Impasse ( -30) x variation du taux ( 5\%) = variation du profit ( -1,5) 
	

\section{Activités hors-bilan}

Les éléments hors-bilan sont composés d'un ensemble de comptes retraçant des engagements qui ne donnent pas lieu à des flux de trésorerie immédiats. Par exemple, un engagement de financement à l'égard de la clientèle, de garantie ou de titre.