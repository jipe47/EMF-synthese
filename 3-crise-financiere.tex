% 11/03/2013
\chapter{Les crises financières}

\section{Les facteurs de la crise financière}	
	
	\subsection{L'effet du marché des actifs sur les bilans}
	
		\subsubsection{Chute du marché boursier}
	
		Une forte baisse des cours boursiers peut gravement détériorer le bilan des entreprises qui empruntent. En effet, une baisse des cours diminue la capitalisation boursière (nombre d'actions émises par une entreprise multiplié par le cours de bourse). Plus cette capitalisation diminue, plus les actifs nets (notamment le fond propre (capital dont on déduit les frais d'établissement)) diminuent et donc plus l'entreprise aura des difficultés à obtenir des crédits.
	
		Par la suite, cette détérioration peut accroître les problèmes d'anti-sélection et de risque moral sur les marchés financiers et provoquer une crise financière. 
	
		Les banques ont normalement un rôle d'intermédiation financière : elles accumulent des liquidités et les prêtent aux entreprises et ménages pour faire tourner l'économie. Afin de réduire les risques, des banques peuvent diminuer leur octroi de crédit et en refuser, sauf à des entreprises très solides. En diminuant ces offres de crédit, elles ne jouent plus leur rôle d'acteur dans l'économie.
	
		Chute de la bourse $\rightarrow$ banques prêtent moins $\rightarrow$ les entreprises investissent moins.

		Risque moral : risque que le bilan d'une entreprise se détériore après qu'elle ait obtenu un crédit.
	
		Accords de Baltroy : demande aux banques de quantifier les risques. Plus il y en a, plus du capital doit être mobilisé. Ainsi, plus une banque accorde des crédits à des entreprises de mauvaise qualité, plus les risques sont élevés et plus les augmentations de capital seront nécessaire.
		
		\subsubsection{Baisse non anticipée du niveau des prix}
		
		Dans les économies où l'inflation est restée modérée, les contrats de dette sont souvent à maturité éloignée et à taux d'intérêt fixe. Une baisse non anticipée du niveau général des prix augmente la valeur en termes réels des dettes de l'entreprise, c'est-à-dire qu'elle accroît la charge de la dette. 
		
		La charge de la dette est plus importante si le taux d'intérêt est fixe. Par exemple, une entreprise peut vendre un bien puis en diminuer le prix : avec un taux fixe, la charge est plus importante après la baisse. Avec un taux variable, on n'a qu'un pourcentage des revenus ou autre.
		
		La dette est plus importante, donc le bilan des entreprises se détériore, ce qui les privent de crédit, donc de capacités d'investir.
		
		\subsubsection{Baisse non anticipée de la valeur de la monnaie nationale}
		
		Si les contrats de dette sont libellés en monnaie étrangère, toute baisse non anticipée de la valeur de la monnaie nationale augmente la charge des entreprises et conduit à une baisse de l'activité économique. Si les contrats de dette sont libellés en monnaie étrangère, toute baisse non anticipée de la valeur de la monnaie nationale augmente la charge des entreprises et conduit à une baisse de l'activité économique.
		
		Pour une entreprise qui a une dette en devise étrangère, si la devise nationale baisse, la dette va augmenter.
		
		\subsubsection{Dépréciation des actifs}
		
		La dépréciation de la valeur des actifs financiers enregistrés au bilan des institutions financières peut provoquer une dégradation des bilans et une contraction du crédit.
		
	\subsection{La détérioration des bilans des institutions 
financières}

	Si les institutions financières souffrent d'une détérioration de leurs bilans, elles subissent en conséquence une contraction de leur capital: les ressources bancaires seront moindres et le crédit bancaire diminuera.
	
	La contraction du crédit conduit alors à une baisse de la dépense d'investissement qui ralentit l'activité économique.
	
	\subsection{Les défaillances bancaires}

	\begin{itemize}
		\item Effet de contagion 
		\item Panique bancaire 
		\item La faillite d'un grand nombre de banques sur une brève période de temps signale que la production d'information sur les marchés financiers est dégradée: par conséquent, l'intermédiation financière par le secteur bancaire risque de s'interrompre 
	\end{itemize}
	
	La chute du crédit diminue l'offre de fonds disponibles pour les 
emprunteurs. Il en résulte une baisse des prêts pour financer les investissements productifs et une contraction encore plus grave de l'activité 

	\subsection{La montée de l'incertitude}
	
	Une forte hausse de l'incertitude sur un marché financier accentue la difficulté pour les prêteurs de distinguer les bons et les mauvais risques de crédit. Par exemple, les éclatements de bulles (immobilières) font chuter les prix et augmenter l'incertitude.
	
	Conséquence : baisse du crédit, de l'investissement, détérioration de l'activité économique.
	
	\subsection{La hausse des taux d'intérêt}
	
Augmentation des charges de dettes pour les entreprises. Il y a dès lors une diminution de l'autofinancement et un risque de diminution des investissements et une contraction de l'activité économique.


	Une augmentation imprévue du taux d'intérêt peut faire éclater une bulle spéculative.

	\subsection{Le déséquilibre budgétaire de l'état}
	
	\begin{itemize}
		\item Le déséquilibre budgétaire peut faire craindre un défaut de l'état sur le paiement de sa dette 
		\item L'état peut rencontrer des difficultés à placer les titres de la dette publique auprès des investisseurs. 
		\item Les bilans des institutions financières peuvent se détériorer. 
		\item Diminution de l'octroi de crédit, contraction de l'activité économique
		\item Crise de change
	\end{itemize}


\section{La dynamique des crises financières}

	\dessinS{schema_crise}{.3}
	
	\subsection{Phase 1 : déclenchement d'une crise financière}
	
	Une mauvaise maîtrise du processus de libéralisation/innovation financière 
\begin{itemize}
	\item Emballement du crédit : les banques ont du s'endetter sur le marché à travers la titrisation, et donc l'augmentation du risque.
	\item Levier financier 
	\item Mauvaise gestion des risques 
	\item Contraction du crédit 
\end{itemize}

	On libéralise un marché pour faciliter la circulation des capitaux. Maintenant, on tente de régulier le marché afin de diminuer les risques.

	\subsubsection{La bulle et la chute du prix des actifs}
	
	Bulle spéculative : prix déconnecté de la réalité et des déterminants économiques.
	
	\begin{itemize}
		\item Prix déconnecté de leurs déterminants économiques fondamentaux 
		\item Exubérance irrationnelle 
		\item Emballement du crédit 
		\item Quand la bulle éclate, 
		\begin{itemize}
			\item les emprunteurs vont décroître leur capacité d'emprunter, contraction du crédit et de la dépense 
			\item Détérioration du bilan des institutions financières, diminution du levier
		\end{itemize}
	\end{itemize}
	
	\subsubsection{Les pics de taux d'intérêt}
	
	la hausse des taux d'intérêt fait baisser les flux de revenus des 
ménages et des entreprises, réduit le nombre de bons risques en 
recherche d'emprunts, ce qui augmente l'anti-sélection et le risque 
moral et fait baisser l'activité économique

	\subsubsection{Montée de l'incertitude}
	Souvent après le début d'une récession, soit après un krach boursier. Un trait commun des crises financières est la défaillance des institutions financières
	

	\subsection{Phase 2 : crise bancaire}
	
	Avec la détérioration de l'activité économique, l'augmentation des 
défaillances d'entreprises et de ménages et l'incertitude sur la solidité des banques, les déposants commencent à retirer leurs fonds des banques ( crise ou panique bancaire).

	La diminution du nombre de banques en activité fait perdre leurs ressources informationnelles, l'anti-sélection et le risque moral s'aggravent sur les marchés du crédit : spirale descendante


	Moyens de surmonter la crise financière :
	\begin{itemize}
		\item Régulation des marchés
		\item Recapitalisation des banques pour réduire l'incertitude sur les marchés
	\end{itemize}

	\subsection{Phase 3 : déflation par la dette}
	
	Quand la déflation par la dette s'établit, les problèmes d'anti-sélection et de risque moral s'aggravent davantage, ce qui provoque une dépression de longue durée des prêts, de la dépense d'investissement et de l'activité économique globale. 


\section{La crise financière de 2007-2010}

	\subsection{La détérioration des prêts au logement}
	
	L'élément déclencheur est la baisse du taux d'intérêt directeur, qui a fait décoller les marchés, car on pouvait s'endetter très fortement.
	
	GSE : agences bénéficiant de la garantie de l'état. Elles ont racheté les crédits hypothécaire accordées par les banques et les ont titrisées. Ces titres ont été mis sur le marché avec la garantie de l'état américain. Le but était de réduire les risques au niveau des banques et transférer les risques au niveau des acteurs de la bourse. Elles ont créé un marché attractif.
	
	RMBS : filiales des GSE, institutions qui ont comme compte propre des actifs titrisés.
	
	\subsection{L'assouplissement des critères dans l'attribution des prêts}
	
	A partir de 2000, les prêteurs ont assoupli les critères liés à l'attribution des crédit immobiliers, ce qui a entraîné les prix à la hausse. Ainsi, l'exigence de capital passait de 4 à 2\% lorsque les titres hypothécaires sont titrisés, car le risque sur le fond propre est moindre.

	Cette double augmentation, crédits attribués et prix des maisons, était très intéressante à la fois pour les vendeurs de prêts (ou brokers) et les prêteurs. 

	Afin de soutenir le marché, les intermédiaires ont rendu les critères 
d'attribution des crédits de moins en moins contraignants. On en est arrivé à des prêts immobiliers à taux ajustable.

	Depuis 1990, le gouvernement souhaitait promouvoir l'accès à la propriété en incitant les prêteurs à attribuer davantage de crédits aux familles à revenus modestes.

	Aux Etats-Unis, toute banque était obligée de donner des crédits hypothécaires, même à des personnes insolvables, afin de faciliter l'accès à la propriété. La titrisation permettait d'exporter ces risques.
	
	\subsection{La titrisation}
	
	Les banques qui accordent des prêts les financent traditionnellement avec des dépôts de clients. 

	La titrisation consiste à regrouper des portefeuilles de prêts et à vendre les promesses de cash-flows associées (intérêt et principal) sous forme d'actifs financiers. Elle permet aussi d'accroître les volumes de prêts consentis plus rapidement que ne le permettrait l'évolution des dépôts. 
	
	\dessinS{14}{.5}
	
	Les banques se débarrassent donc des prêts au SPV (Special Purpose Vehicule), un véhicule spécial de titrisation. Ce dernier va alors émettre sur le marché différents types de titre à destination des investisseurs (des obligations par exemple). Ces différents types d'obligation sont les tranches senior, mezzanine et equity, au sein d'un ABS (Asset-Backed Security).
	
	Le cash-flow des actifs est donc réparti selon un principe de cascade à travers les différentes tranches. On rembourse d'abord les actifs et les intérêts de la tranche senior, puis de la tranche mezzanine et enfin la tranche equity. 
	
	La tranche equity est plus risquée mais rapporte un meilleur rendement. La tranche mezzanine peut être de nouveau titrisée, et donc de nouveau avoir une tranche senior, mezzanine et equity.
	
	\dessinS{15}{.5}
	
	Equity est la première tranche à supporter les pertes sur le portefeuille. On remonte ensuite dans la cascade pour rembourser les pertes. On remonte ensuite dans la tranche equity  de l'ABS (ou la tranche mezzanine) et ainsi de suite.
		
	
	AAA signifie que le risque d'insolvabilité est très faible. On peut remarquer que la tranche senior supporte des pertes à partir de 10\% de pertes sur le portefeuille.
	
	\dessinS{16}{.35}
	
	Les banques achètent ces titres entre elles, ainsi que des compagnies d'assurance, des fonds d'investissement, etc.
	
	Ainsi, les SPE, SIV et VIE sont des véhicules de titrisation. Les titres titrisés AAA ont besoin de moins de fonds propres.
		
	\dessinS{17}{.55}
	
	Le collateral est le fait de mettre des titres en garantie lors de l'octroi de prêts. CDS signifie Credit Depot Swap.
	
	\subsection{L'éclatement de la bulle immobilière}
	
	Il y a eu une augmentation des défaillances sur les prêts hypothécaires, ce qui a entraîné des saisies immobilières et une dépréciation des actifs financiers liés aux crédits immobiliers.
	
	Le marché interbancaire des emprunts de liquidités contre titres (repurchase agreements) et celui des billets de trésorerie adossés à des actifs (Asset- backed commercial papers) se sont grippés étant donné l'incertitude sur la valeur des titres utilisés comme collatéral. 

	Il y a eu de fortes tensions sur les taux d'intérêt et au final les banques et autres institutions financières ont brutalement cessé de se prêter les unes aux autres.
	
	\subsection{Les défaillances et la panique financière de 2008}
	
	Il y a eu une détérioration des bilans bancaires. Pour se refinancer, les SIV émettent du papier commercial ABCP à court terme. Cependant, les SIV et les conduits ont été incapables de continuer à placer leur commercial paper.

	Du coup, l'impossibilité de se refinancer a tari le paiement des coupons. Le marché des ABCP et des repruchase agreements sont les premières sources de financement du système bancaire : assèchement dès le premier semestre 2007.
	
	
	Si on récapitule les voies par lesquelles les bilans des banques et institutions financières ont pu être exposés à la dépréciation des actifs titrisés, on trouve 5 principaux canaux :

	\begin{enumerate}
		\item La banque a produit et vendu des titres adossés à des créances et elle a gardé pour compte propre des tranches superseniors notées AAA afin de structurer le total de ses actifs pondérés des risques (RWA) et de minimiser sa charge en capital réglementaire.
		\item La banque a conservé un lien avec les investisseurs auxquels elle a vendu des titres structurés à travers un véhicule de titrisation. Le véhicule de titrisation ne parvient pas à isoler complètement l'originator de la faillite des investisseurs. 
		\item La banque a acheté des produits structurés à d’autres institutions 
financières à des fins de négoce (trading) ou d’investissement soit pour compte de la clientèle soit pour compte propre. La banque ne sait plus à quel prix elle doit valoriser les titres dépréciés dans ses comptes; ses contreparties peuvent refuser ces titres en collatéral des emprunts de la banque (rep's, ABCP) 
		\item La banque a émis un CDS sur un titre complexe ou elle est partenaire d'un pool de CDS. Si le CDS est activé par un événement de crédit, la banque doit fournir du collatéral supplémentaire ou de meilleure qualité et/ou payer en espèces le bénéficiaire de la protection. 
		\item La banque a pris des positions courtes sur un produit structuré, c'est-à-dire qu'elle l'a vendu à découvert: elle emprunte le titre qu'elle doit livrer à terme en espérant le racheter moins cher.
	\end{enumerate}
