% Conférence 25/03/2013

\section{Accord de Basel}

\subsection{Basel I}
Basel I : nécessité d'un minimum de fonds propres, afin de pouvoir supporter un retrait massif d'argent par des clients qui soldent leur compte. Le ratio (cook) est d'environ 8\%. C'est un accord très général et standardisé, il touche autant des petites banques que des géants.

Le problème de Basel I est le critère de solvabilité (trop compliqué pour des petites banques, trop simples pour des plus grosses). De plus, il manque la gestion du risque opérationel (risque d'une panne du service informatique par exemple).

\subsection{Basel II}

Dans Basel II, Basel I est réutilisé pour les banques dans des pays moins développés car il est suffisant.

Pondération sur trois niveaux :

\begin{itemize}
	\item basic : calcul semblable à Basel I, mais plus précis.
	
	Grosse erreur de l'époque : utilisation de la notation d'un pays (donné par une agence de notation) pour mesurer son risque.
	
	\item foundation : idem mais gestion d'un risque personnel.
	\item advanced : les banques gèrent comme elles veulent les risques, mais ne peuvent pas faire n'importe quoi (le modèle doit être approuvé par un régulateur).
\end{itemize}

Disclosure : information qui doit être légalement publiée. Cela oblige la banque à prévoir des marchés qui dévient ou des situations critiques.

Basel II n'est qu'un agreement, pas un accord : des pays l'ont signé mais cela n'a pas été appliqué partout (par exemple aux Etats-Unis).

Problème de timing, l'accord est arrivé en Europe au début de la crise financière, elle aurait été minimisée.

Autre problème : le critère de solvabilité était désuet : des banques avec un bon seuil de solvabilité ont fait faillite par manque de liquidités.

\subsection{Basel III}

Transition entre Basel II et III initiée par la crise, beaucoup plus rapide qu'entre Basel I et II.

