\chapter{La demande de monnaie}

\section{La théorie quantitative de la monnaie (TGM)}

	\subsection{L'équation des échanges de Fisher (1911)}
	
	$$\underbrace{M}_{\text{quantité de monnaie}} \times \underbrace{V}_{\text{vitesse de circulation de monnaie}} = \underbrace{P}_{\text{prix moyen des transactions}} \times \underbrace{T}_{\text{nombre de transactions sur les marchés}}$$
	
	Au cours d'une année, $P \times T$ est la valeur en euros de ce qui a été échangé sur les marchés.
	
	$V$ est le nombre de transactions effectuées par unité monétaire. C'est le nombre de fois qu'un billet de banque change de main au cours d'une année.
	
	Exemple : supposons que 100 cannettes de bière soient vendues à 1 euro la pièce au cours d'une année.
	
	$\rightarrow T = 100\text{ / an, } P = 1 \text{ euro par cannette}$
	$\rightarrow P \times T = 100 = \text{ nombre total d'euros échangés}$
	
	Supposons que la quantité de monnaie en circulation est de $20$ euros. La vitesse de circulation $V = \frac{P \times T}{M} = \frac{100}{20} = 5$. Chaque euro change donc 5 fois de main sur un an.
	
	Si vitesse est très élevée, il y a de l'inflation car on cherche à se débarrasser de la monnaie.
	
	Si la masse monétaire passe de 20 à 40 euros, il y a 3 possibilités :
	
	\begin{enumerate}
		\item réduction de la vitesse, ou
		\item les prix vont augmenter (à 2 euro), donc la vitesse ne va pas bouger, ou
		\item la vitesse et les prix ne changent pas mais la production augmente (elle passe à 200/an).
	\end{enumerate}
	
	Si on suppose que les firmes sont au maximum des capacités de production, la 3ème hypothèse tombe. Avec la définition classique et néoclassique, on suppose que la vitesse est stable, donc la vitesse va augmenter.
	
	On va remplacer le nombre de transaction par la production ($Y$ est prix comme une approximation de $T$). L'équation devient
	
	$$M \times V = \underbrace{P \times \underbrace{T}_{\text{PIB réel}}}_{\text{PIB nominal}}$$
	
	$PY$ le PIB nominal est distribué sous forme de revenu, en salaire et dividende de capital. Donc $V$ est la vitesse de circulation de la monnaie par euro de revenu.
	
	Cette équation des échanges, sans hypothèses sur les variables, est une identité : on ne sait pas ce qu'il va se passer si une des variables change.
	
	Dans la théorie quantitative, si $M$ augmente, $P$ augmente.