\chapter{La demande de monnaie}

\section{La théorie quantitative de la monnaie (TQM)}

	\subsection{L'équation des échanges de Fisher (1911)}
	
	$$\underbrace{M}_{\text{quantité de monnaie}} \times \underbrace{V}_{\text{vitesse de circulation de monnaie}} = \underbrace{P}_{\text{prix moyen des transactions}} \times \underbrace{T}_{\text{nombre de transactions sur les marchés}}$$
	
	Au cours d'une année, $P \times T$ est la valeur en euros de ce qui a été échangé sur les marchés.
	
	$V$ est le nombre de transactions effectuées par unité monétaire. C'est le nombre de fois qu'un billet de banque change de main au cours d'une année.
	
	Exemple : supposons que 100 cannettes de bière soient vendues à 1 euro la pièce au cours d'une année.
	
	$\rightarrow T = 100\text{ / an, } P = 1 \text{ euro par cannette}$
	$\rightarrow P \times T = 100 = \text{ nombre total d'euros échangés}$
	
	Supposons que la quantité de monnaie en circulation est de $20$ euros. La vitesse de circulation $V = \frac{P \times T}{M} = \frac{100}{20} = 5$. Chaque euro change donc 5 fois de main sur un an.
	
	Si vitesse est très élevée, il y a de l'inflation car on cherche à se débarrasser de la monnaie.
	
	Si la masse monétaire passe de 20 à 40 euros, il y a 3 possibilités :
	
	\begin{enumerate}
		\item réduction de la vitesse, ou
		\item les prix vont augmenter (à 2 euro), donc la vitesse ne va pas bouger, ou
		\item la vitesse et les prix ne changent pas mais la production augmente (elle passe à 200/an).
	\end{enumerate}
	
	Si on suppose que les firmes sont au maximum des capacités de production, la 3ème hypothèse tombe. Avec la définition classique et néoclassique, on suppose que la vitesse est stable, donc la vitesse va augmenter.
	
	On va remplacer le nombre de transaction par la production ($Y$ est prix comme une approximation de $T$). L'équation devient
	
	$$M \times V = \underbrace{P \times \underbrace{T}_{\text{PIB réel}}}_{\text{PIB nominal}}$$
	
	$PY$ le PIB nominal est distribué sous forme de revenu, en salaire et dividende de capital. Donc $V$ est la vitesse de circulation de la monnaie par euro de revenu.
	
	Cette équation des échanges, sans hypothèses sur les variables, est une identité : on ne sait pas ce qu'il va se passer si une des variables change.
	
	Dans la théorie quantitative, si $M$ augmente, $P$ augmente.
	
	% Cours 25/2/2013
	
	\subsubsection{La TQM comme fonction de demande de monnaie}
	
	Par Marshall et Pigou.
	
	Ces économistes font clairement une différence entre
	
	\begin{itemize}
		\item la demande de monnaie, que les agents économiques expriment pour des achats de bien et services.
		
		$$M^d =k.PY$$
		
		$M^d$ est la demande d'encaisse monétaire et $k$ est la part de la valeur nominale du PIB que les agents économiques souhaitent détenir sous forme de monnaie.
		
		\item l'offre de monnaie $M$: exogène, fixée par les autorités monétaires.
	\end{itemize}
	
	\paragraph{Equilibre du marché de la monnaie} Condition d'équilibre : offre ($M$) = demande ($M^d$). On a donc $M = k.PY$.
	
	Cette théorie de demande de monnaie permet d'expliquer le niveau des prix observé dans l'économie.
	
	Ici, $Y$ est la production de viens et services ; les facteurs de production (quantité et prix) et la technologie. $Y$ est déterminé en dehors du modèle.
	
	$M$ : l'offre de monnaie est exogène. Si les autorités monétaires augmentent $M$, l'équilibre est rompu car $M > M^d$. Supposons que $k$ (variable comportementale) est stable, ce sont alors les prix $P$ qui vont rétablir l'équilibre.
	
	En conclusion, la monnaie est neutre à long terme, elle ne permet pas de modifier la production de bien et de services. Le seul effet à long terme est la variation des prix.
	
	Si on soustrait $MV = PY$ et $M = kPY$, on a $v = \frac{1}{k}$. Si $k$ est élevé, une grande part du revenu est détenue en monnaie. Si $k$ est faible, la  vitesse de circulation devra être élevée pour permettre l'ensemble des transactions.
	
	\subsection{La dichotomie classique et la neutralité à long terme de la monnaie}
	
	[Dessin 1]
	
	$L$ : quantité de travail
	$\frac{W}{P}$ : le salaire réel
	$O_L$ : offre de travail (émise par les salariés)
	$D_L$ : demande de travail (émise par les entreprises)
	
	Si $\frac{W}{P}$ augmente, $O_L$ augmente et $D_L$ diminue, donc l'offre de travail est une fonction croissante du salaire réel, alors que la demande de travail est une fonction décroissante.
	
	L'équilibre du travail (offre $O_L$ = demande $D_L$) définit le salaire réel d'équilibre $\frac{W_0}{P_0}$ et la quantité d'équilibre $L_0$.
	
	La frontière technologique est représentée par une fonction de production $Y = A F(K, L)$, avec $Y$ la quantité de biens et services, $F$ la fonction, $K$ le capital, $L$ le travail et $A$ la productivité totale des facteurs (niveau technologique de notre économie). $Y_0$ est le niveau optimal de production nationale à la fois pour les entreprises et pour les consommateurs.                                                                                                                                                                            
	
	$OA$ : offre agrégée, verticale car elle ne dépend pas des prix mais bien de facteurs de production.
	
	$DA_0(M_0)$ est la demande agrégée :
	
	\begin{itemize}
		\item dépend de la masse monétaire en circulation $M_0$
		\item dépend aussi des prix
		\item[$\rightarrow$] la demande augmente quand les prix diminuent
	\end{itemize}
	
	Equilibre du marché des biens : $OA = DA$. Le niveau des prix des biens et services est déterminé par l'équilibre du marché des biens et service.
	
	Supposons que les autorités monétaires augmentent la masse monétaire de $M_0$ à $M_1$ (en vert), donc les agents économiques vont vouloir détenir plus de monnaie et échangeront cette monnaie contre des biens et services. Comme l'offre de biens et services est déterminée par des facteurs réels (donc restera intacte), cette hausse de la demande agrégée ne conduit qu'à la hausse des prix. Si les prix augmentent, le salaire réel va diminuer, l'offre de travail diminue et la demande du travail, on a un déséquilibre sur le marché du travail, un écart entre $L_E$ et $L_S$, qui exercice une pression sur le salaire nominal. Ce déséquilibre favorise l'offre, les entreprises vont augmenter les salaires minimales pour retrouver l'équilibre ($W_1$).
	
	En conclusion, la hausse de quantité de monnaie n'a eu aucune conséquence sur la production de biens et service, sur le salaire réel et sur les niveaux d'emploi : c'est la neutralité de la monnaie à long terme.
	
	
	Prenons l'hypothèse que $V$ est stable. Si les autorités monétaires augmentent la masse monétaire entre $t$ et $t + 1$.
	
	En $t$, $M_t V_t = P_t Y_t$, en $t + 1$, $M_{t + 1} V_{t + 1} = P_{t + 1} Y_{t + 1}$. On a $1 + \text{taux} = \frac{M_{t + 1} V_{t + 1}}{M_t V_t}$. Si on prend le logarithme :
	
	$$(\ln M_{t + 1} + \ln V_{t + 1}) - (\ln M_t + \ln V_t) = (\ln P_{t + 1} + \ln Y_{t + 1}) - (\ln P_t + \ln Y_t)$$
	$$\Leftrightarrow \underbrace{\ln M_{t + 1} - \ln M_t}_{\text{taux de variation de la masse monétaire}} + \underbrace{\ln V_{t + 1} - \ln V_t}_{0\text{ par hypothèse}} = \underbrace{\ln P_{t + 1} -\ln P_t}_{\text{taux d'inflation}} + \underbrace{\ln Y_{t + 1} - \ln T_t}_{\text{taux de croissance du PIB réel}}$$
	
\section{La théorie keynésienne de la demande de monnaie}

	