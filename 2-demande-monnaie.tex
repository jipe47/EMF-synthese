\chapter{La demande de monnaie}

\section{La théorie quantitative de la monnaie (TQM)}

	Dans la théorie classique et néo-classique, le principe de base est que la monnaie est neutre à long terme. Il y a également une dichotomie entre les variables réelles et les variables nominales (déterminées par l'offre et la demande et par les prix relatifs). La TQM de Fisher va expliquer la détermination des variables nominales.
	
	\subsection{L'équation des échanges de Fisher (1911)}
	
	$$MV = PT$$
	
	\begin{itemize}
		\item $T$ : nombre de transactions sur les marchés  ; au cours d'une année, c'est le nombre de fois que l'on échange de la monnaie contre des biens et des services.
		\item $M$ : quantité de monnaie en circulation dans l'économie.
		\item $V$ : vitesse de circulation de monnaie ; nombre de transactions effectuées par unité monétaire, soit le nombre de fois qu'un billet change de main au cours d'une période. Lors d'inflation, la monnaie circule plus vite car on cherche à s'en débarrasser.
		\item $P$ : prix moyen des transaction ; nombre moyen d'unités monétaires (UM) échangées par transaction.
		
		$\rightarrow$ $PT$ est la valeur en euros de ce qui a été échangé sur les marchés au cours d'une année, c'est la valeur nominale des transactions.
	\end{itemize}
	
	Exemple : supposons que 100 cannettes de bière soient vendues à 1 euro la pièce au cours d'une année.
	
	$\rightarrow T = 100\text{ / an, } P = 1 \text{ euro par cannette}$
	$\rightarrow P \times T = 100 = \text{ nombre total d'euros échangés}$
	
	Supposons que la quantité de monnaie en circulation est de $20$ euros. La vitesse de circulation $V = \frac{P \times T}{M} = \frac{100}{20} = 5$. Chaque euro change donc 5 fois de main sur un an.
	
	\paragraph{Augmentation de la masse monétaire}
	
	Si la masse monétaire passe de 20 à 40 euros, il y a 3 possibilités :
	
	\begin{enumerate}
		\item réduction de la vitesse, ou
		\item les prix vont augmenter (à 2 euro), donc la vitesse ne va pas bouger, ou
		\item la vitesse et les prix ne changent pas mais la production augmente (elle passe à 200/an).
	\end{enumerate}
	
	Si on suppose que les firmes sont au maximum des capacités de production, la 3ème hypothèse tombe. Avec la définition classique et néoclassique, on suppose que la vitesse est stable, donc le prix va augmenter.
	
	\paragraph{Intégration de la production}
		
	On va remplacer le nombre de transaction par la production. Comme il est impossible de calculer toutes les transactions, $T$ est approximé par $Y$ le PIB réel. L'équation devient
	
	$$M \times V = \underbrace{P \times \underbrace{Y}_{\text{PIB réel}}}_{\text{PIB nominal}}$$
	
	\begin{itemize}
		\item $PY$ : PIB nominal distribué sous forme de revenu, en salaire et dividende de capital.
		\item $V$ : vitesse de circulation de la monnaie par euro de revenu.
	\end{itemize}
	
	Cette équation des échanges, sans hypothèses sur les variables, est une identité : on ne sait pas ce qu'il va se passer si une des variables change. Par exemple, dans la théorie quantitative, si $M$ augmente, $P$ augmente.
		
	Nb : $Y \neq T$ car la production n'est pas proportionnelle aux transactions (il faut considérer les ventes de seconde main, ce qui n'est pas connu), mais $Y$ et $T$ varient proportionnellement.
	
	% Cours 25/2/2013
	
	\subsubsection{La TQM comme fonction de demande de monnaie}
	
	Par Marshall et Pigou, qui font clairement une différence entre l'offre ($M$) et la demande ($M^d$) de monnaie :
	
	\begin{itemize}
		\item la demande de monnaie, que les agents économiques expriment pour des achats de biens et services.
		
		$$M^d =k.PY$$
		
		$M^d$ est la demande d'encaisse monétaire et $k$ est la part de la valeur nominale du PIB que les agents économiques souhaitent détenir sous forme de monnaie.
		
		\item l'offre de monnaie $M$ : exogène, fixée par les autorités monétaires.
	\end{itemize}
	
	\paragraph{Equilibre du marché de la monnaie} La condition d'équilibre est que offre ($M$) = demande ($M^d$). On a donc 
	
	$$M^d = M = k.PY$$
	
	Cette théorie de demande de monnaie permet d'expliquer le niveau des prix observé dans l'économie.
	
	Ici, $Y$ est la production de biens et services ; les facteurs de production (quantité et prix) et la technologie. $Y$ est déterminé en dehors du modèle.
	
	$M$ : l'offre de monnaie est exogène. Si les autorités monétaires augmentent $M$, l'équilibre est rompu car $M > M^d$. Supposons que $k$ (variable comportementale) est stable, ce sont alors les prix $P$ qui vont rétablir l'équilibre en augmentant.
	
	En conclusion, la monnaie est neutre à long terme, elle ne permet pas de modifier la production de biens et de services. Le seul effet à long terme est la variation des prix.
	
	\paragraph{Variation des paramètres}
	
	\begin{itemize}
		\item si $PY$ augmente :
		
		\begin{itemize}
			\item[$\Leftrightarrow$] on doit réaliser plus de transactions
			\item[$\Leftrightarrow$] on doit demander plus d'encaisses monétaires
			\item[$\Leftrightarrow$] $M^d$ augmente, en supposant que $k$ est constant et $M$ exogène.
		\end{itemize}
		
		\item si $M$ augmente, $P$ augmente
		\item si $M$ augmente, $M > M^d$, par conséquence la demande de biens et de services augmente, mais la production reste inchangée : la demande de biens est plus grande que l'offre de biens, donc le prix augmente.
	\end{itemize}
	
	\paragraph{Liens entre les deux versions de la TQM}
	
	Pour Fisher, $MV = PY$, alors que pour Marshal et Pigou, $M = kPY$. Si on soustrait, on a $V = \frac{1}{k}$.
	
	Si $k$ est élevé, une grande part du revenu est détenue en monnaie, alors la vitesse de circulation va diminuer. Si $k$ est faible, la  vitesse de circulation devra être élevée pour permettre l'ensemble des transactions.
	
	\subsection{La dichotomie classique et la neutralité à long terme de la monnaie}
	
	\dessin{2}
	
	\begin{itemize}
		\item $L$ : quantité de travail
		\item $\frac{W}{P}$ : le salaire réel
		\item $O_L$ : offre de travail (émise par les salariés)
		\item $D_L$ : demande de travail (émise par les entreprises)
	\end{itemize}
	
	
	\paragraph{Sur le marché du travail} L'offre et la demande de travail déterminent le salaire réel d'équilibre $\frac{W_0}{P_0}$ et la quantité de travail d'équilibre $L_0$.
	
	\paragraph{Sur le marché de la production} Le facteur de production $L_0$ est fixé. La technologie est exogène, donc $Y_0$ est déterminé à l'intersection de $L_0$ et $Y$. Le niveau de production est déterminé par le travail.
	
	\paragraph{Sur le marché des biens et des services} L'offre est déterminée par les facteurs de production et la technologie uniquement. Elle ne dépend pas du prix, on a donc une droite verticale en $Y_0$, $O_A$.
	
	La demande agrégée de biens et services ($D_A$) dépend du niveau général des prix, dès lors $\frac{dD_A}{dP} < 0$.
	
	On suppose que les autorités ont déterminé la masse monétaire $M_0$. Supposons qu'elles décident d'augmenter la masse monétaire à $M_1$ :
	
	\begin{itemize}
		\item les encaisses monétaires augmentent.
		\item la demande agrégées de biens et de services augmente. L'offre agrégée ne dépend pas de l'offre de travail et de la technologie, donc reste constante.
		\item le prix augmente
	\end{itemize}
	
	
	\paragraph{Courbe du salaire nominal}
	
	Plus les prix sont élevés, plus le salaire réel (le pouvoir d'achat) est faible. Si $P_0$ passe à $P_1$, $\frac{W_0}{P_0}$ descend à $\frac{W_0}{P_1}$, ce qui a pour conséquence que
	
	\begin{itemize}
		\item moins de travailleurs sont prêts à travailler pour ce prix
		\item excès de la demande sur l'offre
		\item $O_L < D_L$
		\item le salaire réel doit augmenter
		\item le prix augmente
		\item les entreprises doivent proposer un salaire nominal plus élevé, donc $W_0$ monte à $W_1$
	\end{itemize}
	
	\paragraph{Conclusion}
	
	La hausse de la quantité de monnaie n'a aucun effet sur la production de biens et services ($Y_0$), sur le salaire réel ($\frac{W_0}{P_0}$ = $\frac{W_1}{P_1}$) ou sur le niveau d'emploi ($L_1$). On a donc bien la neutralité de la monnaie à long terme. Le seul effet de l'augmentation de $M$ se traduit sur les prix.
	
	Nb : on sous-entend que $V$ est stable.
	
	\bigbreak
	\begin{center}
	-- Anciennes notes --
	\end{center}
	
	Si $\frac{W}{P}$ augmente, $O_L$ augmente et $D_L$ diminue, donc l'offre de travail est une fonction croissante du salaire réel, alors que la demande de travail est une fonction décroissante.
	
	L'équilibre du travail (offre $O_L$ = demande $D_L$) définit le salaire réel d'équilibre $\frac{W_0}{P_0}$ et la quantité d'équilibre $L_0$.
	
	La frontière technologique est représentée par une fonction de production $Y = A F(K, L)$, avec $Y$ la quantité de biens et services, $F$ la fonction, $K$ le capital, $L$ le travail et $A$ la productivité totale des facteurs (niveau technologique de notre économie). $Y_0$ est le niveau optimal de production nationale à la fois pour les entreprises et pour les consommateurs.                                                                                                                                                                            
	
	$OA$ : offre agrégée, verticale car elle ne dépend pas des prix mais bien de facteurs de production.
	
	$DA_0(M_0)$ est la demande agrégée :
	
	\begin{itemize}
		\item dépend de la masse monétaire en circulation $M_0$
		\item dépend aussi des prix
		\item[$\rightarrow$] la demande augmente quand les prix diminuent
	\end{itemize}
	
	Equilibre du marché des biens : $OA = DA$. Le niveau des prix des biens et services est déterminé par l'équilibre du marché des biens et service.
	
	Supposons que les autorités monétaires augmentent la masse monétaire de $M_0$ à $M_1$ (en vert), donc les agents économiques vont vouloir détenir plus de monnaie et échangeront cette monnaie contre des biens et services. Comme l'offre de biens et services est déterminée par des facteurs réels (donc restera intacte), cette hausse de la demande agrégée ne conduit qu'à la hausse des prix. Si les prix augmentent, le salaire réel va diminuer, l'offre de travail diminue et la demande du travail, on a un déséquilibre sur le marché du travail, un écart entre $L_E$ et $L_S$, qui exerce une pression sur le salaire nominal. Ce déséquilibre favorise l'offre, les entreprises vont augmenter les salaires minimales pour retrouver l'équilibre ($W_1$).
	
	En conclusion, la hausse de quantité de monnaie n'a eu aucune conséquence sur la production de biens et service, sur le salaire réel et sur les niveaux d'emploi : c'est la neutralité de la monnaie à long terme.
	
	\bigbreak
	\begin{center}
	-- Fin anciennes notes --
	\end{center}
	
	\bigbreak
		
	\paragraph{Application dynamique}
	Prenons l'hypothèse que $V$ est stable. Si les autorités monétaires augmentent la masse monétaire entre $t$ et $t + 1$.
	
	En $t$, $M_t V_t = P_t Y_t$, en $t + 1$, $M_{t + 1} V_{t + 1} = P_{t + 1} Y_{t + 1}$. On a $1 + \text{taux} = \frac{M_{t + 1} V_{t + 1}}{M_t V_t}$. Si on prend le logarithme :
	
	$$(\ln M_{t + 1} + \ln V_{t + 1}) - (\ln M_t + \ln V_t) = (\ln P_{t + 1} + \ln Y_{t + 1}) - (\ln P_t + \ln Y_t)$$
	$$\Leftrightarrow \underbrace{\ln M_{t + 1} - \ln M_t}_{\text{taux de variation de la masse monétaire}} + \underbrace{\ln V_{t + 1} - \ln V_t}_{0\text{ par hypothèse}} = \underbrace{\ln P_{t + 1} -\ln P_t}_{\text{taux d'inflation}} + \underbrace{\ln Y_{t + 1} - \ln Y_t}_{\text{taux de croissance du PIB réel}}$$


% 27/02/2013
\section{La théorie keynésienne de la demande de monnaie}

	Keynes remet en cause la stabilité de la demande de monnaie:  pour lui, elle est instable.
	
	\subsection{Le marché du travail et la neutralité de la monnaie}
	
	\dessinS{3}{.2}
	
	%\dessinS{4}{.2}
	
	
	Soient
	
	\begin{itemize}
		\item $L_E$ la demande de travail des entreprises
		\item $L_O$ l'offre de travail au prix $P$
		\item $L_O - L_E$ est le nombre de chômeurs (involontaires, car ils veulent bien travailler mais les entreprises ne veulent pas les embaucher au salaire $\frac{W_0}{P_0}$)
	\end{itemize}
	
	Le point de départ est l'équilibre en $(L_0, \frac{W_0}{P_0})$ et $(Y*, P_0)$. On suppose que la demande agrégée sur le marché des biens diminue (pendant la crise de 1929 par exemple) ; on passe de $D_{A_0}$ à $D_{A_1}$.  Vu que $D_{A_0} < D_{A_1}$, $D_{A_1} < O_A$, les prix doivent diminuer (de $P_0$ à $P_1$).
	
	Keynes conteste la flexibilité des salaires nominaux ; pour lui, vu qu'ils sont négociés à long terme, ils sont rigides, donc si $P$ augmente, $\frac{W_0}{P_0}$ augmente, donc $\frac{W_0}{P_0} <\frac{W_0}{P_1}$, le marché du travail est alors en déséquilibre. Les entreprises vont donc moins embaucher et la production va diminuer, de $Y*$ à $Y_C$, car elles produisent $L_E$ au lieu de $L_O$, ce qui va modifier l'offre agrégée.
	
	On a deux situations :
	
	\begin{itemize}
		\item $Y < Y*$ et un salaire nominal rigide : l'économie est en dessous de son niveau optimal et l'offre agrégée $OA$ dépend du prix. Si $P$ augmente, le salaire réel diminue (à salaire nominal constant), les entreprises ont un incitant à embaucher puisque le salaire réel $\frac{W}{P}$ est plus faible. Elles peuvent embaucher car il y a des chômeurs. La production $Y$ va donc augmenter. On a bien un effet des prix sur l'offre.
		
		\item lorsque $Y \geq Y*$ et un salaire nominal pas nécessairement rigide, si le prix augmente alors $\frac{W}{P}$ diminue. Les entreprises ont un incitant à embaucher mais elles ne peuvent plus embaucher car il n'y a pas de chômeurs. $Y$ ne change plus et donc l'offre agrégée $OA$ n'est plus dépendante du prix $P$, on retrouve donc la verticalité de $OA$, qui ne dépendra plus du prix. La masse monétaire ne doit plus être augmentée.
	\end{itemize}
	
	
%	 A court terme, les salaires vont augmenter, les entreprises vont donc moins embaucher, donc le chômage va augmenter car il y a un déséquilibre; $L_C < L_D$, le nombre de chômeurs est de $L_S - L_E$ (chômage involontaire ; les salariés aimeraient travailler au salaire en vigueur $\frac{W_0}{P_1}$, ce que les entreprises ne veulent pas). Alors qu'on aurait une renégociation du salaire nominal, il est ici rigide à la baisse à court terme. Cela a pour conséquence :
%	
%	\begin{itemize}
%		\item le chômage va durer
%		\item la production va diminuer de $Y_0$ à $Y_E$ car les entreprises produisent avec $L_E$ au lieu de $L_O$. L'offre agrégée est donc modifiée
%	\end{itemize}
%	
%	On a deux situations :
%	
%	\begin{itemize}
%		\item $Y < Y_0$ : l'économie est en dessous de son niveau optimal. L'offre agrégée $OA$ dépend du prix. Si $P$ augmente, le salaire réel diminue (à salaire nominal constant), les entreprises ont un incitant à embaucher puisque le salaire réel $\frac{W}{P}$ est plus faible. Elles peuvent embaucher car il y a des chômeurs. La production $Y$ va donc augmenter. On a bien un effet des prix sur l'offre.
%		\item lorsque $Y \geq Y_0$, si le prix augmente alors $\frac{W}{P}$ augmente. Les entreprises ont un incitant à embaucher mais elles ne peuvent plus embaucher. $Y$ ne change plus et donc l'offre agrégée $OA$ n'est plus dépendante du prix $P$, on retrouve donc la verticalité de $OA$, qui ne dépendra plus du prix. La masse monétaire ne doit plus être augmentée.
%	\end{itemize}
	
	
	On a deux réponses politiques possibles :
	
	\begin{itemize}
		\item Classique : diminuer les salaires nominaux
		\item Keynésiennes : augmenter la masse monétaire
	\end{itemize}
	La politique économique sera d'utiliser la politique monétaire, c'est-à-dire augmenter les prix en augmentant la quantité de monnaie en circulation, ce qui fera diminuer les salaires réels.
	
	Conséquences d'une augmentation de masse monétaire :
	
	\begin{itemize}
		\item (directe) augmentation des prix
		\item (indirecte) baisse du salaire réel.
	\end{itemize}
	
	Si le salaire réel diminue, les entreprises embauchent les chômeurs qui sont sur le marché du travail, donc la production augmente ainsi que la demande agrégée (car les agents économiques détiennent plus d'encaisses monétaires pour acheter des biens et services) :
	
	$$M \uparrow \Rightarrow \frac{W}{P} \downarrow \Rightarrow Y \uparrow \text{ et } DA \uparrow$$
	
	L'accroissement de la masse monétaire a permit de rétablir l'équilibre de plein emploi par l'augmentation des prix. Lorsque $Y < Y_0$, la variation de la quantité de monnaie a des effets réels. Dans ce cas, la monnaie n'est plus neutre.
	
	Remarques :
	
	\begin{itemize}
		\item On a supposé que l'économie était fermée. Si elle est ouverte, il y aura une baisse de la compétitivité du pays. Elle peut être récupérée si le taux de change est variable. En zone euro, le taux est fixe.
		\item Cela ne peut pas marcher en Belgique ou au Luxembourg car le salaire est indexé sur les prix (si les prix augmentent, le salaire aussi).
	\end{itemize}
	
	
	
	\subsection{La fonction de demande de monnaie : la préférence pour la liquidité}
	
	Innovation majeure de Keynes : introduction du taux d'intérêt dans la fonction de demande de monnaie. Il voit trois motifs pour que les agents économiques choisissent de détenir de la monnaie :
	
	\begin{enumerate}
		\item \underline{motif de transaction} (idem que les classiques) : on veut de la monnaie pour effectuer des transactions.
		
		$$\frac{M^D}{P} = L(Y)$$
		
		\begin{itemize}
			\item $M^D$ est la demande de monnaie
			\item $P$ le prix
			\item $\frac{M^D}{P}$ est le pouvoir d'achat
			\item $L$ est la fonction de préférence pour la liquidité
			\item $Y$ le revenu courant.
		\end{itemize}
			
		Si $Y$ augmente, la demande de monnaie augmente.
		
		\item \underline{motif de précaution} : les agents économiques demandent de la monnaie pour des transactions mais aussi comme épargne, pour faire face à des imprévus car l'avenir est incertain.
		
		La demande d'encaisse dépend du revenu, car il est plus facile d'économiser quand on a un revenu important.
		
		\item \underline{motif de spéculation} : la demande de monnaie dépend du taux d'intérêt.
	\end{enumerate}
	
	2 actifs monétaires sont possibles : la monnaie (pas de rémunération) et les obligations (rapportent un taux d'intérêt mais moins liquide que la monnaie).
	
	Le taux d'intérêt est le prix de la renonciation à la liquidité. Plus il est élevé, plus les agents économiques ont des obligations. Plus il est faible et plus ils ont des liquidités.
	
	Keynes pensait qu'il existait un taux d'intérêt de long terme normal. Deux situations peuvent survenir :
	
	\begin{enumerate}
		\item $r > r_{\text{normal}}$ : les agents économiques s'attendent à ce que le taux d'intérêt baisse dans le futur et à ce que le prix des obligations augmentent. Cette demande d'obligations va donc augmenter et la demande de monnaie va baisser.
		
		\item $r < r_{\text{normal}}$, les agents s'attendront à une hausse du taux d'intérêt et donc à une baisse du prix des obligations. La demande des obligations va donc baisser.
	\end{enumerate}
		
	
	Quantitative easing : aux USA, les taux sont trop bas. Les banques vont alors acheter des obligations pour relancer le marché obligataire. 
	
	La demande de monnaie est inversement proportionnelle au taux d'intérêt. On a
	
	$$\frac{M^D}{P} = L(r, Y)$$
	
	\dessinS{5}{.4}
	
	Si on prend l'inverse de la fonction et qu'on la multiplie par $Y$, on a
	
	$$\frac{PY}{M^D} = \frac{Y}{L(r, Y)}$$
	
	A l'équilibre du marché monétaire, $M = M^D$ et le membre de droite devient $\frac{PY}{M}$, qui est $V$ dans l'équation de Fisher : $MV = PY$. On a donc
	
	$$V = \frac{Y}{L(r, Y)}$$
	
	On a donc que la vitesse de circulation de la monnaie dans l'économie varie en fonction de $r$ ; si $r$ est instable, $L(r, Y)$ sera aussi instable.
	
	2 réponses à la critique selon laquelle la monnaie n'est plus vraiment une réserve de valeur :
	
	\begin{itemize}
		\item Baumol-Tobin : on ne va pas au guichet bancaire chaque fois qu'on doit faire une transaction.
		\item Tobin (1958) : introduit le motif de spéculation de la monnaie en disant que la monnaie est un actif peu risqué et donc permet d'optimiser le risque agrégé du porte-feuille d'actifs.
	\end{itemize}
	
	
\section{La version moderne de la TQM : Milton Friedman (1912-2006)}

	\subsection{La fonction de demande de monnaie de Friedman}
	
	Pourquoi les agents économiques veulent-ils avoir de la monnaie ? Friedman pose que c'est une demande d'un actif parmi d'autres actifs financiers et des actifs non financiers (bien durables, par ex un bien immobilier, un terrain).
	
%	- Contraintes de ressources : 
%	- Prix des actifs
	Il définit la contrainte de richesse globale : 
	
	\begin{itemize}
		\item la somme actualisée des revenus futurs de l'individu est utilisée pour calculer la richesse globale d'un individu au cours de sa vie.
		\item le rendement de la monnaie est comparé par rapport au rendement des autres actifs.
		\item Les goûts et les préférences de l'individu
	\end{itemize}
	
	L'allocation de la richesse globale entre ces différents actifs se fait en fonction des taux de rendement relatifs de ces différents actifs :
	
	$$\frac{M^D}{P} = f(Y^P, r_b - r_m, r_e - r_m,\pi_e - r_m, u)$$
	
	\begin{itemize}
		\item $\frac{M^D}{P}$ est la demande d'encaisse réelle
		\item  $Y^P$ le revenu permanent (valeur actualisée de tous les revenus futurs)
		\item $r_m$ le rendement attendu de la monnaie (rémunération des comptes courants + les services rendus par la monnaie, c'est-à-dire la possibilité de faire des achats)
		\item $r_b$ le rendement attendu/espéré des obligations,
		\item $r_e$ le rendement attendu des actions
		\item $\pi_e$  le taux d'inflation anticipé
		\item $u$ : autres facteurs
	\end{itemize}
	
	On a que
	
	\begin{itemize}
		\item Si $Y^P \uparrow \Rightarrow$ demande de monnaie $\uparrow$, $r_b - r_m$ ou $r_e - r_m \uparrow \Rightarrow f \downarrow$
		\item Si $\pi_e \uparrow$, les prix des biens durables augmentent, le pouvoir d'achat de la monnaie diminue et donc la demande de monnaie diminue.
		
		\item La demande de monnaie sera d'autant plus grande que le niveau de richesse $Y^P$ sera élevé, que le rendement des autres actifs sera faible et que $\pi_e$ sera faible. 
		
		\item[$\rightarrow$] les individus ajusteront l'allocation de leur richesse globale jusqu'à ce que les taux de rendement marginaux des différents actifs s'égalisent.
	\end{itemize}
		
	
	On obtient donc l'équilibre quand les taux de rendement marginaux des différents actifs sont égaux. De plus, Friedman considère que la monnaie est un substitut à des actifs financiers réels, mais aucun actif ou ensemble d'actifs ne peut être un substitut parfait à la monnaie.
	
	Si la banque centrale injecte des liquidités ($M \uparrow$) en achetant des obligations vendues par les banques, le rendement de l'actif monnaie diminue et les taux de rendement ne seront plus égaux. $r_m \neq r_b \neq r_e \neq \pi_e$, $r_m < r_b, re, \pi_e$. Les agents économiques vont alors acheter des obligations, des biens et des services et d'autres actifs financiers. Si la demande des actifs augmente, le prix de ces actifs augmente.
	

	\subsection{Comparaison Keynes/Friedman}
	
\begin{center}
 	\begin{tabular}{|c|c|}
	\hline 
	Keynes & Friedman \\ 
	\hline 
	Demande de monnaie $\frac{M^D}{P}$ instable & Demande de monnaie $\frac{M^D}{P}$ stable \\ 
	\hline 
	\specialcell{Contrôler l'offre de monnaie \\ risque d'échouer à cause \\ de cette instabilité} & \specialcell{Contrôler l'offre de monnaie \\ pour contrôler les prix} \\  
	\specialcell{$\Rightarrow$ contrôle du taux d'intérêt \\ pour contrôler l'inflation} &  \\ 
	\hline 
	\end{tabular}
 \end{center} 
 
 	Pour Keynes, la politique des taux est efficace vu que c'est $r$ qui fait varier la demande de monnaie, et donc la masse monétaire. On va donc fixer $r*$.
 	
 	Pour Friedman, la politique monétaire est monétariste, on se fixe des objectifs d'offre de monnaie, la croissance de $M$ est donc fixée à l'avance. On va donc fixer $M*$. Cette politique était correcte à son époque car la vitesse $V$ était stable.