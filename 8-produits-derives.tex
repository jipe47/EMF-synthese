% 29/04/2013
\chapter{Les produits financiers dérivés}

Produit dérivé : actif financier dont la valeur dépend de la valeur d'actifs sous-jacents

\section{Les forwards de taux d'intérêt}

Un forward est un contrat irrévocable entre 2 parties stipulant qu'une transaction financière déterminée sera réalisée à une date future fixée. C'est engagement ferme entre deux parties d'acheter ou de vendre un actif sous-jacent à un prix convenu et à une date fixée (donc avec une certaine échéance).

Un forward de taux d'intérêt implique la vente future d'un instrument de dette.

L'acheteur a ce qu'on appelle une position longue, tandis que le vendeur a une position courte.

Dans l'exemple, Axa compte sur une diminution des taux sur le temps pour avoir une augmentation des prix des actions. A l'inverse, BNP spécule sur une augmentation des taux, donc une diminution du prix des actions qu'elle a dans son porte-feuille.

Un forward de taux d'intérêt suppose la définition de plusieurs paramètres: 
\begin{itemize}
	\item Les caractéristiques de l'instrument de dette qui fera l'objet de la vente 
	\item Le montant de la dette 
	\item Le prix (le taux d'intérêt) de l'instrument de dette 
	\item La date à laquelle s'effectuera la livraison
\end{itemize}

L'exemple soulève plusieurs questions :

\begin{itemize}
	\item Pourquoi BNP Paribas accepte-t-elle de conclure ce contrat avec Axa ? 
	\item[$\rightarrow$] il y a un risque de parte car les actions sont liées au taux d'intérêt, s'il augmente le prix des actions diminuera.
	\item Comment peut-elle se couvrir contre ce risque ?
	\item[$\rightarrow$] elle peut se couvrir avec un forward avec taux d'intérêt, afin d'avoir une position courte. Grâce au forward de taux, on est sûr du prix qu'il y aura dans un an vu qu'il est fixé.
	\item Pourquoi Axa accepte-t-elle de conclure le contrat forward ?
	\item[$\rightarrow$] Le risque d'Axa est que le taux du marché diminue et que le prix de ses actions augmente.
\end{itemize}


	\subsection{Avantages et inconvénients}
	
	\begin{itemize}
		\item[+] souplesse car marché de gré à gré
		\item[-] déficit de liquidité
		\item[-] risque de défaut ou de contrepartie 
	\end{itemize}
	
\section{Les futures de taux d'intérêt}

Un contrat financier future de taux d'intérêt long terme est semblable à un forward de taux d'intérêt, dans la mesure où il spécifie qu'un titre de dette fera l'objet d'une transaction future entre 2 parties, à une date déterminée. La différence est qu'un future permet de s'affranchir des risques de liquidité et de défaut liés aux contrats forward.

La différence est qu'on se trouve sur un marché boursier, et donc les risques de liquidité et de défaut sont moindres.

Exemple : pour un future sur obligation dont le notionel (montant des obligations sous-jacentes) est de 100 000, il faut payer 149 890 avec une cote 149,89 (avec un point de base (<=> un pourcent) égal à 1000\$).

microcouverture : par un future on a ...
macrocouverture : on ne couvre pas un seul actif mais sur plusieurs types d'actif. On choisit l'actif sous-jacent qui permet au mieux de couvrir l'ensemble du porte-feuille

	\subsection{Organisation des transactions sur le marché des futures}
	\begin{itemize}
		\item Standardisation des montants des contrats et des échéances 
		\item Marché liquide 
		\item Plusieurs actifs sous-jacents peuvent indifféremment être livrés 
		\item Possibilité de traiter les futures en continu 
		\item Chambre de compensation donc pas de risque de défaut 
		\item Dépôt de marge initial ( deposit ou margin requirement) sur un compte de marge ; compte de garantie
		\item Evaluation des futures au jour le jour (aussi appelé le principe marked to market). Les gains ou les pertes alimentent le compte détenu à la chambre de compensation
		\item Si le solde d'un compte de marge passe en dessous d'un minimum préétabli, la chambre de compensation demande de procéder à un nouveau versement sur le compte (assurance contre le risque de défaut de l'agent)
		
		\item avec un futur, possibilité de conclure un contrat arrivé à échéance sans procéder à la livraison physique de l'actif sous-jacent ( position fermée)
	\end{itemize}
	
	\subsection{Couverture du risque de change}
	
	(Illustration) Le 1er mars, une entreprise américaine s'attend à recevoir 50 millions de yens. Le risque de change est une baisse du dollar par rapport au yen. Elle peut alors contracter un futur sur le yen, avec une échéance à la fin du trimestre (fin juin). La taille du contrat est de 12,5 millions de yens, elle peut vendre 4 contrats (car 4 . 12,5 millions = 20 millions). Elle rachetera alors ses futures à un taux plus bas.

	Exemple chiffré : le 1er mars, le taux de change pour un futur est 1 yen = 0.78 cents. Le 30 juin, sur le marché des changes comptant, 1 yen = 0.72 cents et le prix du futur est de 0.725 cents. Si une entreprise prend la position de vendeuse sur 4 contrats à terme, son gain sur le marché des futures est de 0.78 - 0.725 = 0.055 cents par yen. Le 30 juin, elle vendra ses yens à 0.72 cents. Grâce à l'opération de couverture, c'est comme si elle vendait ses yens le 30 juin à 0.72 + 0.055.
	
	
\section{Les options}

Les options sont des contrats qui donnent le droit à leur acheteur d'acheter 
ou de vendre l'actif sous-jacent à un prix spécifique ( le prix d'exercice ou 
strike) pendant une période donnée ou à une date donnée. 

Le vendeur de l'option est obligé de vendre ou d'acheter l'actif sous-jacent 
si l'acheteur de l'option décide d'exercer son droit d'acheter ou de vendre.

L'acheteur de l'option doit payer un montant défini ( la prime) au vendeur. 
2 types d'options: américaines et européennes (ne peut être exercé qu'à l'échéance).
Un call est un contrat qui donne à son détenteur le droit d'acheter le sousjacent au prix d'exercice fixé. 
Un put est inversement un contrat qui donne le droit à son détenteur de 
vendre l'actif sous-jacent au prix d'exercice .

	\subsection{Profils de gains et de pertes des futures et des options}
	
	En février, un agent A achète contre 2000\$ un call sur le contrat future des 
obligations du Trésor américain, échéance fin juin, montant facial 100000\$ 
et prix d'exercice 115. Option européenne. Le montant notionnel est de 100 000\$.

Supposons: à l'échéance, le future sous-jacent cote 110, donc les 
obligations sous-jacentes valent 110. L'agent A n'exerce pas son option et 
perd la prime de 2000\$.

Dans une telle situation, le prix de l'actif sous-jacent est inférieur au prix 
d'exercice= option « out of the money » 

Supposons, à la date d'échéance, le prix du future est de 115. L'agent A est 
indifférent entre exercer ou non l'option. Perte du montant de la prime. 
Option "at the money" car le prix d'exercice est égal à la quotation de l'actif sous-jacent.

Supposons, à la date d'échéance, le future cote 120. L'agent A exerce 
l'option. Gain: ((120-115) x 1000\$)- 2000\$ = 3000\$ . Option « in the money ». 

Supposons, à la date d'échéance, le future cote 125. Gain : 8000\$

Supposons qu'au lieu d'acheter l'option sur le future, l'agent A ait investi 
directement sur le future. 

Si le prix des obligations décline jusqu'à 110, le prix du futur baisse lui 
aussi à 110. Perte : 5 x 1000\$ = 5000\$ 

Si le prix des obligations vaut 115, perte ou gain nul 

Si le prix des obligations vaut 120, gain 5000\$ 

Si le prix des obligations vaut 125, gain : 10000\$

Principales différences futures/options: 
Les fonctions de gains des futures sont linéaires, contrairement à celles des 
options qui sont non linéaires (voir graphique) 

Achat d'un future : appel de marge /Achat d'une option : prime 

Future : marked to market / Option européenne : non

Illustration: se couvrir grâce aux options : les options apparaissent comme des instruments extrêmement précieux pour la macrocouverture du risque de taux encouru par les banques et les institutions financières.

CALL  PUT 
Prix du sous-jacent  +  - 
Prix d'exercice  -  + 
Temps jusqu'à échéance  +   +  
Volatilité  +  + 
Taux d'intérêt  +   - 
Dividendes attendus  -  + 

\section{Les swaps de taux d'intérêt}

Le swap le plus courant : swap de taux « vanille » . On échange les cash-flows liés aux taux d'intérêt, donc seule la charge d'intérêt est échangée, mais on n'échange pas le principal (contrairement à un swap de devises).

Dans un tel contrat, une entreprise s'engage à payer des cash-flows égaux aux intérêts à taux fixe sur un principal donné, pendant un certain nombre d'années. En retour, elle reçoit des intérêts à un taux variable sur le même principal, pendant la même durée. 

Le taux variable le plus courant dans les contrats de swap est le taux LIBOR. 

[Beaucoup de choses très intéressantes sont dans les slides et devraient être insérées à la place de cette phrase sarcastique.]

Grâce au swap, Microsoft transforme sa dette à taux variable en une dette à taux fixe de 5,1 \%.

\section{Les swaps de change}