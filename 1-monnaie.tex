\chapter{La monnaie}

\section{Les conceptions de la monnaie}

Recherche d'une explication théorique de la monnaie.

	\subsection{Du troc à la monnaie}
	
	Il y a une conception classique et néoclassique de la monnaie. Dans tous les cas, son but est de fluidifier les échanges.
	
	Le troc est l'échange de biens réels contre d'autres biens réels, sans utilisation de monnaie. L'économie du troc est ainsi non monétaire, cependant on rencontre des difficultés :
	
	\begin{enumerate}
		\item il y a la recherche d'un partenaire d'échange potentiel ;
		\item il faut une double coïncidence des besoins : la personne avec laquelle on troque doit être intéressée par ce qu'on propose et inversement ;
		\item il faut une définition des termes d'échange, autrement dit un prix.
	\end{enumerate}
	
	La monnaie permet de résoudre les deux premières difficultés et de simplifier la troisième.
	
	Par exemple, supposons que l'on ait une économie de trois biens A, B et C. Il faut définir des prix pour toutes les permutations (AB, BA, AC, CA, BC et CB), soit 6 prix (permutation, $3!$), 3 (combinaison, $C_3^2 = 3$) sans de la redondance.
	
	Sans monnaie, il faut trois définitions de terme des échanges pour une économie à trois biens. Si on introduit de la monnaie, par exemple C, on n'a que deux prix à définir : AC et BC.
	
	Définition 1 : la monnaie est un bien échangeable contre tous les autres biens. Cette action d'échangeabilité est la liquidité.
	
	Définition 2 : la monnaie est une institution sociale. Un bien acquière le statut de monnaie si tout le monde croit que ce bien est échangeable contre les autres biens, lorsqu'il y a confiance en la monnaie.
	
	Définition générale : la monnaie est le stock d'actifs qui peut être immédiatement utilisé pour réaliser des transactions.