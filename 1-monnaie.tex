\part{La monnaie}

\section{Les conceptions de la monnaie}

	\subsection{Du troc à la monnaie}
	
	Il y a une conception classique et néoclassique de la monnaie. Dans tous les cas, le but de la monnaie est de fluidifier les échanges.
	
	\textbf{Le troc} est l'échange de biens réels contre d'autres biens réels, sans utilisation de monnaie. L'économie du troc est ainsi non monétaire, cependant on rencontre des difficultés :
	
	\begin{enumerate}
		\item il y a la recherche d'un partenaire d'échange potentiel ;
		\item il faut une double coïncidence des besoins : la personne avec laquelle on troque doit être intéressée par ce qu'on propose et inversement ;
		\item il faut une définition des termes d'échange, autrement dit un prix.
	\end{enumerate}
	
	La monnaie permet de résoudre les deux premières difficultés et de simplifier la troisième.
	
	Par exemple, supposons que l'on ait une économie de trois biens A, B et C. Il faut définir des prix pour toutes les permutations (AB, BA, AC, CA, BC et CB), soit 6 prix (permutation, $3!$), 3 (combinaison, $C_3^2 = 3$) sans de la redondance.
	
	Sans monnaie, il faut trois définitions de terme des échanges pour une économie à trois biens. Si on introduit de la monnaie, par exemple C, on n'a que deux prix à définir : AC et BC.
	
	On peut imaginer trois définition de la monnaie :
	
	\begin{enumerate}
		\item la monnaie est un bien échangeable contre tous les autres biens. Cette action d'échangeabilité est la liquidité.
		
		\textbf{Un actif} est \textbf{liquide} si sa valeur nominale est stable et s'il est convertible immédiatement et sans coût en un moyen de payement.
		
		\item La monnaie est une institution sociale qui permet de fluidifier les échanges entre les acteurs de cette société. Un bien acquière le statut de monnaie si tout le monde croit que ce bien est échangeable contre les autres biens, lorsqu'il y a confiance en la monnaie.
		
		NB : l'Etat n'est pas à l'origine de la monnaie, mais un ensemble de personnes. L'Etat ne peut rien faire si la monnaie n'est pas acceptée.
		
		\item[$\rightarrow$] La monnaie est le stock d'actifs qui peut être immédiatement utilisé pour réaliser des transactions.
	\end{enumerate}
	
	% Cours 18/2/2013
		
	\subsection{La conception keynésienne}
	
	Keynes accepte la conception classique et néoclassique (fluidifier les échanges de biens et services) et y ajoute  la notion de réserve de valeur. Il suppose qu'il y a une incertitude dans l'environnement économique et le fait que La monnaie (au même titre que tous les actifs, par exemple un bien immobilier) peut transférer de la valeur au cours du temps, mais elle est érodée par l'inflation. 
	
	Face à l'incertitude, les agents économiques forment généralement une épargne de précaution, une réserve de valeur qui peut aider à financer une dépense non anticipée. La monnaie est alors demandée pour elle-même. La fonction de réserve de valeur de la monnaie a une influence sur le marché des biens et des services, jouer sur la politique monétaire peut entraîner une fluctuation de l'économie en général. Actuellement, la monnaie n'est plus vraiment utilisée comme réserve de valeur, sauf en temps de crise.
	
	\begin{itemize}
		\item[$\rightarrow$] Classique et néo-classique : monnaie de marché pour échanger des biens et services.
		\item[$\rightarrow$] Dichotomie classique entre variables réelles et variables monétaires : la monnaie n'a pas d'influence sur l'économie réelle.
		\item[$\rightarrow$] Keynes ajoute une dimension supplémentaire : la monnaie est demandée pour elle-même.
		\item[$\rightarrow$] Influence de la monnaie sur les variables réelles.
	\end{itemize}

\section{Les fonctions de la monnaie}

	\subsection{L'unité de compte}
	
	Toutes les dettes sont exprimées en unité de compte (en Belgique et en zone euro : euro). C'est en 1999 que l'euro est une unité de compte mais les monnaies nationales sont encore utilisées. C'est en 2002 que l'euro devient un intermédiaire des échanges. Cela permet de simplifier la 3ème difficulté du troc.
	
	$\rightarrow$ la monnaie établit les termes dans lesquels les prix et les dettes inscrites dans les livres de compte sont exprimées
	
	\subsection{Réserve de valeur}
	
	$\rightarrow$ La monnaie comme réserve de valeur permet le transfert de pouvoir d'achat dans le futur. 
	
	Ce transfert est imparfait car la valeur de la monnaie au cours du temps n'est pas stable : elle dépend de la confiance en la monnaie et de l'inflation. Ainsi, si l'inflation augmente, la confiance en la monnaie diminue, ce qui fait augmenter l'inflation, etc.
	
	\subsection{L'intermédiaire des échanges}
	
	La monnaie est utilisée pour faciliter les transactions de biens et de services. Cela permet de régler les 1ère et 2ème difficultés du troc.
	
\section{Les différentes formes de monnaie}

	\subsection{La monnaie marchandise}
	
	Collectivement, dans les marchés, un bien était élevé à l'état de monnaie (bétail, céréale, les métaux précieux) jusqu'à la seconde guerre mondiale.
	
	\subsection{Monnaie métallique}
		
	Elle a été
		
	\begin{itemize}
		\item pesée,
		\item comptée, et
		\item frappée, d'où l'apparition des pièces de monnaie avec une valeur faciale.
	\end{itemize}
		
	La monnaie frappée a été inventée vers -500/-600 avec JC, par Crésus (royaume de Lydie). Cela permet de différencier la valeur faciale et la valeur intrinsèque (par exemple une pièce de 5g d'or qui n'en contient que 2g). Cela permet d'augmenter la production de monnaie par l'Etat (échange de 5g d'or contre des pièces n'en contenant que 2g). La différence entre la valeur faciale et la valeur intrinsèque s'appelle le seigneuriage, c'était avant tout un moyen de prélever un impôt.
		
	Au fil du temps, cette différence augmente, il y a de moins en moins de métaux précieux dans les pièces mais la confiance reste (cela marche tant qu'il y en a) : c'est la dématérialisation de la monnaie. Vu que c'est totalement dématérialisé, les coûts de production sont faibles et on serait tenté d'inonder le marché de monnaie, au risque de rompre la confiance étable.
		
	Le bimétallisme est le fait de ne considérer que l'or et l'argent, où généralement l'argent permet d'échanger des biens et des services et l'or est utilisé comme réserve. C'est dû à la tendance de vouloir faire partir la monnaie qui vaut le moins (loi de Gresham) : la mauvaise monnaie chasse la bonne.
		
	\subsection{La monnaie de papier ou monnaie fiduciaire}
		
		Ce sont les billets de banque, où la valeur faciale diffère aussi de la valeur intrinsèque. Un billet représente une quantité d'or stocké banque, ils seront au final utilisés pour tous les transferts. La monnaie est \textbf{dématérialisée}.
		
		Jusqu'à la seconde guerre mondiale, il y avait une relation entre le papier et l'or : un billet de banque était un titre certifiant le dépôt d'une certaine quantité d'or dans une banque. Ces titres se sont échangés petit à petit jusqu'à ce qu'il n'y ait plus que ça (ce qui montre la confiance de tous les agents économiques), sans qu'on demande leur conversion en or.
		
		Panique bancaire (bank run) : s'il y a émission de plus de billets de banque qu'il n'y a d'or et que beaucoup trop de personnes viennent faire un échange de leurs billets contre de l'or.
		
	\subsection{La monnaie à cours forcé, à cours légal (fiat money)}
		
	Les autorités gouvernementales décident que la monnaie a cours forcé, c'est-à-dire que les agents économiques sont obligés d'accepter les billets comme moyen de paiement des biens et services.
		
	\subsection{La monnaie scripturale}
		
	C'est la monnaie inscrite au bilan des banques. Elle est plus sûre et pratique que la monnaie fiduciaire, il n'y a pas de transformation à effectuer.
		
	\subsection{La monnaie électronique}
		
	Moyen de stocker de la monnaie au format électronique (cartes prépayées, Proton).
		
\section{Définition technique de la monnaie}

	Comment mesure-t-on la quantité de monnaie en circulation.
	
	Il y a 3 agrégats monétaires en zone euro : M1, M2 et M3.
	
	\begin{enumerate}
		\item[M1]narrow money, la monnaie la plus liquide, qui regroupe
		
		\begin{itemize}
			\item Numéraire : billets et pièces
			\item Dépôts à vue dans les institutions financières : comptes courants
		\end{itemize}
	
		\item[M2] regroupe M1 et
		
		\begin{itemize}
			\item les dépôts à terme ($<$ 2 ans)
			\item les comptes d'épargne
		\end{itemize}
		
		\item[M3] regroupe M2 et des instruments négociables émis par les institutions financières (certificat de dépôt, dépôts à terme, obligations à échéance $<$ 2 ans).
	\end{enumerate}
	
	Plus $i$ est petit et plus M$i$ est liquide.
	
	Pour faire des transactions, il faut M1 (donc convertir M2 et M3). La banque centrale surveille surtout M3 et les taux de conversion en M1, car les prix augmenteront s'il y a une conversion soudaine de M3 en M1 (il y a plus de monnaie en circulation).
	
	Comment mesurer la quantité de monnaie dans l'économie : la monnaie est un stock d'actifs, c'est-à-dire que pour mesurer la quantité de monnaie il suffit de mesurer le stock d'actifs en circulation, donc mesurer la masse monétaire.
