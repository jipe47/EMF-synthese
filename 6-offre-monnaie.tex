% 15/4/2013
\part{Offre de monnaie}

\section{La finance intermédiée}

	\subsection{Le rôle des intermédiaires financiers}
	
	Intermédiaire financier : institution servant d'interface entre des emprunteurs et des épargnants. Il y a des intermédiaires financiers bancaires et non bancaires (ex : compagnie d'assurance).
	
	Le besoin d'intermédiaires financiers est né de l'imperfection des marchés financiers :
	
	\begin{itemize}
		\item L'information est asymétrique et incomplète pour les acteurs, les intermédiaires prennent le risque qui devrait être pris par la personne qui prête de l'argent.
	
		\item Il peut y avoir une inadéquation entre les besoins du prêteur et ceux du demandeur. Les intermédiaires s'arrangent pour que les besoins soient satisfais des deux côtés
\end{itemize}
	Le risque est à la charge des banques et intermédiaires financiers. Vu leur taille, ils ont la possibilité de diversifier les risques.
	
	
	\subsection{Le rôle de la banque}
	
	Un intermédiaire financier bancaire peut créer de la monnaie (avec un agrément auprès des institutions de régulation bancaires). Un intermédiaire non bancaire ne peut pas recevoir des dépôts bancaires pour pouvoir créer de la monnaie.
	
	La banque va créer de la monnaie en octroyant des crédits aux emprunteurs. Elle va également sélectionner des projets d'investissements. Pour faire cette sélection, elle va regarder
	
	\begin{itemize}
		\item les facteurs externes de rentabilité (conjonctions macroéconomiques)
		\item facteurs liés  au projet et à l'entreprise. Pour cela, la banque dispose d'informations privatives sur l'emprunteur, elles ne sont pas publiques.
	\end{itemize}
	
	L'actif bancaire (prêt octroyé aux ménages et aux entreprises) est très difficilement cessible sur les marchés financiers, il est illiquide car il porte sur le long terme et par le secret bancaire on ne connaît pas le risque associé. La titrisation permet de rendre cet actif plus liquide car en agrégeant beaucoup d'actifs on répartit et diversifie le risque. On peut avoir des problèmes si le modèle statistique utilisé est erroné et qu'au final le risque soit plus grand que ce qui est attendu.
	
	[note manuscrites]
	
\section{La création monétaire}

Il y a trois sources à la création de monnaie, appelées les contreparties de la masse monétaire :

\begin{itemize}
	\item contrepartie extérieure de la masse monétaire. Elle est issue des opérations de change de la banque centrale et du solde de la balance courante.
	
	\item contrepartie "créances nettes" de l'état : les banques achètent de la dette publique. La banque centrale européenne ne peut acheter de la dette des banques de l'union européenne.
	\item contrepartie "créance nette" sur l'économie (l'essentiel de la création monétaire) : ensemble des crédits octroyés aux ménages et aux entreprises.
\end{itemize}

On a l'offre de monnaie $M$ = billet et pièces $C$ + objets à vue bancaire $D$.

	\subsection{Un système bancaire avec 100\% de réserves}
	
	Supposons une économie sans manque, où $D = 0$, donc $M = C$, la masse monétaire n'est constituée que des billets et des pièces. Par exemple, prenons $M = C = 1000$ euros.
	
	Introduisons les banques mais celles-ci n'acceptent que les dépôts et n'octroient aucun crédit. Si les 1000 euros sont déposés à la banque, $M = D = 1000$ euros.
	
	Bilan de la banque :
	
	\begin{tabular}{c|c}
	Actifs & Passif \\ 
	\hline 
	Réserves, 1000 & Dépôts, 1000
	\end{tabular} 
	
	Avec 100\% de réserve, il n'y a pas de création monétaire.
	
	\subsection{Un système bancaire avec des réserves partielles}
	
	\subsubsection{Avec une seule banque}
	
	Les banques peuvent octroyer des crédits. Supposons que la banque octroie un crédit de 800 euros.
	
	\begin{tabular}{c|c}
	Actif & Passif \\ 
	\hline 
	Réserve, 200 & Dépots, 1000 \\ 
	Crédit, 800 & 
	\end{tabular} 
	
	Masse monétaire : $M = C + D = 800+  1000 = 1800$. On a donc une création monétaire de $1800 - 1000 = 800$.
	
	\begin{itemize}
		\item supposons que les 800 de liquidité sont déposés à la banque. Cette dernière n'octroie pas de crédit.
		
		\begin{tabular}{c|c}
		Actif & Dépot \\ 
		\hline 
		Réserves, 1000 & Dépots, 1800 \\ 
		Crédit, 800 &  
		\end{tabular} 
		
		La masse monétaire est de 1800 sous forme de dépôt ; on n'a pas de création monétaire, seule la composition de la masse monétaire a changé.
		
		\item Si un individu retire 200 euros, le dépôt ne s'élève plus qu'à 1600 et les réserves sont à 200. (??)
		
		\item supposons que l'emprunteur rembourse son prêt.
		
		\begin{tabular}{c|c}
		Actif & Passif \\ 
		\hline 
		Réserves, 1000 & Dépôt, 1000 \\ 
		Crédit, 0 & 
		\end{tabular}
		
		La masse monétaire est de 1000 euros ; chaque fois qu'un crédit est remboursé on a une destruction monétaire (ici de 800 euros).
	\end{itemize}
	
	La création monétaire augmente si le nombre de crédits ouverts est plus grand que le nombre de crédits remboursés.
	
	
	\subsubsection{Avec plusieurs banques}
	
	Supposons deux banques, A et B. La masse monétaire en circulation est égale à 1000 euros, déposés à la banque A. Celle-ci a un taux de réserve sur dépôt de 20\%.
	
	\begin{tabular}{c|c}
	Actif (A) & Passif (A)\\ 
	\hline 
	Réserve, 200 & Dépôts, 1000 (D1) \\ 
	Crédits, 800 &  
	\end{tabular} 
	
	Création monétaire : 1800 - 1000 = 800 euros. La masse monétaire est de 800 + 1000 = 1800.
	
	L'emprunteur dépose les 800 euros à la banque B, qui a un taux de réserve de 20\%.
	
	\begin{tabular}{c|c}
	Actif (B) & Passif (B) \\ 
	\hline 
	Réserve, 160 & Dépôts, 800 (D2) \\ 
	Crédit, 640 &  
	\end{tabular} 
	
	$M = 1800 + 640 = 2440$. La création monétaire est de 680 euros.
	
	Relations possibles entre A et B. Supposons qu'un client de la banque A retire 100 euros.
	
	\begin{tabular}{c|c}
	Actif (A) & Passif (A) \\ 
	\hline 
	Réserves, 100 & Dépôts, 900 \\ 
	Crédit, 800 &  
	\end{tabular} 
	
	Le taux de réserve est inférieur à 20\%, il y a un manque de liquidité. La banque A s'adresse à la banque B sur le marché interbancaire et emprunte 100 euros. Des opérations interbancaires (OI) s'insèrent dans le bilan (au taux EONIA) :
	
	\begin{tabular}{c|c}
	Actif (A) & Passif (A) \\ 
	\hline 
	Réserves, 200 & Dépôts, 900 \\ 
	Crédit, 800 &   OI, 100
	\end{tabular}
	
	\begin{tabular}{c|c}
	Actif (B) & Passif (B) \\ 
	\hline 
	Réserves, 60 & Dépôts, 800 \\ 
	Crédit, 640 &  \\
	OI, 100 &
	\end{tabular}
	
	Si la banque B refuse ou ne peut pas prêter le crédit, la banque A va s'adresser à la banque centrale. Elle ne s'y adresse pas directement, elle va d'abord voir chez les autres banques, car les taux d'intérêts sont plus élevés. Cet emprunt à la banque centrale appelé une opération de refinancement.
	
	\begin{tabular}{c|c}
	Actif (A) & Passif (A) \\ 
	\hline 
	Réserves, 200 & Dépôts, 900 \\ 
	Crédit, 800 &   Refinancement, 100
	\end{tabular}
	
		
	\begin{tabular}{c|c}
	Actif (B) & Passif (B) \\ 
	\hline 
	Réserves, 160 & Dépôts, 800 \\ 
	Crédit, 640 &
	\end{tabular}
	
	
	\begin{tabular}{c|c}
	Actif (BC) & Passif (BC) \\ 
	\hline 
	X, 500 & Billets, 240 \\ 
	Refinancement, 100 & Réserves, 360 \\
	\end{tabular}
	
	Les réserves de la BC sont l'addition des réserves de A et B, les dépôts des banques à la BC.
	
	Supposons que la BC émette davantage de billets en circulation.
	
	\begin{tabular}{c|c}
	Actif (C) & Passif (BC) \\ 
	\hline 
	X, + 100 & Billets, +100 
	\end{tabular}
	
	Une émission de billets entraîne une dévaluation de la monnaie et une réévaluation des monnaies étrangère : le stock n'a pas changé mais bien sa valeur.
	
	Si l'économie connaît un déficit de la balance courante (c'est-à-dire importation $>$ exportation), il y a un achat de devises auprès de la BC.
	
	\begin{tabular}{c|c}
	Actif (C) & Passif (BC) \\ 
	\hline 
	X, -100 & Billets, -100
	\end{tabular}
	
	Tant que les importations sont $>$ exportations, il y a une diminution des réserves de devises.
	
	% 17 / 4 / 2013
	
	Supposons que les 640 de la banque B sont déposés à la banque C.
	
	\begin{tabular}{c|c}
	Actif & Passif \\ 
	\hline 
	Réserves, 128 & Dépôt, 640 (D3) \\ 
	Crédits, 512 &  
	\end{tabular} 
	
	La création monétaire de la banque C est de 512 euros.
	
	NB : les réserves sont toujours positives, car il faut faire face à des demandes de retrait des clients.
	
	$$M = D1 + D2 + D2 + \dots$$
	
	On peut observer que la création monétaire est de plus en plus réduites, donc le processus n'est pas illimité. Peut-on calculer le montant de la création monétaire à partir d'une masse monétaire initiale de 1000 ?
	
	$M_{\text{initiale}}$ = 1000 (sous forme de dépôt), $rr$ (taux de réserve) = $\frac{\text{Réserves}}{\text{Dépôts}} = 20\%$.
	
	$$M = D1 + \underbrace{(1 - rr) D1}_{\text{création monétaire banque A}} + (1 - rr) D1 + (1 - rr) D3 + \dots$$
		
	C'est une série géométrique de la forme $1 + x + x^2 + \dots$. Si $\vert x \vert < 1$, alors $1 + x + x2 + \dots = \frac{1}{1 - x}$. $x = (1 - rr)$, donc $M = \frac{1}{rr} D1$.
	
	Au maximum, $M = \frac{1}{0.2} 1000 = 5000$, donc la création monétaire maximale est de 4000 pour un montant initial de 1000.
	
	$$\lim_{rr \rightarrow 0} M = + \infty$$
	$$\lim_{rr \rightarrow 1} M = D1$$
	
	Cette limite de création n'arrange pas les banques car les crédits qu'elles peuvent délivrer sont limités. Elles ont mis en place la titrisation pour dépasser cette limite.
	
	La création de monnaie n'est pas une création de richesse, mais de liquidités.
	
	\subsection{Petit modèle de création monétaire}
	
	[schéma]
	
	Demande de monnaie : $ M^d = P. L (Y, i)$ (L fonction de demande de liquidité). Si $Y$ est constant, $M = P.Y.L(i)$
	
	Demande d'espèce : $E^d = c M^d$, $c$ est la part de la demande de monnaie sous forme d'espèce.
	
	Demande de dépôt à vue : $D^d = (1 - c) M^d$
	
	Il y aura une demande de réserves de la part de banque commerciale : $R^d = \Theta D$, $\Theta$ étant le taux de réserve. Vu la troisième équation, c'est équivalent à $R^d = (1 - c)\Theta M^d$.
	
	NB : avec petite lettre c'est une fonction, sans lettre c'est le niveau d'équilibre.
	
	La demande de monnaie BC :
	
	$$\underbrace{E^d + R^d}_{\text{demande de monnaie BC}} = \underbrace{H}_{\text{offre de monnaie BC}}$$
	
	Donc
	
	$$H = c M^D + (1 - c) \Theta M^d) = (c + (1 - c) \Theta) P . Y . L(i)$$
	
	Si $c = 1$, $E^d = M^d$ et $H = P . Y . L(i)$, l'ensemble de la monnaie est détenue sous forme de billets ; l'offre de monnaie est entièrement contrôlée par la BC.
	
	Si $c = 0$, on n'a pas de billets ($E^d = 0$), tout est détenu sous forme de dépôts : $D^d = P.Y.L(i)$ et $H = \Theta . P . Y . L(i)$.
	
	Le cas général est $0 < c < 1$. La demande de monnaie BC est moins élevée que la demande de monnaie.
	
	\subsubsection{Le marché des réserves}
	
	$$H = E^d + R^d \Leftrightarrow \underbrace{H - E^d}_{\text{offre de réserve}} = \underbrace{R^d}_{\text{demande de réserves}}$$
	
	En zone euro le taux d'intérêt de ce marché est l'EONIA, aux EU c'est le Fed Funds rate. Chaque fois que l'EONIA (moyenne des taux d'intérêt des transactions d'un jour) s'écarte du taux d'intérêt directeur, la BC intervient (injection de liquidités). C'était un marché très liquide jusqu'en 2007.
	
	[graphique]
	
	\subsubsection{Le marché interbancaire}
	
	La BC intervient par une politique d'open market.
	
	[schéma]
	
	\subsubsection{Offre et la demande de monnaie et le multiplicateur monétaire}
	
	On sait que l'offre de monnaie BC est $H = (C + ( 1 - c) \Theta) P Y . L(i)$. On a que $\underbrace{H \underbrace{\frac{1}{(c + ( 1  - c) \Theta)}_{\text{multiplicateur monétaire} > 1}}}_{\text{offre de monnaie}} = \underbrace{P . Y . L(i)}_{\text{demande de monnaie}}$. Cela signifie que l'offre de monnaie est un multiple de l'offre de monnaie BC ; $H$ est appelé la base monétaire.
	
\section{L'évolution du système bancaire, stabilité financière versus concurrence}
	
	\subsection{Marché bancaire avant les années 80}

	Le marché est peu concurrentiel et très fortement réglementé.
	
	\begin{itemize}
		\item marché oligopolistique
		\item changement du marché bancaire aux frontières nationales
		\item métiers peu diversifiés, collecter de dépôts et octroyer des crédits
		\item les taux créditeurs et taux débiteurs sont réglementés
		\item politique monétaire très interventionniste
	\end{itemize}
	
	Le marché est très stable mais peu concurrencé.
	
	\subsection{Années 80 : 3D}
	
	
	
	\begin{itemize}
		\item Déréglementation : assouplissement des règles, taux libres et ouverture à la concurrence internationale. Les capitaux peuvent circuler et les prix sont libres.
		\item Décloisonnement : "tout le monde peut faire tout ce qu'il veut". Les banques n'ont plus le monopole de collecte des dépôts (si l'argent est placé ailleurs de dans une banque, ce n'est plus de la monnaie mais un titre).
		\item Désintermédiation :  les entreprises peuvent se financer directement sur le marché financier, auprès des épargnants ; le système bancaire est court-court-circuité. Il y a un changement qualitatif des actifs bancaires ; il y a moins de grandes entreprises et plus de crédit immobilier.
	\end{itemize}
	
	Les banques sont attaquées (moins de dépôts, plus de concurrence et les taux diminuent) et ne gagnent presque plus sur le long terme (la marge d'intérêt est faible, les profits baissent). Elles vont alors jouer sur le court terme et entrer dans la spéculation. Elles vont acheter et revendre des actifs sur leur propre compte ou ceux de clients.
	
	Il y a plus de concurrence mais l'instabilité financière est introduire au coeur des banques.
	
	1987 : crach boursier, retour de réglementation (Basel I, II, III) mais différente : la concurrence est gardée mais la réglementation est adaptée aux les actifs.
	
	Nouveaux produits financiers grâce aux innovations technologiques, qui renforcent également cet effet sur les banques.
	
	Il y a plus de concurrence et plus d'instabilité poussent aux marchés dérivés, où on échange du risque.